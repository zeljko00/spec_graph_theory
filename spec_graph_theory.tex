	% !TEX TS-program = pdflatex
	% !TEX encoding = UTF-8 Unicode
	
	\documentclass[11pt]{article}
	
	\usepackage[utf8]{inputenc}
	\usepackage{geometry}
	\usepackage{graphicx}
	\usepackage{booktabs}
	\usepackage{array}
	\usepackage{paralist}
	\usepackage{verbatim}
	\usepackage{subfig}
	\usepackage{amsthm}
	\usepackage{amsmath}
	\usepackage{amsfonts}
	\usepackage{graphicx}
	\usepackage[T1]{fontenc}

	
	\geometry{a4paper}
	
	\usepackage{fancyhdr}
	\pagestyle{fancy}
	\renewcommand{\headrulewidth}{0pt}
	\lhead{}\chead{}\rhead{}
	\lfoot{}\cfoot{\thepage}\rfoot{}
	
	\usepackage{sectsty}
	\allsectionsfont{\sffamily\mdseries\upshape}
	
	\usepackage[nottoc,notlof,notlot]{tocbibind}
	\usepackage[titles,subfigure]{tocloft}
	\renewcommand{\cftsecfont}{\rmfamily\mdseries\upshape}
	\renewcommand{\cftsecpagefont}{\rmfamily\mdseries\upshape}
	
	\newtheorem{definition}{Definicija}
	\newtheorem{theorem}{Teorema}
	\newtheorem*{custom_proof}{Dokaz tvrđenja}
	\newtheorem{consequence}{Posljedica}
	\newtheorem{example}{Primjer}
	
	\title{Spektralna teorija grafova}
	\author{Željko Tripić}
	\date{28.02.2025}
	
	\begin{document}
	\maketitle
	
	\section{Motivacija}
	Upotreba grafova u računarskim naukama je jako zastupljena. Bilo da se grafovi koriste za skladištenje podataka, 
	kreiranje grafova znanja za nepredna pretraživanje. Izazov prilikom rada sa grafovima u oblasti računarskih nauka je oduvijek
	bila njihova reprezentacija. Grafovi, zbog svoje strukture, nisu najpogodniji za procesiranje u računarima.
	Grafovi se najčešće reprezentuju u vidu matrica.

    ???
	
	\section{Uvod}
	Doći do svojstava grafa proučavanjem svojstava njemu pridruženih matrica. 

    ???

    dodati latinicna slova
	
	\section{Osnove teorije grafova}
	
		\begin{definition}[Graf]
		Graf je uređen par $G = (V, E)$, gdje je $V$ skup elemenata koji se nazivaju čvorovi, a $E = \{ \{v_1, v_2\} : v_1, v_2 \in V \}$ je skup grana, tj. dvočlanih neuređenih podskupova skupa čvorova.
		\end{definition}
	
        \paragraph{}
        Primjetimo da je po definiciji grafa, grana neuređeni dvočlani skup čvorova. Ukoliko uzmemo u obzir svojstva neuređenog skupa, dolazimo do zaključka da $\{v_1,v_2\} = \{v_2, v_1\}$, tj. svaka grana grafa ima dva ravnopravna zapisa.
    
        \paragraph{}
        Bez umanjenja opštosti, graf sa $n$ čvorova možemo predstavljati sa $G = (V, E)$, gdje je $V = \{1, 2, \dots, n\}$.
	
		\begin{definition}[Petlja]
		Petlja je grana koja spaja čvor grafa $G=(V,E)$ sa samim sobom, tj. grana $\{v,v\} \in E$
		\end{definition}
	
		\begin{definition}[Multigraf]
		Multigraf je graf kod kojeg dva proizvoljna čvora mogu biti povezana sa više grana, koje se u tom slučaju nazivaju višestruke grane.
		\end{definition}
	
		\begin{definition}[Težinski graf]
		Težinski graf $G =(V,E,w)$ je graf kod kojeg je svakoj grani dodijeljena težina, tj. realan pozitivan broj. Pored skupa čvorova i skupa grana, težinski graf je definisan i sa funkcijom $w: E \to \mathbb{R}^+$ koja svakoj grani dodjeljuje težinu.
		\end{definition}
	
		\begin{definition}[Usmjeren graf]
		Usmjeren graf $G =(V,E)$ je graf kod kojeg grane imaju usmjerenje, tj. orjentaciju. $E$ je u ovom slučaju skup uređenih parova čvorova $E = \{ (v_1,v_2) : v_1, v_2 \in V\}$. 
		Ukoliko $(v_1,v_2) \in E$, to znači da graf sadrži granu koja vodi od $v_1$ do $v_2$, ali ne i da postoji grana koja vodi od $v_2$ do $v_1$, što je posljedica $(v_1,v_2) \neq (v_2,v_1).$
		\end{definition}
	
		\paragraph{}
		Ukoliko drugačije nije naglašeno, grafovi sa kojima ćemo se baviti u nastavku su podrazumijevano netežinski neusmjereni grafovi koji ne sadrže petlje i višestruke grane.
	
		\begin{definition}[Susjednost čvorova]
		Za čvorove $v_1$ i $v_2$ grafa $G = (V, E)$ kažemo da su susjedni ako u okviru grafa $G$ postoji grana koja povezuje ta dva čvora. 
		\[\text{ čvorovi } v_1 \text{ i } v_2 \text{ su susjedni } \Leftrightarrow \{v_1, v_2\} \in E \]
		\end{definition}
		
		\begin{definition}[Incidentnost čvora i grane]
		Za čvor $v$ i granu $e$ grafa $G = (V, E)$ kažemo da su incidentni ako dati čvor predstavlja jedan kraj date grane.
		\[ \text{čvor } v \in V \text{i grana }\ e \in E \text{ su incidentni} \Leftrightarrow v \in e \Leftrightarrow e = \{i,j\} \land ( v = i  \lor  v = j) \] 
		\end{definition}
	
		\begin{definition}[Okolina čvora]
		Okolina čvora $v$ u oznaci $N_G(v)$ je skup svih čvorova grafa $G$ koji su susjedni sa datim čvorom $v$.
		\[N_G(v) = \{ v_1 : \{v, v_1\} \in E\}\]
		\end{definition}
	
		\begin{definition}[(Algebarski) Stepen čvora]
		Stepen (valencija) čvora $v$ u oznaci $\deg(v)$ je broj njemu susjednih čvorova, tj. kardinalnost njegove okoline.
		\[\deg(v) = |N_G(v)|\]
		\end{definition}
		
		\begin{definition}[Regularan graf]
		Graf $G=(V,E)$ je regularan sa stepenom regularnosti $x$ ako je stepen svakog njegovog čvora jednak $x$, tj. ako važi $deg(v) = x$ za $\forall v \in V$. 
		\end{definition}
	
		\begin{definition}[Podgraf]
		Neka je dat graf $G = (V, E)$. Podgraf grafa $G$ je svaki graf čiji skup čvorova je podskup skupa čvorova grafa $G$, a skup grana je podskup skupa grana grafa $G$.
		\[ G_1 = (V_1, E_1) \text{ je podgraf od } G \Leftrightarrow V_1 \subseteq V \text{ i } E_1 \subseteq E \]
		\end{definition}
	
		\begin{definition}[Šetnja i-j]
		Neka je dat graf $G = (V, E), V = \{1, 2, \dots, n\}$. Šetnja od čvora $i$ do čvora $j$ u oznaci $i-j$ je podgraf $G_1 = (V_1, E_1)$ za kojeg važi 
		\[
		 	V_1 \subseteq V \land  i \in V_1 \land j \in V_1
		\]
		\[ 
		 	E_1 = \{\{i,v_1\},\{v_1,v_2\},\dots,\{v_{k-1},v_k\},\{v_k,j\}\} \land k \geq 0
		\] 
		Čvorovi $v_1,v_2 \dots v_k \in V_1 $ ne moraju isključivo biti različiti.
			\paragraph{}
			Alternativno, šetnja $i-j$ je niz čvorova $i, v_1, v_2, \dots, v_k, j \in V$ za koje važi 
			\[
				\{i,v_1\} \in E \land \{v_k,j\} \in E \land \{v_p,v_{p+1}\} \in E \text{ za } \forall p \in \{1,2, \dots, k-1\} 
			\]
			\paragraph{}
			Dužina šetnje je broj grana u datom podgrafu/nizu.
		\end{definition}
	
		\begin{definition}[Put i-j]
		Put $i-j$ je šetnja $i-j$ kod koje se su svi čvorovi u nizu različiti, osim možda $i$ i $j$.
		\end{definition}
		
		\begin{definition}[Ciklus i]
		Ciklus je put $i-j$ kod kojeg je $i=j$.
		\end{definition}
		
		\begin{definition}[Prost put i-j]
		Prost put $i-j$ je put kod kojeg važi $i \neq j$.
		\end{definition}
	
		\begin{definition}[Povezanost čvorova]
		Za čvorove $v_1$ i $v_2$ grafa $G = (V, E)$ kažemo da su povezani ukoliko postoji $v_1-v_2$ šetnja u grafu $G$. 
		\end{definition}
	
		\begin{definition}[Komponenta povezanosti]
		Za podgraf $G_1 = (V_1, E_1)$ grafa $G = (V, E)$ kažemo da je njegova komponenta povezanosti ukoliko je svaki par čvorova podgrafa $G_1$ povezan, a niti jedan njegov čvor nije povezan sa čvorom grafa $G$ koji nije u sastavu podgrafa $G_1$. 
		\[
			\begin{split}
			&\text{podgraf } G_1 = (V_1, E_1) \text{ je komponenta povezanosti grafa } G = (V, E)  \Leftrightarrow \\
			& \forall v_1,v_2 \in V_1 \text{ važi } v_1 \text{ i } v_2 \text{ su povezani } \land \not \exists \{v_3, v_4\} \in E \text{ takva da } v_3 \in V_1 \land v_4 \in (V \setminus V_1)
			\end{split}
		\]
			\paragraph{}
			Iz navedenih definicija povezanosti čvorova i komponente povezanosti, lako se uočava da je je relacija povezanosti, u stvari, relacije ekvivalencije na skupu čvorova grafa $G$. Relacija povezanosti čvorova razbija skup $V$ na klase ekvivalencije tj. na disjunktne skupove čvorova koji odgovaraju skupovima cvorova komponenti povezanosti grafa $G$. 
		\end{definition}
	
		\begin{definition}[Povezan graf]
		Povezan graf je graf koji ima samo jednu komponentu povezanosti. 
		Alternativno, graf $G=(V,E)$ je povezan ako $\forall v_1,v_2 \in V$ postoji šetnja $v_1-v_2$
		\end{definition}
	
		\begin{definition}[Kompletan graf]
		Kompletan graf sa $n$ čvorova u oznaci $K_n$ je graf kod kojeg je svaki čvor povezan sa svim ostalim čvorovima grafa. 
		\[ 
			K_n = (V,E) \text{ gdje } |V| = n \text{ i } v_1,v_2 \in V \Rightarrow \{v_1,v_2\} \in E
		\]
		\end{definition}
		
		\begin{definition}[Put]
		Put sa $n$ čvorova je graf $P_n=(V,E)$ za koga važi $V = \{1,2, \dots ,n\}$, $E = \{\{i, i+1\} : i \in \{1,2, \dots , n-1\}\}$.
		\end{definition}
		
		\begin{definition}[Ciklus]
		Ciklus sa $n$ cvorova je graf $C_n=(V,E)$ za koga važi $V = \{1,2, \dots ,n\}$, $E = \{\{i, i+1\} : i \in \{1,2, \dots , n-1\} \} \cup \{\{1,n\}\}$.
		\end{definition}
	
		\begin{definition}[Stablo]
		Stalo je povezan graf koji ne sadrži niti jedan ciklus.
		\end{definition}
	
		\begin{definition}[Razapinjujući graf]
		Razapinjujucć graf grafa $G = (V,E)$ jeste njegov podgraf koji sadrži sve njegove čvorove.
		\[
			G' = (V, E_1) \text{ gdje je } E_1 \subseteq E \Leftrightarrow G' \text{ je razapinjujući graf} 
		\]
		\end{definition}
	
		\begin{definition}[Razapinjujuće stablo]
		Razapinjujucć stablo je razapinjujući graf koji je stablo.
		\end{definition}
	
		\begin{definition}[Kompleksnost grafa]
		Kompleksnost grafa $G$ u oznaci $k(G)$ je ukupan broj njegovih razapinjujućih stabala. 
		\end{definition}
	
	\section{Grafovi i njima pridružene matrice}
	Svakom grafu se mogu pridružiti matrica susjedstva, matrica incidencije, matrica stepeni i Laplasova matrica.
	Za spektralnu teoriju grafova od najvećeg značaja je Laplasova matrica.
	
		\subsection{Matrica susjedstva}
		Matrica susjedstva je najčešći vid reprezentacije grafa pomoću matrice. 
		Dimenzije matrice su $|V| \times |V|$ i njene vrste i kolone su indeksirane čvorovima grafa $G$.  
		Element matrice susjedsta u i-tom redu i j-toj koloni nosi informaciju o susjedstvu čvorova sa indeksima $i$ i $j$.
	
			\begin{definition}[Matrica susjedstva] 
			Neka je dat graf $G = (V, E)$, gdje je $V = \{1, 2, \dots, n\}$. Matrica $A_G = [a_{ij}]_{n \times n}$ koja odgovara grafu $G$ i koja je definisana sa
			 \[
				 a_{ij} =
				 \begin{cases} 
				 0 & \text{ako } \{i, j\} \notin E \\ 
				 1 & \text{ako } \{i, j\} \in E
				 \end{cases}
			 \]
			naziva se matrica susjedstva (eng. adjacency matrix) grafa $G$.
				\paragraph{}
				Suma svih elemenata u i-toj vrsti/koloni matrice susjedstva predstvlja stepen čvora sa indeksom $i$,
				tj. važi $\sum_{k=1}^{n}a_{ik} = \sum_{k=1}^{n}a_{ki} = deg(i) \text{ za } \forall i \in \{1,2, \dots, n\}$.
				Iz definicije matrice susjedstva neusmjerenog grafa jasno je da se radi o realnoj simetričnoj matrici, što je čini ermitskom matricom. 
			\end{definition}
	
			\begin{definition}[Matrica susjedstva težinskog grafa] 
			Matrica susjedstva $A_G$ težinskog grafa  $G = (V, E, w)$ je definisana sa
			\[
				a_{ij} =
				\begin{cases} 
				w(\{i,j\}) & \text{ako } \{i, j\} \notin E \\ 
				0 & \text{ako } \{i, j\} \in E
				\end{cases}
			\]
			\end{definition}
	
		\subsection{Matrica incidencije}
		Matrica incidencije grafa $G$ je realna matrica dimenzije $|V| \times |E|$ kod koje su vrste indeksirane čvorovima, a kolone granama grafa $G$.
		Element matrice incidencije u i-toj vrsti i j-toj koloni nosi informaciju o incidenciji čvora sa indeksom i i grane sa indeksom j.
	
			\begin{definition}[Matrica incidencije] 
			Neka je dat graf $G = (V, E)$, gdje je $V = \{1, 2, \dots, n\}$. Matrica $I_G = [b_{ij}]_{n \times m}$, gdje je $n = |V|$ i $m = |E|$, koja je definisana sa
			 \[
				 b_{ij} =
				 \begin{cases}
				 1 & \text{ako } e_j = \{i,v\} \text{, gdje je } e_j \in E \text{ grana sa indeksom j} \\
				 0 & \text{u suprotnom}
				 \end{cases}
			 \]
			naziva se matrica incidencije grafa $G$.
			\end{definition}
	
			\begin{definition}[Matrica incidencije usmjerenog grafa] 
			Neka je dat usmjeren graf $G = (V, E)$, gdje je $V = \{1, 2, \dots, n\}$. Matrica $I_G = [b_{ij}]_{n \times m}$, gdje je $n = |V|$ i $m = |E|$, koja je definisana sa
			 \[
				 b_{ij} =
				 \begin{cases}
				 1 & \text{ako } e_j = (i,v) \text{, gdje je } e_j \in E \text{ grana sa indeksom j} \\
				-1 & \text{ako } e_j = (v,i) \text{, gdje je } e_j \in E \text{ grana sa indeksom j} \\
				 0 & \text{u suprotnom}
				 \end{cases}
			 \]
			naziva se matrica incidencije usmjerenog grafa $G$.
			\end{definition}
	
			Ponekad se matrica incidencije neusmjerenog grafa definiše kao matrica incidencije usmjerenog grafa, pri čemu se granama dodijele proizvolje orjentacije. 
			Ovako definisana matrica incidencije je pogodna za dokazivanje Matrix-tree teoreme.
			Matrica incidencije nije najpogodnija za analizu putem metoda linearne algebre, jer nije kvadratna, niti simetrična matrica.
	
		\subsection{Matrica stepeni}
		Matrica stepeni (eng. degree matrix) grafa $G$ je realna dijagonalna matrice dimenzije $|V| \times |V|$ kod koje su vrste i kolone indeksirane čvorovima grafa $G$.
		Elementi na glavnoj dijagonali matrice stepeni imaju vrijednost stepena čvora koji odgovara indeksu vrste/kolone, a svi ostali elementi su jednaki nuli. 
	
			\begin{definition}[Matrica stepeni] 
			Neka je dat graf $G = (V, E)$, gdje je $V = \{1, 2, \dots, n\}$. Matrica $D_G = [d_{ij}]_{n \times n}$, gdje je $n = |V|$ , koja je definisana sa
			\[
				d[ij] = 
				 \begin{cases}
				 deg(i) & \text{ ako } i = j  \\ 
				 0  & \text{ ako } i \neq j 
				 \end{cases}
			\]
			naziva se matrica stepeni grafa $G$.
			\end{definition}

			Matrica stepeni težinskog grafa se može definisati na dva načina. Kod prvog načina na glavnoj dijagonali matrice se nalaze algebarski stepeni odgovarajućih čvorova. U ovom slučaju matrica se ne razlikuje
			od matrice odgovarajućeg neusmjerenog grafa. Dok se kod drugog načina koriste težinski stepeni čvorova.
			Algebarski stepen čvora smo već definisali, a težinski stepen čvora v je definisan sa $sum_{\{v,v_k\} \in E} w(\{v,v_k\})$ gdje $v_k$ predstavlja susjede čvora v.  
		
		\subsection{Laplasova matrica}
			Motiv za definisanje Laplasove matrice grafa je bila analogija Laplasovog linearnog operatora u domenu grafova.

			\begin{definition}[Laplasova matrica] 
			Neka je dat graf $G = (V, E)$, gdje je $V = \{1, 2, \dots, n\}$. Matrica $L_G = [l_{ij}]_{n \times n}$, definisana sa $L_G = D_G - A_G$ naziva se Laplasova matrica grafa $G$.
			Laplasova matrica uspostavlja relaciju između dvije prethodno definisane matrice, matrice susjedstva i matrice stepeni grafa $G$.
			Iz navedenog izraza se lako vidi da su elementi Laplasove matrice dati sa
			 \[
				 l_{ij} =
				 \begin{cases}
				 \deg(i)  & \text{ za } i = j \\
				 -1 & \text{ za } \{i, j\} \in E \\ 
				 0  & \text{u suprotnom}
				 \end{cases}
			 \]
			
			Na osnovu definicije Laplasove matrice, možemo zaključiti da je matrica realna i simetrična.
			Zanimljivo svojstvo Laplasove matrice je da je zbir elemenata svake vrste/kolone Laplasove matrice jednak nuli.
			\end{definition}
	
			\paragraph{}
			Iako smo već dali definiciju Laplasove matrice, ispostavlja se da ta definicija nije najpogodnija za ispitivanje njenih svojstava. Iz tog razloga dajemo i narednu karakterizaciju Laplasove matrice.
	
			\begin{definition}[Laplasova matrica]
			Neka je dat graf $G = (V, E)$, gdje je $V = \{1, 2, \dots, n\}$. 
			\[
				L_G = \sum_{\{v_1,v_2\} \in E} L_{\{v_1,v_2\}}
			\] 
			
			Laplasova matrica se može dobiti sumiranjem po svim granama grafa $G$ matrica $L_{\{v_1,v_2\}} = [l_{ij}]_{n \times n}$ definisanih sa
			\[
				 l_{ij} =
				 \begin{cases}
				 1  & \text{ za } i = j  \land i \in \{v_1,v_2\} \\
				 -1 & \text{ za } \{i,j\} = \{v_1,v_2\} \\ 
				 0  & \text{u suprotnom}
				 \end{cases}
			 \]
			Matrica $L_{\{v_1,v_2\}}$ se može posmatrati kao Laplasova matrica koja odgovara grani $\{v_1.v_2\}$ grafa $G$. Iz definicije, zaključujemo da se radi o realnoj i simetričnoj matrici.
			\end{definition}
			
			\begin{custom_proof}
			Nakon sumiranja matrica $L_{\{v_1,v_2\}}$ po svim granama grafa $G$ dobijamo:
				\begin{itemize}
				\item element na glavnoj dijagonali $l_{ii}$ će imati vrijednost $p_i$ koja odgovara broju granu kojima je dati čvor incidentan, tj. važi $l_{ii} = deg(i)$
				\item element $l_{ij}$ van glavne dijagonale ($i \neq j$) matrice će imati vrijednost -1 samo ako graf $G$ sadrži granu $\{i,j\}$, a u suprotnom vrijednost 0.  
				\end{itemize}
			\end{custom_proof}
		
			\paragraph{}
			Ni ova karakterizazija nam nije dovoljna, pa zato dajemo još jednu koju ćemo kasnije iskoristiti za dokazivanje Kirhofove teoreme.
	
			\begin{definition}[Laplasova matrica] 
			Neka je dat graf $G = (V, E)$, gdje je $V = \{1, 2, \dots, n\}$, $|E| = m$. Svakoj grani grafa $G$ dodijelimo proizvoljnu orjentaciju i formirajmo matricu incidencije $I_G$ u odnosu na izabranu orjentaciju.
			Za ovako dobijenu matricu incidencije važi
			 \[
				 L_G = I_G I_G^T
			 \]
			\end{definition}
	
			\begin{custom_proof}
			\[
				I_G I_G^T = I_{n \times m}  I^T_{m \times n}  = [b_{ij}]_{n \times n}
			\]
			\[
				 b_{ij} = \sum_{k=1}^{m} I_{ik} I_{kj}^T
			\]

			Iskoristimo svojstvo transponovane matrice $M_{ij}^T = M_{ji}$
			\[
				 b_{ij} = \sum_{k=1}^{m} I_{ik} I_{jk}
			\]

			Element $I_{ik}$ nosi informaciju o incidentnosti čvora $i$ i grane $k$, a element $I_{jk}$ nosi informaciju o incidentnosti čvora $j$ i grane $k$.
			\[
				 I_{ik}I_{jk} =
				 \begin{cases}
				 (-1) * (-1) & \text{ ako } i = j \text{ predstavlja čvor u koji ulazi grana k } \\
				 1 * 1 & \text{ ako } i = j \text{ predstavlja čvor iz kojeg polazi grana k } \\
				 -1 * 1 & \text{ ako grana k polazi od čvora j i ulazi u čvor i} \\
				 1 * (-1) & \text{ ako grana k polazi od čvora i, i ulazi u čvor i} \\
				 0 * 1 & \text{ ako čvor i nije incidentan sa granom k koja polazi iz čvora j } \\ 
				 0 * (-1) & \text{ ako čvor i nije incidentan sa granom k koja ulazi u čvor j } \\ 
				 1 * 0 & \text{ ako čvor j nije incidentan sa granom k koja polazi iz čvora i } \\ 
				 -1 * 0 & \text{ ako čvor j nije incidentan sa granom k koja ulazi u čvor i } \\ 
				 0 * 0 & \text{ ako čvorovi i i j nisu incidentni sa granom k } \\ 
				 \end{cases}
			\]

			Nakon sređivanja prethodnog izraza dobijamo sljedeće
			\[
				 b_{ij} = I_{ik}I_{jk} =
				 \begin{cases}
				 1 & \text{ ako je } i = j \text{ incidentan sa granom k, tj. ako važi }  i = j \in k \\
				 -1 & \text{ ako su čvorovi i i j incidentni sa granom k, tj. ako važi } k = \{i,j\} \in E \\
				 0 & \text{ ako čvor i ili j nije incidentan sa granom k, tj, ako važi } i \notin k \lor j \notin k 
				 \end{cases}
			\]

			Uzimajući u obzir da graf ne sadrži višestruke grane, $I_{ik}I_{jk} = -1$ za fiksno i i j je moguće samo za tačno jedno $k$. S toga, važi
			
			\[
				\sum_{k=1}^{m} I_{ik}I_{jk} =
				 \begin{cases}
				deg(i)  & \text{ za } i = j \\
				 -1  & \text{ za } \{i,j\} \in E \\
				 0 & \text{ u suprotnom }
				 \end{cases}
			 \]

			Prethodni izraz predstavlja, upravo, definiciju elementa Laplasove matrice. Uočimo, takođe, da tvrđenje važi za proizvoljnu orjentaciju grana grafa, pa samim tim važi za bilo koju
		    od mogućih $2^m$ orjentacija.
			\end{custom_proof}
	
	\section{Primjena matrice susjedstva u prebrojavanju šetnji}
		\paragraph{}
		U nauci, ali i našoj svakodnevnici često se susrećemo sa problemima koji se mogu modelovati i riješiti pomoću grafova. 
		Mnogi od tih problema se, u domenu grafova, svode na pitanje da li od nekog čvora $v_1$ možemo doći do čvora $v_2$, i ukoliko je moguće, na koliko naćina se to može postići. Primjeri takvih problema su:
				\begin{itemize}
				\item Na koliko načina je moguće poslati, tj. rutirati poruka kroz računarsku mrežu, od računara A do računara B? Računari su modelovani čvorovima grafa, a direktne veze između uređaja sa granama grafa.
				\item Na koliko načina DNK sekvenca A može da mutira u DNK sekvencu B? DNK sekvence su modelovane čvorovima, a proste mutacije supstitucije gena granama grafa. 
				\end{itemize}
	
		Ukoliko sada još podignemo kriterij i zahtijevamo rješavanje problema u određenom broju koraka, npr. da tačno $n$ uređaja posreduje u razmjeni poruke, 
		da se odigra tačno $n$ prostih mutacija gena, lako se vidi, da je u pozadini problem određivanja broja $i-j$ šetnji koje su dužine $n$.
	
		\paragraph{}
		Kako bismo odredili broj šetnji dužine $n$ između dva proizvoljna čvora grafa, pogodno je graf predstaviti pomoću matrice susjedstva. 
		Naredna teorema daje relaciju između elemenata matrice susjedstva i broja šetnji između čvorova grafa.
	
		\begin{theorem}
			Neka je dat graf $G = (V, E), V = \{1, 2, \dots, n\}$. Element matrice susjedstva $A_G$ u presjeku i-te vrste i j-te kolone označićemo sa $a_{ij}$, 
			a element m-tog stepena matrice susjedstva $A_G^m$ u presjeku i-te vrste i j-te kolone označićemo sa $a_{ij}^m$. Element $a_{ij}^m$ matrice $A_G^m$ predstavlja broj $i-j$ šetnji dužine $m$. 
		\end{theorem}
		
		\begin{custom_proof}
		    Teoremu ćemo dokazati putem matematičke indukcije. Radi jednostavnosti zapisa koristićemo $A_G = A$.
			
			\paragraph{Baza indukcije}
			Za $n=1$, jasno je da $a_{ij}^1 = a_{ij}$ predstavlja broj šetnji dužine 1 između čvorova, jer to odgovara broju grana kojima su povezani čvorovi i i j. 
			Kod grafa koji nije multigraf, broj grana između čvorova i i j jednak je vrijednosti elementa $a_{ij}$ matrice $A_G$, tj. 0 ukoliko čvorovi nisu susjedni ili 1 ukoliko su čvorovi susjedni. 
			Nije moguće imati šetnju $i-j$ dužine 1 ukoliko čvorovi i i j nisu susjedni, tj. direktno povezani granom grafa. 
			
			\paragraph{Induktivna pretpostavka}
			Pretpostavimo da tvrđenje važi za $k>1$ i proizvoljne čvorove i i j. Tada, $a_{ij}^k$ predstavlja broj šetnji dužine $k$ između čvorova i i j u grafu $G$ za $\forall i,j \in V$. 
			
			\paragraph{Induktivni korak}
			Pokažimo sada da tvrđenje važi i za $k+1$.
			Iskoristimo pravila operacije množenja matrica, da bi smo odredili vrijednost elementa $a_{ij}^{k+1}$ matrice $A_G^{k+1}$.  
			\[
				A_G^{k+1} = A_G^k * A_G
			\]
			\[
				a_{ij}^{k+1} = A_{i.}^k * A_{.j} 
			\]

			\begin{equation}
			a_{ij}^{k+1} = \sum_{p=1}^{n} a_{ip}^k * a_{pj}
			\label{eq:element}
			\end{equation}
		
			\paragraph{}
			Vrijednost elementa $a_{ij}^{k+1}$ matrice $A_G^{k+1}$ se dobija kao proizvod i-te vrste matrice $A_G^k$ i j-te kolone matrice $A_G$. 
			
			\paragraph{}
			Svaku $i-j$ šetnju dužine veće od 1 možemo predstaviti kao $i-p$ šetnju na koju nadovežemo granu $\{p,j\} \in E$ (što je ekvivalentno šetnji dužine 1). 
			Zaključujemo da se broj $i-j$ šetnji dužine $k+1$ može dobiti tako što za svaki čvor $p$ grafa $G$ prebrojimo $i-p$ šetnje dužine $k$ i dodamo ih u konačnu sumu ukoliko graf $G$ sadrži granu $\{p,j\}$,
			tj. ako su čvorovi p i j susjedni.
		
			\paragraph{}
			Ako sada uzmemo u obzir da prema induktivnoj pretpostavci $a_{ip}^k$ predstavlja broj $i-p$ šetnji dužine $k$ i da prema definiciji matrice susjedstva $a_{pj}=1$ ako graf $G$ sadrži granu $\{p,j\}$, 
			izraz $a_{ip}^k * a_{pj}$ predstavlja ukupan broj $i-j$ šetnji dužine $k+1$ koje su konstruisane na prethodno izloženi način. 
			Sumiranjem izraza po svim čvorovima grafa $G$ dobija se ukupan broj $i-j$ šetnji dužine $k+1$ 
			\[ \sum_{p=1}^{n} a_{ip}^k * a_{pj} \]. 
			Vrijednost ove sume, prema jednakosti (1), jednaka je elementu $a_{ij}^{k+1}$ matrice $A_G^{k+1}$ čime dokaz je završen.
		\end{custom_proof}
	
		\paragraph{}
		Sada, iako znamo kako da prebrojimo $i-j$ šetnje u zadatom grafu, nailazimo na problem koji proizilazi iz same prirode šetnji. Naime, definicija šetnje dozvoljava da se više puta posjeti neki od čvorova grafa.
		Na primjer, naredne šetnje su obuhvaćene prethodno izloženom teoremom a nisu prosti putevi:
		\[
			\begin{split}
			i \rightarrow v_1 \rightarrow i \\
			i \rightarrow  j \rightarrow v_1 \rightarrow j \\
			i \rightarrow j \rightarrow v_1 \rightarrow v_2 \rightarrow j \\
			i \rightarrow v_1 \rightarrow j \rightarrow v_2 \rightarrow j \\
			i \rightarrow v_1 \rightarrow i \rightarrow v_2 \rightarrow j \\
			i \rightarrow v_1 \rightarrow v_2 \rightarrow i \rightarrow j 
			\end{split}
		\]

		Bilo bi veoma korisno kada bismo na sličan način mogli prebrojati proste puteve $i-j$ dužine $n$ u zadatom grafu. 
		Ovakvo ograničenje je sasvim prirodno, jer možda, u kontekstu rutiranja poruke kroz računarsku mrežu,
		ne želimo da isti uređaj više puta posreduje u prenosu jedne poruke. 
		
		\subsection{Prebrojavanje prostih puteva}

			Po ugledu na prethodnu primjenu, matricu susjedstva ćemo iskoristiti za prebrojavanje prostih puteva.
	
			\begin{theorem}
			Neka je dat graf $G = (V, E), V = \{1, 2, \dots, n\}$. Element $s_{ij}^{(2)}$ matrice $S_G^{(2)}$ predstavlja broj prostih puteva $i-j$ dužine 2. Matrica $S_G^{(2)}$ je definisana sa
			\[
			S_G^{(2)} = A_G^{2} - D_G
			\]  
			Napomenimo da $S_G^{(2)} \neq S_G^{2}$, tj. $S_G^{(2)}$ ne predstavlja drugi stepen matrice $S_G$.
			\end{theorem}
		
			\begin{custom_proof}
			Kada su u pitanju šetnje $i-j$ dužine 2, problematične su samo one šetnje koje počinju i završavaju u istom čvoru grafa, tj. one koje predstavljaju ciklus. 
		 	Na primjer sve šetnje oblika $i \rightarrow v \rightarrow i$ gdje $v \in (V \setminus \{i\})$ su ciklusi, a obuhvaćene su prilikom prebrojavanja po prethodno izloženoj teoremi.
			Jasno je da $v \neq i$ jer razmatramo grafove bez petlji. 
			Kod ovakvih šetnji polazimo iz čvora $i$, idemo do njegovog susjeda $v$ in onda se vraćamo nazad u čvor $i$ istom granom.
			Kako su u pitanju šetnje oblika $i-i$, njihov broj je dat sa vrijednostima elemenata $a_{ii}^2$ matrice $A_G^2$ (elementi na glavnoj dijagonali matrice). 
			Jasno je da je broj ovakvih šetnji jednak broju susjeda čvora i, tj. $deg(i)$. Odavde zaključujemo da važi
				\[
				s_{ij}^{(2)} = 
				 \begin{cases}
				 0 = a_{ij}^2 - deg(i) &  \text{ za } i = j  \\ 
				 a_{ij}^2 = a_{ij}^2 - 0   & \text{ ako } i \neq j 
				 \end{cases}
				\]  
		
			Matrica definisana prethodnim izrazom je upravo matrica $S_G^{(2)}$ data u teoremi, čime je teorema dokazana.
			\end{custom_proof}
	
			\begin{theorem}
			Neka je dat graf $G = (V, E), V = \{1, 2, \dots, n\}$. Element $s_{ij}^{(3)}$ matrice $S_G^{(3)}$ predstavlja broj $i-j$ prosti puteva dužine 3. Matrica $S_G^{(3)}$ je definisana sa
			\[
			S_G^{(3)} = S_G^{(2)} A_G - A_G (D_G - I) - Diag(S_G^{(2)} A_G) 
			\]  

			gdje matrica $Diag([a_{ij}])$ definisana sa
			\[
			d_{ij} = 
			 \begin{cases}
				a_{ij} &  \text{ za } i = j  \\ 
			 	0   & \text{ za } i \neq j 
			 \end{cases}
			\]
			\end{theorem}
	
			\begin{custom_proof}
			Ideja je opet ista, prost put dužine 3 predstaviti kao prost put dužine 2 na kojeg je nadovezana još jedna grana.
			Posmatrajmo prost put $i-v$ dužine 2 na koga je nadovezana grana $\{v,j\}$. 
			Mora da važi $v \neq j$ jer u tom slučaju graf bi sadržavao petlju $\{j,j\}$ i $v \neq i$, jer u tom slučaju $i-v$ nije prost put. 
			Konstrukcija prostog puta $i-x-v-j$ je moguća samo ako su čvorovi v i j susjedni. 
			Broj tako konstruisanih puteva, gdje je v fiksan čvor, jednak je $s_{iv}^{(2)} * a_{vj}$.
			Nakon sumiranja po svim čvorovima grafa, osim i i j, dobijamo $\sum_{v \in (V \setminus \{i,j\})} (s_{iv}^{(2)} * a_{vj})$.
			Da bi ovako konstruisana šetnja $i-j$ bila prost put mora da važi da prost put $i-v$ ne posjećuje čvor j. 
			Da bismo odredili broj prostih puteva $i-j$ dužine 3 moramo da eliminišemo sve šetnje oblika $i \rightarrow j \rightarrow v \rightarrow j$. 
			Ukoliko je moguće konstruisati ovakve šetnje, njihov broj odgovara broju grana $\{j,v\}$ gdje $v \in (V \setminus \{i,j\})$. Jasno je da 
			\[
				card\{\{j,v\}:  v \in (V \setminus \{i,j\})\} =
				\begin{cases}
				deg(j) & \text{ ako } \{i,j\} \notin E  \\ 
				deg(j) -1 & \text{ ako } \{i,j\} \in E   
				\end{cases} 
			\]
		
			Uočimo da ukoliko važi $\{i,j\} \notin E$ onda šetnja $i \rightarrow j \rightarrow v \rightarrow j$ ni ne postoji.
			S obzirom na sve navedeno potrebno je isključiti $a_{ij} * (deg(j) - 1)$ šetnji iz konačnog broja.
		
			\[
			s_{ij}^{(3)} =  \sum_{v \in (V \setminus \{i,j\})} (s_{iv}^{(2)} * a_{vj}) - (a_{ij} * (deg(j)  - 1))
				       = \sum_{v \in (V \setminus \{i,j\})} (s_{iv}^{(2)} * a_{vj}) - (a_{ij} * deg(j)) - a_{ij}
			\] 
		
			Iako smo rekli da $v \neq j$ i $v \neq i$, možemo sumirati i za te vrijednosti jer u tim slučajevima $a_{vj} = a_{jj} = 0$ i $s_{iv}^{(2)} = s_{ii}^{(2)} = 0$, pa je suma ostaje ista.
			
			\[
			s_{ij}^{(3)} =  \sum_{v \in V} (s_{iv}^{(2)} * a_{vj}) - (a_{ij} * (deg(j)  - 1))
				       = \sum_{v \in V} (s_{iv}^{(2)} * a_{vj}) - (a_{ij} * deg(j)) - a_{ij}
			\] 
		 
			Uočimo da izraz pod sumom liči na množenje matrica i da izraz $a_{ij} * deg(j)$ predstavlja množenje svih elemenata u istoj koloni matrice susjedstva sa istim skalarom,
			što odgovara množenju matrice susjedstva sa dijagonalnom matricom, u ovom konkretnom slučaju, matricom stepeni. Kada prethodni izraz prikažemo u matričnom zapisu dobijamo
			\[
				\begin{split}
					S_G^{(3)} & = S_G^{(2)} A_G - A_G D_G - A_G \\
			 				  & = S_G^{(2)} A_G - A_G D_G - A_G I \\
			    			  & = S_G^{(2)} A_G - A_G (D_G - I) 
				\end{split}
			\]

			Prethodni izraz predstavlja broj prostih puteva $i-j$ dužine 3 za $i \neq j$. 
			Za $i = j$ broj prostih puteva je nula pa mora da važi $s_{ii} = 0$ za $\forall i \in V$. 
			Da bismo to postigli za elemente na glavnoj dijagonali moramo anulirati vrijednost $\sum_{v \in E} (s_{iv}^{(2)} * a_{vj}) - a_{ij} * (deg(j)  - 1)$.
			Kako je $a_{ij} = 0$ za elemente na glavnoj dijagonali matrice susjedstva, ostaje nam da anuliramo vrijednost $\sum_{v \in E} (s_{iv}^{(2)}*a_{vj}) = \sum_{v = 1}^{n} (s_{iv}^{(2)}*a_{vj}) = S_{i.}^{(2)} A_{.j}$.
			To možemo postići tako što ćemo od izraza oduzeti matricu čiji elementi na glavnoj dijagonali imaju vrijednost odgovarajućeg elementa matrice $S_G^{(2)} A_G$, a ostali elementi su jednaki 0.
			
			\[
				S_G^{(3)} = S_G^{(2)} A_G - A_G (D_G - I) - Diag(S_G^{(2)} A_G)
			\]
			\end{custom_proof}
	
			\paragraph{}
			Razmotrimo i broj prostih puteva $i-j$ dužine 4. Ponovo, polazimo od iste ideja, a to je predstaviti prost put dužine 4 preko prostog puta dužine 3 na koji je nadovezana još jedna grana.
			Posmatrajmo prost put $i-v$ dužine 3 na koga je nadovezana grana $\{v,j\}$. Naravno, $v \neq j$ jer u tom slučaju graf bi sadržavao petlju. Takođe, i $v \neq i$, jer u tom slučaju $i-v$ nije prost put.
			Konstrukcija ove šetnje je moguća samo ako su čvorovi $v$ i $j$ susjedni. Broj tako konstruisanih šetnji, gdje je v fiksan čvor, je dat izrazom
			\[
				\sum_{v \in  (V \setminus \{i,j\})} s_{iv}^{(3)} * a_{vj}
			\]
			Ponovo iz istih razloga možemo dopustiti sumiranje i po čvoru $j$.
			\[
				\sum_{v \in (V \setminus \{i\})} s_{iv}^{(3)} * a_{vj}
			\]

			\paragraph{}
			Konstruisane šetnje se mogu predstaviti sa $i \rightarrow v_1 \rightarrow v_2 \rightarrow v \rightarrow j$. 
			Da bi šetnja $i-j$ bila prost put mora da važi da prost put $i-v$ ne posjećuje čvor j.
			To znači da moramo eliminisati šetnje koje su oblika $i \rightarrow j \rightarrow v_2 \rightarrow v \rightarrow j$ i $i \rightarrow v_1 \rightarrow j \rightarrow v \rightarrow j$. 
	
			\paragraph{}
			Prosti putevi $i \rightarrow j \rightarrow v_2 \rightarrow v$ su mogući samo ako su i i j susjedni čvorovi. Za fiksan čvor v ima ih koliko i prostih puteva $j \rightarrow v_2 \rightarrow v$. 
			S tim da moramo da pazimo da $v_2 \neq i$, tj. moramo izuzeti šetnju $j-i-v$. Ne moramo da vodimo računa da li je $v=i$ jer ne vršimo sumiranje u tom slučaju.
			\[
				\sum_{v \in (V \setminus \{i\})} \{a_{vj} * [s_{iv}^{(3)} - a_{ij} * (s_{jv}^{(2)} - a_{ji}a_{iv})]\}
			\]
		
			\paragraph{}
			Sa druge strane, prostih puteva $i \rightarrow v_1 \rightarrow j \rightarrow v$ za fiksan cvor $v$ ima koliko i prostih puteva $i \rightarrow v_1 \rightarrow j$. 
			S tim da moramo da pazimo da $v_1 \neq v$. Ove šetnje su moguće samo ako su $j$ i $v$ susjedni čvorovi.
			\[
				\sum_{v \in (V \setminus \{i\})} \{a_{vj} * [s_{iv}^{(3)} - a_{ij} * (s_{jv}^{(2)} - a_{ji}a_{iv}) - a_{jv} * (s_{ij}^{(2)} -  a_{iv}a_{vj})]\}
			\]
			Jasno je da važi $a_{xy}a_{xy}=a_{xy}$ i $a_{xy}=a_{yx}$, tako da možemo ukloniti $a_{ji}$ i $a_{vj}$ iz malih zagrada, kao i $a_{jv}$ iz srednje zagrade, nakon čega dobijamo konačan izraz
			 \[
				s_{ij}^{(4)} = \sum_{v \in (V \setminus \{i\})} \{a_{vj} * [s_{iv}^{(3)} - a_{ij} * s_{jv}^{(2)} - s_{ij}^{(2)} +  a_{iv}( a_{ij} + 1)]\}
			 \]
			Zbog kompleksnosti samog izraza ovdje ćemo se zaustaviti. Evidentno je da se povećanjem dužine prostog puta povećava i kompleksnost izraza.
		
			\paragraph{}
			Maksimum broja prostih puteva između dva čvora u grafu možemo izračunati na osnovu znanja koja imamo o kompletnom grafu sa $n$ čvorova $K_n$.
			Potrebno je da izaberemo $k-1$ čvor koji će pored čvorova i i j biti sadržan u prostom putu. Prvi od $k-1$ čvorova možemo odabrati na $n-2$ načina, 
			jer ne možemo odabrati čvorove i i j. Naredni čvor pored pored i, j, ne može biti ni prethodno izabrani čvor. Istom strategijom biramo čvorove sve dok ne odaberemo svih $k-1$. 
			Ispostavlja se da je u kompletnom grafu broj prostih puteva $i-j$ jednak 
			\[
				\begin{split}
				s_{ij}^{(k)} & = (n-2)(n-2-1) \dots (n-2-(k-2)) \\
				                  & = (n-2)(n-3) \dots (n-k) \\
				                  & = \prod_{i=0}^{k-2} (n-2-i) \\ 
				                  & = \frac{(n-2)!}{(n-k-1)!}
				\end{split}
			\]
	\section{Prebrojavanje razapinjujućih stabala grafa}
	
		\paragraph{}
		Prisjetimo se rješavanja strujnih kola upotrebom Kirhofovih zakona. Prema drugom Kirhofovom zakonu zbir padova napona u zatvorenoj konturi jednak je nuli. 
		Kako bismo uopšte primijenili ovaj zakon, potrebno je odrediti odgovarajuće zatvorene konture.
		Jedan od načina da uvijek izaberemo prave konture jeste da posmatramo strujno kola kao graf u kome su provodnici grane, 
		a mjesta račvanja provodnika čvorovi i da formiramo razapinjujuće stablo takvog grafa.
		Nakon formiranja razapinjujućeg stabla, zatvorene konture konstruišemo tako što na stablo nadovežemo još jednu granu,
	    tj. provodnik, koji nije u njegovom sastavu i tako formiramo ciklus u grafu.
	
		\paragraph{}
	        S druge strane, posmatrajmo situaciju u kojoj želimo da dizajniramo računarsku mrežu koja će spajati sve veće gradove u nekoj državi. 
		Cilj je da upotrebom skupih kablova sa velikom propusnim opsegom uspostavimo direktnu ili indirektnu vezu između svih tih gradova.
		Između pojedinih gradova nije moguće uspostaviti direktnu vezu iz razloga što reljef to ne dopušta, dok je kod nekih udaljenost prevelika.
		Jasno je da će optimalna mreža imati strukturu razapinjujućeg stabla grafa. Takođe, jasno je i da razapinjujuće stablo ne mora biti jedinstveno.
	
		\paragraph{}
		Sada je već jasno da je određivanje razapinjujućeg stabla problem koji se jako često javlja. Kako razapinjujuće stablo vrlo često nije moguće jedinstveno odrediti, 
		potrebno je pronaći sva takva stabla, pa onda odrediti koje nam najvise odgovara vodeći se nekim kriterijumom odabira. 
		Ukoliko je već moguće formirati više od jednog razapinjujućeg stabla, korisno bi bilo prebrojati ih za dati graf.
		
		\paragraph{}
		Kako bismo konačno opravdali naslov rada koji ukazuje na to da ćemo koristiti spektar matrice,
		krajnje je vrijeme da uspostavimo vezu između spektra neke od grafu pridruženih 	     
		matrica i nekog njegovog svojstva. 
		S toga, navodimo narednu teoremu koja nam daje relaciju između sopstvenih vrijednosti Laplasove matrice grafa i broja njegovih razapinjujućih stabala.
	
		\begin{theorem}[Kirhofova teorema]
		Neka je dat povezan graf $G= (V,E) $ sa n čvorova, njegova Laplasova matrica $L_G$ i njene nenula sopstvene vrijednosti $\alpha_2, \alpha_3, \dots, \alpha_n$. Tada važi
		\[
			k(G) = |V|^{-1}\alpha_2 \alpha_3 \dots \alpha_n
		\]
		Kompleksnost povezanog grafa $G$, tj. broj njegovih razapinjujućih stabala jednak je količniku proizvoda nenula sopstvenih vrijednosti Laplasove matrice i broja čvorova. 
		Jasno je da se za nepovezan graf ne može formirati razapinjujuće stablo, tj. u tom slučaju $\alpha_2 = 0$ pa dobijamo $k(G) = 0$. 
		Ova teorema je još poznata i kao teorema "matrica-stablo" (eng. Matrix-Tree theorem).
		\end{theorem}
	
		\begin{custom_proof} Za dokazivanje teoreme biće nam potrebna i neke pomoćna tvrđenja koja ćemo usput navoditi.

		Neka je $M$ proizvoljna matrica dimenzije $a \times b$. Sa $M_{R,S}$ označavaćemo redukovanu matricu koja se od matrice $M$ dobije tako što se zadrže samo vrste iz skupa $R$ i kolone is skupa $S$. 
		$M_{R,:}$ je matrica koja se dobije od matrice $M$ tako što se zadrže samo vrste iz skupa $R$, a $M_{:,S}$ je matrica koja se dobije od matrice $M$ tako što se zadrže samo kolone iz skupa $S$.
		Matrica $M_{R,S}$ je dimenzije $|R| \times |S|$.
	
		Kako svako razapinjujuće stablo sadrži sve čvorove grafa i $n-1$ granu,
		jasno je da treba odrediti koliko podskupova skupa grana sa $n-1$ elementom zadovoljava uslove razapinjujućeg stabla.
		Kako je u pitanju povezan graf, jasno je i da $|E| \geq n-1$.
		Matrica incidencije je matrica koja najneposrednije sadrži informacije o granama grafa,
		s toga se ona nameće kao logičan izbor prilikom formiranja razapinjujućeg stabla grafa. Matricu incidencije ćemo formirati u odnosu na proizvoljnu orjentaciju koju ćemo dodijeliti svakoj grani grafa.
	    Iz grafa je potrebno je ukloniti odgovarajuće grane, odnosno iz matrice incidencije potrebno je ukloniti odgovarajuće kolone kako bi se formiralo razapinjujuće stablo. 
		Tako dobijenu matricu incidencije razapinjujućeg stabla možemo predstaviti sa $I_{:,S}$ gdje je $S \subseteq E$ podskup skupa grana koje ulaze u sastav stabla.
		Tako redukovana matrica incidencije je dimenzije $n \times (n - 1)$.
	
		Posmatrajmo slučaj u kome matrica $I_{:,S}$ odgovara podgrafu koji nije stablo. U tom slučaju odgovarajući podgraf sadrži ciklus. 
		Drugim riječima u okviru skupa $S$ postoji niz grana koje formiraju ciklus.
		Uočimo jedan takav ciklus i formirajmo redukovanu matricu $I_{V \setminus \{q\},S}$ uklanjanjem vrste koja predstavlja proizvoljan čvor $q$ van uočenog ciklusa.
	    Jasno je da takav čvor postoji, jer bi u suprotnom značilo da su svi čvorovi sadržani u uočenom ciklusu a to nije moguće, jer bi za to trebalo n grana, dok ih mi imamo dostupno samo $n-1$. 
		Bez umanjenja opštosti, možemo pretpostaviti da se radi o posljednjoj vrsti u matrici, jer je indeksiranje čvorova i grana prilikom formiranja matrice incidencije urađeno na proizvoljan način, 
		što znači da je uvijek moguće namjestiti da uočeni čvor q bude predstavljen posljednjom vrstom matrice incidencije. 
	    Formirana matrica je kvadratna dimenzije $n - 1$. Krećimo se sada po uočenom ciklusu u grafu i pri tome vršimo elementarne transformacije kolona matrice $I_{V \setminus \{q\},S}$,
	    na način da prolaskom granom $(i,j)$ u smjeru od i ka j, množimo odgovarajuću kolonu sa 1, a u suprotnom sa -1. Dobijenu matricu označimo sa $I_{V \setminus \{q\},S}'$.
	    Jasno je da važi $det(I_{V \setminus \{q\},S}') = \pm det(I_{V \setminus \{q\},S}$). Jasno je i da je svaki čvor sadržan u ciklusu incidentan sa dvije grane ciklusa,
		i to na način da jedna grana izlazi iz datog čvora, a druga ulazi u dati čvor. Ukoliko sada saberemo kolone koje odgovaraju svim granama sadržanim u uočenom ciklusu, dobićemo da je njihov zbir nula. 
		Drugim riječima, matrica $I_{V \setminus \{q\},S}'$ ima netrivijalnu sumu kolona koja je jednaka nuli. Prema linearnoj algebri ovo znači da matrica ima linerano zavisne kolone i da je singularna.
		
		\[det(I_{V \setminus \{q\},S}') = det(I_{V \setminus \{q\},S}) = 0\]
	
		Posmatrajmo sada slučaj u kome matrica $I_{,S}$ odgovara podgrafu koji jeste stablo. Formirajmo redukovanu matricu $I_{V \setminus \{q\},S}$ uklanjanjem posljednje vrste matrice $I_{:,S}$ koja odgovara čvoru q. 
	    Pošto smo uklonili vrstu koja odogovara čvoru q iz matrice, a q je incidentan sa barem jednom granom stabla (jer je ono povezan graf), sada imamo slučaj da bar jedna kolona matrice $I_{V \setminus \{q\}:,S}$
		ima jedan element jednak $\pm 1$, a sve ostale elemente jednake 0. Iskoristimo Laplasov metod računanja determinante matrice i izvrsimo razvoj po uočenoj koloni.
		Dobijamo $det(I_{V \setminus \{q\},S}) = \pm det(I_{V \setminus \{q,w\},S})$. Čvor w odgovara vrsti koju smo uklonili prilikom razvoja determinante.
		Jasno je da su $q$ i $w$ bili susjedni cvorovi. 
		Kako nije moguće da su q i w bili samo međusobno susjedni, mora postojati još jedan čvor koji je bio susjedan sa jednim od njih, tj. u matrici $I_{V \setminus \{q,w\},S}$ mora postojati kolona koja ima jedan element
		jednak $\pm 1$, a sve ostale jednake 0. Ponovo na isti način, upotrebom Laplasovog razvoja računamo $det(I_{V \setminus \{q,w\},S})$. 
		Ovaj iterativni postupak ponavljamo sve dok ne dođemo do determinante koja odgovara matrici dimenzije $2 \time 2$. 
		Nije moguće da dva preostala čvora budu povezana samo izmedju sebe, tj. bar jedan od njih je u izvornom grafu susjedan sa nekim od uklonjenjih čvorova.
		Ukoliko preostala dva čvora nisu susjedna, onda i drugi čvor mora biti susjedan sa nekim od uklonjenih čvorova.
		Takođe, dva preostala čvora ne mogu biti istovremeno susjedni i pri tome oba biti povezana sa nekim od prethodno uklonjenih čvorova, jer bi u tom slučaju imali ciklus.
		Na osnovu prethodnog, dolazimo do zaključka da je rezultantna matrica oblika 
		\[
		\begin{bmatrix} \pm 1 &  0 \\ \mp 1 & \pm 1 \end{bmatrix}
		\text{ ili } \begin{bmatrix} \pm 1 &  \pm 1 \\ \mp 1 & 0 \end{bmatrix}
		\text{ ili } \begin{bmatrix} \pm 1 &  \pm 1 \\ 0 &  \mp 1 \end{bmatrix}
		\text{ ili } \begin{bmatrix} 0 & \pm 1 \\ \pm 1 & \mp 1 \end{bmatrix}
		\text{ ili } \begin{bmatrix} 0 &  \pm 1 \\ \pm 1 & 0 \end{bmatrix}
		\text{ ili } \begin{bmatrix} \pm 1 &  0 \\ 0 & \pm 1 \end{bmatrix}
		\] 
		Vrijednost determinante svih ovih matrica je $\pm 1$, 
		što znači i da je $det(I_{V \setminus \{q\},S}) = \pm 1$.
	
		\paragraph{}
		Konačno dobijamo:
		\[
		 det(I_{V \setminus \{q\},S}) = 
		\begin{cases}
		\pm 1 \text{ ako skup S odgovara skupu grana koje formiraju razapinjujuce stablo grafa G} \\
		0 \text{ u suprotnom }
		\end{cases} 
		\]
		gdje je $q$ čvor predstavljen posljednom vrstom matrice $I_G$. 
		Kako bismo se rijesili $\pm 1$, kvadrirajmo determinantu.
		\[
		 det(I_{V \setminus \{q\},S})^2 = 
		\begin{cases}
		1 \text{ ako skup S odgovara skupu grana koje formiraju razapinjujuce stablo grafa G}\\
		0 \text{ u suprotnom}
		\end{cases} 
		\]
	
		Ukoliko sada izračunamo sumu po svim podskupovima grana koji sadrže tačno $n-1$ granu i formiraju razapinjujuće stablo grafa dobićemo upravo kompleksnost grafa $G$.
		\[
		\sum_{S \subseteq E \land |S| = n -1}   det(I_{V \setminus \{q\},S})^2 = k(G)
		\]
	
		Iskoristimo činjenicu da $det(M^T) = det(M)$, da $M_{R,S} = (M_{:,S})_{R,:} = (M_{R,:})_{:,S}$ i da $(M_{R,S})^T = (M^T)_{S,R}$.
		\[
		\begin{split}
		\sum_{S \subseteq E \land |S| = n -1}   det(I_{V \setminus \{q\}:,S})^2  & =
		\sum_{S \subseteq E \land |S| = n -1}   det(I_{V \setminus \{q\}:,S}) * det\big((I_{V \setminus \{q\}:,S})^T\big) \\
		&  = \sum_{S \subseteq E \land |S| = n -1}   det\big((I_{V \setminus \{q\},:})_{:,S}\big) * det\Big(\big((I_{V \setminus \{q\},:})_{:,S}\big)^T\Big)\\
		&  = \sum_{S \subseteq E \land |S| = n -1}   det\big((I_{V \setminus \{q\},:})_{:,S}\big) * det\Big(\big((I_{V \setminus \{q\},:})^T\big)_{S,:}\Big)\\
		\end{split}
		\]
	
		Prisjetimo se Cauchy-Binet formule koja tvrdi da ako su date matrice $A_{p \times q}$ i $B_{q \times p}$ i ako je $q \geq p$ onda važi
		\[
			det(AB) = \sum_{|S| = q} det(A_{\{:,S\}}) det(B_{\{S,:\}})
		\]
		gdje se sumiranje vrši po svim podskupovima skupa kolona matrice A i skupa vrsta matrice B sa kardinalnosti q.
		
		Prema ovoj formuli, za matricu $A = I_{V \setminus \{q\},:}$ dimenzija $(n - 1) \times |E|$ i matricu $B = (I_{V \setminus \{q\},:})^T$ dimenzija $|E| \times (n - 1)$, pri čemu je $|E| \geq (n-1)$, jer u pitanju
		povezan graf, važi
		\[
		det \big(I_{V \setminus \{q\},:} (I_{V \setminus \{q\},:})^T\big) = \sum_{|S| = n - 1} det\big((I_{V \setminus \{q\},:})_{:.S}\big) det\Big(\big((I_{V \setminus \{q\},:})^T\big)_{S,:}\Big) = k(G)
		\] 
		
		Iskoristimo jednu od karakterizacija Laplasove matrice $L_G = I_G I_G^T$.
		Matrica incidencije u prethodnom izrazu je kreirana prema proizvoljnoj orjentaciji koja je dodijeljena svakoj grani.
		
		\[
			L_{ij} = \sum_{k=1}^{|E|} I_{ik}I_{k,j}
		\]

		Ukoliko pak posmatramo redukovnu matricu incidencije $I_{V \setminus \{q\},:}$, imamo
		\[
		\big(I_{V \setminus \{q\},:}(I_{V \setminus \{q\},:})^T\big)_{ij} = \sum_{k=1}^{|E|} I_{ik}I_{k,j} \text{ pri cemu } i,j \in \{1,2, \dots n-1\}
		\]
	
		Množenjem redukovane matrice $I_G$ i odgovarajuće transponovane matrice kao rezultat dobijamo redukovanu Laplasovu matricu koja je dobijena izostavljanjem posljednje vrste i kolone Laplasove matrice.
		Označimo tu matricu sa $L_{V \setminus \{n\}, E \setminus \{m\}}$, gdje n odgovara posljednoj vrsti a m posljednoj koloni Laplasove matrice.
		\[
			L_{V \setminus \{n\}, E \setminus \{m\}} = I_{V \setminus \{q\},:}(I_{V \setminus \{q\},:})^T
		\]
		\[
			det(L_{V \setminus \{n\}, E \setminus \{m\}}) = k(G)
		\]

		Kako n odgovara posljednoj vrsti, a m posljednjoj koloni matrice $L_G$ jasno je da $det(L_{V \setminus \{n\}, E \setminus \{m\}}) = L_{nn} = k(G)$ gdje je $L_{nn}$ minor koji odgovara elementu $l_{nn}$ Laplasove matrice.
	
	 	Sada je još potrebno uspostaviti vezu između minora $L_{nn}$ i sopstvenih vrijednosti matrice $L_G$.
		Pošto su u pitanju sopstvene vrijednosti, jasno je da će nam polazna tačka biti karakteristična jednačina $L_G \vec{x} = \alpha \vec{x}$.

		\[
			\begin{split}
				L_G \vec{x} &= \alpha \vec{x} \\
				L_G \vec{x} - \alpha \vec{x} &= \vec{0}  \\
				(L_G - \alpha I) \vec{x} &= \vec{0}  
			\end{split}
		\]
	
		Posmatrajmo matricu $ M = L_G - \alpha I_{n \times n}$, gdje $\alpha \in \mathbb{R}$.
		\[
				 (L_G - \alpha I_{n \times n})_{ij} =
				 \begin{cases}
				 deg(i) - \alpha  & \text{ako } i = j \\
				 -1 & \text{ako } \{i,j\} \in E \\ 
				 0  & \text{u suprotnom}
				 \end{cases}
		\]

		Kako je zbir elemenata u svakoj vrsti/koloni matrice $L_G$ jednak nuli, onda je zbir elemenata u svakoj vrsti/koloni matrice $M$ jednak $-\alpha$.
		Ukoliko izvršimo elementarnu transformaciju vrsta i prvih $n-1$ vrsta saberemo u n-tu vrstu matrice, dobijamo  $M_{nj} = -\alpha$ za $ j \in \{1,2,3 \dots n\}$.
		
		Uradimo sada isto i sa kolonama. Nakon sumiranja prvih $n-1$ kolona u posljednju kolonu, dobijamo $m_{in} = -\alpha$ za $i \in \{1,2, \dots n-1\}$ i $m_{nn} = -n \alpha$.
		\[
			M' =\begin{bmatrix}
				deg(1) - \alpha & ? & ? & \dots & ? & -\alpha \\
				? & deg(2) - \alpha & ? & \dots & ? & -\alpha \\
				? & ? & deg(3) - \alpha & \dots & ? & -\alpha \\
				\vdots & \vdots & \vdots & \ddots & \vdots & -\alpha \\
				? & ? & ? & \dots &deg(n-1) - \alpha & -\alpha \\
				-\alpha & -\alpha & -\alpha & \dots &-\alpha & -n \alpha \\
		\end{bmatrix}
		\]
		Kako se radi o elementarnim transformacija jasno je da važi $det(M) = det(M')$.
		Kako svi elementi u n-toj vrsti imaju faktor $\alpha$, važi $det(M')=-\alpha det(M")$, gdje je M" data sa
		\[
			M" =\begin{bmatrix}
				deg(1) - \alpha & ? & ? & \dots & ? & -\alpha \\
				? & deg(2) - \alpha & ? & \dots & ? & -\alpha \\
				? & ? & deg(3) - \alpha & \dots & ? & -\alpha \\
				\vdots & \vdots & \vdots & \ddots & \vdots & -\alpha \\
				? & ? & ? & \dots &deg(n-1) - \alpha & -\alpha \\
				1 & 1 & 1 & \dots & 1 & n \\
				\end{bmatrix}
		\]

		Iskoristimo sada Laplasov razvoj $det(M")$ po posljednjoj koloni.
		\[ 
		\begin{split}
			det(L_G - \alpha I) & = det(M) = -\alpha det(M") \\
			                    & = -\alpha \big( (-1)^{n+n}nM"_{nn} + \sum_{k=1}^{n-1}(-1)^{n+k}(-\alpha) M"_{nk} \big)  \\
			                    & = -\alpha \big( (-1)^{n+n}nM"_{nn} + \alpha^2 \sum_{k=1}^{n-1}(-1)^{n+k}(-1)M"_{nk} \big) \\
					     	    & = (-1)^{2n+1} \alpha nM"_{nn} + \alpha^2 \sum_{k=1}^{n-1}(-1)^{n+k+1}M"_{nk}
		\end{split}
		\]
		Sa $M"_{ij}$ predstavljen je minor koji odgovara elementu $m"_{ij}$ matrice $M"$.
		Primijetimo da za minore važi $M"_{nn} = (L_G - \alpha I)_{nn}$, jer se matrice  $M"$ i $L_G - \alpha I$ razlikuju isključivo u n-toj vrsti i koloni.
		\[
		det(L_G - \alpha I) = (-1)^{2n+1} \alpha n (L_G- \alpha I)_{nn} + \alpha^2 \sum_{k=1}^{n-1}(-1)^{n+k+1}M"_{nk}
		\]
		
		$(L_G- xI)_{nn}$ je polinom stepena $n-1$, pa važi $(L_G- \alpha I)_{nn} = a_{n-1}\alpha^{n-1} + \dots + a_0$.
		Uvrštavanjem u prethodni izraz dobijamo:
		\[
			\begin{split}
				det(L_G - \alpha I) &= (-1)^{2n+1} n \alpha (a_{n-1}\alpha^{n-1} + a_{n-2}\alpha^{n-2} + \dots + a_0) + \alpha^2 \sum_{k=1}^{n-1}(-1)^{n+k+1}M"_{nk} \\
									&= (-1)^{2n+1} n (a_{n-1}\alpha^n + a_{n-2}\alpha^{n-1} + \dots + a_0 \alpha) + \alpha^2 \sum_{k=1}^{n-1}(-1)^{n+k+1}M"_{nk}
			\end{split}
		\]

		Primijetimo da je koeficijent uz $\alpha$ jednak $(-1)^{2n+1} n a_0$, gdje je $a_0$ slobodni član polinoma $(L_G- \alpha I)_{nn}$ koga dobijamo za $\alpha=0$.
		\[
		  a_0 = (L_G- 0I)_{nn} = (L_G- O)_{nn} = (L_G)_{nn}
		\]
		Konačno, koeficijent uz $\alpha$ jednak $ (-1)^{2n+1} n (L_G)_{nn} = - n (L_G)_{nn}$.
	
		Kako je $det(L_G - \alpha I)$ karakteristični polinom matrice $L_G$, i $\alpha_1 = 0$ važi za svaki graf, dobijamo:
		\[
			\begin{split}
				det(L_G - \alpha I)  & = (-1)^n \alpha (\alpha - \alpha_2) \dots (\alpha - \alpha_n) \\
									 & = (-1)^n(-1)^{n-1} \alpha_2 \dots \alpha_n x + \dots \\
									 & = (-1)^{2n-1} \alpha_2 \dots \alpha_n x + \dots \\
								     & = (-1) \alpha_2 \dots \alpha_n x + \dots 
									\end{split}
									\]

		Sada znamo da je koeficijent uz $\alpha$ dat izrazom $- \alpha_2 \dots \alpha_n$, gdje su $\alpha_i$ sopstvene vrijednosti Laplasove matrice. 
	
		Poređenjem dva dobijena izraza za koeficijent uz $\alpha$ dobijamo 
		\[
			\begin{split} 
				-n (L_G)_{nn}  = &  - \alpha_2 \dots \alpha_n \\
				(L_G)_{nn}  = & n^{-1} \alpha_2 \dots \alpha_n \\
				 L_{nn} = & n^{-1} \alpha_2 \dots \alpha_n \\ 
			\end{split} 
		\]
		Ako sada iskoristimo izraz do kojeg smo prethodno došli, $ L_{nn} = k(G)$, dolazimo do kraja dokaza $k(G) = n^{-1} \alpha_2 \dots \alpha_n$
		
	\end{custom_proof} 
		
	\section{Povezanost grafa}
	\paragraph{}
	Već smo upoznati sa pojmom povezanog grafa koji nam sugeriše da graf ima samo jednu komponentu povezanoti, tj. da postoji šetnja između svaka dva čvora u datom grafu.
	Iako je sama definicija povezanog grafa veoma jasno, često je za grafove sa većim brojem čvorova i grana uvidom u njegovu grafičku reprezentaciju ili neku od pridruženih matrica jako teško utvrditi
	da li je graf povezan. Naravno da tu nije kraj problemima i izazovima. Šta ukoliko bismo željeli da na neki način uspostavimo poredak među povezanim grafovima i odredimo koji od grafovi su bolje povezani od drugih.
	Relacija "biti bolje povezan graf" bi se mogla interpretirati i iz drugog ugla posmatranja, na način da je to graf kojeg je teže dovesi u stanje nepovezanosti.
	Ova ideja je blisko povezana sa pojmom "graph edge connectivity" koji se označava sa $\alpha(G)$ i predstavlja broj grana koje je potrebno ukloniti iz grafa kako bi graf postao nepovezan.
	
	Na prvi pogled, pomislili bismo da veći broj grana kod grafova sa istim brojem čvorova ukazuje na bolju povezanost grafa, međutim, to nije slučaj.
	Posmatrajmo sljedeći graf za koga važi $\alpha(G)=1$.

	\begin{figure}[h]
		\centering
		\includegraphics[width=0.6\textwidth]{Figure_3.png}
	\end{figure}
	
	Dodavanjem još jedne grane između čvorova 1 i 3, ne utičemo na vrijednost $\alpha(G)=1$. Razlog zbog kojeg je vrijednost $\alpha(G)$ ostala nepromijenjena je taj što povezanost grafa potiče, tj. zavisi isključivo od najslabije povezanog dijela grafa, koji je u ovom slučaju ostao netaknut.
	Primijetimo da ni vrijednost $\alpha(G)$ kod grafova sa velikim brojem čvorova i grana u velikom broju slučajeva nije očigledna.
	
	\begin{figure}[h]
		\centering
		\includegraphics[width=0.6\textwidth]{Figure_4.png}
	\end{figure}

	Klasična teorija grafova nam ne nudi rješenja za postavljene probleme i zbog toga se okrećemo spektralnoj teoriji grafova. 
	Kako bismo došli do informacija o mjeri povezanosti grafova, potrebno je prethodno ispitati odgovarajuću Laplasovu matricu $L_G$ i odrediti njen spektar.
	U nastavku ćemo detaljnije proučiti svojstva Laplasove matrice, a zatim dati nekoliko teorema koje uspostavljaju relaciju između pojedinih svojstava Laplasove matrice i povezanosti grafa.
	Napomenimo da mjera povezanosti grafa koju nam daje spektralna teorija grafova nije isto što i graph edge connectivity. Iako su ove dvije vrijednosti idejno bliske, do danas nije uspostavljena relacija između njih.
	
	\subsection{Svojstva Laplasove matrice}
	
	\begin{theorem} Laplasova matrica $L_G$ je ermitska matrica.
	\[
	L_G = L_G^*
	\] 
	\end{theorem}
	
	\begin{custom_proof}
	\[
	\text{za }\forall i,j \in \{1, 2, \dots, n\} \text{ važi } a_{ij} \in (\mathbb{N}_0 \cup \{-1\}) \Rightarrow L_G \text{ je realna matrica}
	\]
	\[
	 L_G \text{ je realna matrica} \Rightarrow \overline{L_G} = L_G 
	\]
	\[
	\text{Iz definicije Laplasove matrice se vidi da za } \forall i,j \in \{1, 2, \dots, n\} \text{ važi } a_{ij} = a_{ji} \Rightarrow L_G^T = L_G 
	\]
	\[
	  L_G^T = L_G \Rightarrow L_G \text{ je simetrična matrica}
	\]
	\[
	\overline{L_G} = L_G \land L_G^T = L_G \Rightarrow L_G^* = \overline{L_G^T} = \overline{L_G} = L_G 
	\]
	\[
	L_G^* = L_G \Rightarrow L_G\text{ je ermitska matrica}
	\]
	\end{custom_proof}
	
	\begin{theorem} Laplasova matrica $L_G$ je pozitivno poluodređena matrica. 
	\[
	  \vec{x}^TL_G\vec{x} \geq 0 \text{ za } \forall \vec{x} \in \mathbb{R}^n
	\] 
	\end{theorem}
	\begin{custom_proof}
	Neka je $\vec{x}$ proizvoljan vektor is vektorskog prostora $\mathbb{R}^n$. Iskoristićemo prvu karakterizaciju Laplasove matrice.
	\[
	\vec{x} = 
	\begin{bmatrix} 
	x_1 \\ x_2 \\ \dots \\ x_{n-1} \\ x_n
	\end{bmatrix}
	\]

	\[
	  \vec{x}^TL_G\vec{x} = \vec{x}^T (\sum_{\{v_1,v_2\} \in E} L_{\{v_1,v_2\}}) \vec{x}  =  (\sum_{\{v_1,v_2\} \in E} \vec{x}^TL_{\{v_1,v_2\}}) \vec{x}
	\]
	
	Posmatrajmo proizvod $\vec{x}^T L_{\{v_1,v_2\}}$. Rezultantna matrica je matrica vrsta dimenzije $1 \times n$.
	\[
	M = \vec{x}^T L_{\{v_1,v_2\}} = [m_{1j}]_{1 \times n}
	\]
	\[
	m_{1j} = \vec{x}^T L_{\{v_1,v_2\} . j}
	\]
	Sve kolone matrice $L_{\{v_1,v_2\}}$ su nula kolone osim kolona sa indeksom $v_1$ i $v_2$.
	\[
	 m_{1j} =
	 \begin{cases}
	 \vec{x} L_{\{v_1,v_2\} . v_1}  & \text{ako } j = v_1 \\ 
	 \vec{x} L_{\{v_1,v_2\} . v_2}  & \text{ako } j = v_2  \\ 
	 0  & \text{u suprotnom}
	 \end{cases}
	\]
	Kolona matrice $L_{\{v_1,v_2\}}$ indeksom $v_1$ ima vrijednost 1 na poziciji $v_1$, tj. $a_{v_1 v_1} = 1$, a vrijednost -1 na poziciji $v_2$, tj. $a_{v_2 v_1} = -1$ .
	Kolona matrice $L_{\{v_1,v_2\}}$ indeksom $v_2$ ima vrijednost 1 na poziciji $v_2$, tj. $a_{v_2 v_2} = 1$, a vrijednost -1 na poziciji $v_1$, tj. $a_{v_1 v_2} = -1$ .
	Svi ostali elementi matrice su 0.
	\[
	 m_{1j} =
	 \begin{cases}
	 (\vec{x}_{v_1} * a_{v_1 v_1}) + (\vec{x}_{v_2} * a_{v_2 v_1}) = \vec{x}_{v_1} - \vec{x}_{v_2}   & \text{ako } j = v_1 \\ 
	 (\vec{x}_{v_2} * a_{v_2 v_2}) + (\vec{x}_{v_2} * a_{v_1 v_2}) = \vec{x}_{v_2} - \vec{x}_{v_1}  & \text{ako } j = v_2  \\ 
	 0  & \text{u suprotnom}
	 \end{cases}
	\]
	\[
	M = 
	\begin{bmatrix} 
	0 & \dots &  \vec{x}_{v_1} - \vec{x}_{v_2} & \dots &  \vec{x}_{v_2} - \vec{x}_{v_1} & \dots
	\end{bmatrix}
	_{1 \times n}
	\]
	Posmatrajmo sada proizvod $M \vec{x}$.
	\[
	  M \vec{x} =\begin{bmatrix} 
	0 & \dots &  \vec{x}_{v_1} - \vec{x}_{v_2} & \dots &  \vec{x}_{v_2} - \vec{x}_{v_1} & \dots
	\end{bmatrix}
	_{1 \times n} \vec{x}  
	\]
	
	Rezultantna matrica je matrica vrsta dimenzije $1 \times 1$, tj. skalarna vrijednost. Matrica vrsta $M$ ima nenula vrijednost samo u kolonama $v_1$ i $v_2$.
	\[
	M \vec{x} = m_{1 v_1} * \vec{x}_{v_1} + m_{1 v_2} * \vec{x}_{v_2} =  (\vec{x}_{v_1} - \vec{x}_{v_2}) * \vec{x}_{v_1} + (\vec{x}_{v_2} - \vec{x}_{v_1}) * \vec{x}_{v_2} 
	\] 
	\[
	M \vec{x} =  \vec{x}_{v_1}^2 - \vec{x}_{v_1} * \vec{x}_{v_2} + \vec{x}_{v_2}^2 - \vec{x}_{v_1} * \vec{x}_{v_2} = \vec{x}_{v_1}^2 - 2 *\vec{x}_{v_1} * \vec{x}_{v_2} + \vec{x}_{v_2}^2 = (\vec{x}_{v_1} - \vec{x}_{v_2})^2   
	\] 

	Kada dobijeni rezultat uvrstimo u početni izraz dobijamo:
	\[
	  \vec{x}^TL_G\vec{x} = \sum_{\{v_1,v_2\} \in E} (\vec{x}_{v_1} - \vec{x}_{v_2})^2
	\]
	\[
	  (\vec{x}_{v_1} - \vec{x}_{v_2})^2 \geq 0 \text{ za } \forall v_i \in \{1,2,3 \dots n\} 
	\]

	Kako je odabrani vektor $\vec{x}$ proizvoljan, dobijamo:
	\[
	  \Rightarrow \vec{x}^TL_G\vec{x} \geq 0 \text{ za } \forall \vec{x} \in \mathbb{R}^n
	\]

	Pokazali smo da je proizvod $\vec{x}^TL_G\vec{x}$ nenegativan za svaki vektor $\vec{x}$, čime smo dokazali i da je Laplasova matrica $L_G$ pozitivno poludodređena.
	\end{custom_proof}
	
	\begin{theorem} Sve sopstvene vrijednosti Laplasove matrice su nenegativni realni brojevi.
	\[
	 L_G\vec{x} = \alpha \vec{x} \text{ gdje je } \vec{x} \in ( \mathbb{C}^n \setminus \{\vec{0}\}) \text{ nenula vektor} \Rightarrow \alpha \in (\mathbb{R}^+ \cup \{0\})
	\]
	\end{theorem}
	
	\begin{custom_proof}
	Iako je data teorema direktna posljedica svojstava Laplasove matrice, u nastavku će biti izložen dokaz ovog tvrđenja.
	
	Pretpostavimo da je $\alpha \in \mathbb{C}$ proizvoljna sopstvena vrijednost matrice $L_G$ i da je $\vec{x} \in (\mathbb{C}^n \setminus \{\vec{0}\})$ odgovarajući sopstveni vektor. Tada važi:
	\[
		L_G\vec{x} = \alpha \vec{x}
	\]
	Pomnožimo izraz s lijeva vektorom $\vec{x}^T$.
	\[
		\vec{x}^TL_G\vec{x} = \vec{x}^T\alpha \vec{x}
	\]
	\[
		\vec{x}^TL_G\vec{x} = (\vec{x}^T \vec{x}) \alpha
	\]
	\[
		\vec{x}^TL_G\vec{x} = \sum_{i=1}^{n}(Re(x_i)^2 + Im(x_i)^2 ) \alpha =  \sum_{i=1}^{n}|x_i|^2 \alpha= ||\vec{x}||_2^2 \alpha  
	\]
	\begin{equation}
		 \alpha =   \frac{\vec{x}^TL_G\vec{x}}{||\vec{x}||_2^2}
	\label{eq:alpha}
	\end{equation}

	Izraz $||\vec{x}||_2^2$ predstavlja kvadrat euklidske vektorkse norme u prostoru $\mathbb{R}^n$, a vektorska norma je uvijek nenegativan realan broj, tj. važi
	\[
		||\vec{x}||_2^2 \in (\mathbb{R}^+ \cup \{0\})  
	\]
	U našem slučaju $\vec{x}$ je nenula vektor pa važi još strožiji uslov $||\vec{x}||_2^2 > 0$, koji nam je omogućio da se vektorska norma nađe u imeniocu razlomka.
	
	Sada je još potrebno pokazati da je proizvod $\vec{x}^TL_G\vec{x}$ nenegativan broj.

	To ne moramo ni dokazivati, jer već znamo da je Laplasova matrica pozitivno poludodređena.
	\[
		\vec{x}^* L_G \vec{x} = \vec{x}^T L_G \vec{x} \geq 0
	\]
	Za proizvoljan kompleksan broj $z \in \mathbb{C}$ važi
	\[
	 \overline{z} = z \Rightarrow z \in \mathbb{R}
	\]

	Dobijamo da $\vec{x}^*L_G\vec{x} \in (\mathbb{R} \cup \{0\})$. Kako su i imenilac i brojilac na desnoj strani jednakosti (\ref{eq:alpha}) realni nenegativni brojevi i lijeva strana jednakosti mora biti nenegativan realan broj, tj. $\alpha \in \mathbb{R}$.

	\[
		\vec{x} \in \mathbb{R}^n \Rightarrow \vec{x}^* = \vec{x}^T
	\]

	\end{custom_proof}
	
	\begin{consequence}
	S obzirom da su sve sopstvene vrijednosti matrice Laplasove matrice dimenzije $n \times n$ nenegativni realni brojevi i da sopstvenih vrijednosti ima tačno $n$
	(računajući i njihove višestrukosti), možemo uspostaviti sljedeći poredak između sopstvenih vrijednosti Laplasove matrice 
	\[
	0 \leq \alpha_1 \leq \alpha_2 \leq \dots \leq \alpha_{n-1} \leq \alpha_n
	\]
	\end{consequence}
	
	\begin{theorem} Nula je najmanja sopstvena vrijednost Laplasove matrice.
	\end{theorem}
	
	\begin{custom_proof}
	Posmatrajmo proizvod Laplasove matrice i jediničnog vektora $\vec{x}=\begin{bmatrix} 1 & 1 & \cdots & 1 & 1  \end{bmatrix}^T_{n \times n}$.
	\[
	L_G\vec{x} = [a_{ij}]_{n \times 1}
	\]
	\[
		a_{ij} = L_{i.}\vec{x} = \sum_{j=1}^{n}L_{ij} \vec{x}_j 
	\]
	Ukoliko sada iskoristimo činjenicu da su sve koordinate vektora $\vec{x}$ jednake 1 i da je zbir elemenata svake vrste/kolone Laplasove matrice jednak 0, dobijamo
	\[
		a_{ij} = \sum_{j=1}^{n}L_{ij} \vec{x}_j = \sum_{j=1}^{n}L_{ij} * 1 = \sum_{j=1}^{n}L_{ij} = 0
	\]
	Sada je jasno da je svaka koordinata rezultujuće vrsta matrica jednaka 0, tj. da je rezultat množenja Laplasove matrice i jediničnog vektora nula vektor.
	\[
	L_G\vec{x} = \vec{0} = 0 * \vec{x}
	\]
	Kako vektor $\vec{x}$ nije nula vektor zaključujemo da je $\alpha = 0$ sopstvena vrijednost proizvoljne Laplasove matrice.
	\[
		0 \leq \alpha_1 \leq \alpha_2 \leq \dots \leq \alpha_{n-1} \leq \alpha_n \land 0 \in \{\alpha_1, \alpha_2, \dots, \alpha_n\} \Rightarrow \alpha_1 = 0
	\]
	\end{custom_proof}
	
	\begin{theorem} Druga najmanja sopstvena vrijednost Laplasove matrice povezanog grafa je veća od nule.
	\paragraph{}
	Navedena tvrdnja je ekvivalentna tvrdnji da je algebarska višestrukost nule kao sopstvene vrijednosti Laplasove matrice povezanog grafa jednaka jedan.
	\end{theorem}
	
	\begin{custom_proof}
	Neka je dat proizvoljan graf $G=(V,E)$ i neka je $\vec{z} \in \mathbb{R}^n$ sopstveni vektor Laplasove matrice koji odgovara sopstvenoj vrijednosti $\alpha_1 = 0$.
	Pokušajmo odrediti neka svojstva vektora $\vec{z}$.
	\[
		L_G\vec{z} = 0 * \vec{z} = \vec{0}
	\]
	Pomnožimo izraz s lijeva vektorom $\vec{z}^T$.
	\[
		\begin{split}
			\vec{z}^T L_G \vec{z} = \vec{z}^T * \vec{0} = &0 \\
			\sum_{\{v_1,v_2\} \in E} (z_{v_1} - z_{v_2})^2 &= 0
		\end{split}
	\]
	
	Zaključujemo da su koordinate sopstvenog vektora $\vec{z}$, koje odgovaraju susjednim čvorovima grafa G, jednake.
	S obzirom da je graf G povezan, tj. da po definiciji povezanog grafa postoji šetnja između bilo koja dva čvora u grafu,
	dolazimo do zaključka da su sve koordinate sopstvenog vektora $\vec{z}$ jednake. 
	\[
	\begin{split}
	 \vec{z} \text{ je sopstveni vektor koji odgovara sopstvenoj vrijednosti } & \alpha_1 = 0 \text{ matrice }  L_G \Rightarrow \\
	 \vec{z} =p *
	\begin{bmatrix} 
	1 \\ 1 \\ \vdots \\ 1 \\ 1
	\end{bmatrix}_{n \times 1}
	\text{ gdje je } p \in (\mathbb{R} \setminus \{\vec{0}\})
	\end{split}
	\]

	Kako smo odredili opšti izraz sopstvenog vektora Laplasove matrice koji odgovara sopstvenoj vrijednosti $\alpha_1 = 0$,
	mi smo odredili skup svih sopstvenih vektora koji odgovaraju datoj sopstvenoj vrijednosti.
	
	\[
	S_{\alpha_1} = \{p *
	\begin{bmatrix} 
	1 \\ 1 \\ \vdots \\ 1 \\ 1
	\end{bmatrix}_{n \times 1} : p \in (\mathbb{R} \setminus \{0\}) \} 
	\]
	Ukoliko dozvolimo da $p=0$, tj. odradimo uniju skupa sopstvenih vektora $S_{\alpha_1}$ i skupa koji sadrži nula vektor,
	dobićemo sopstveni vektorski prostor koji odgovara $\alpha_1$.
	\[
	U_{\alpha_1} = S_{\alpha} \cup \{\vec{0}\}= \{
	\begin{bmatrix} 
	1 \\ 1 \\ \vdots \\ 1 \\ 1
	\end{bmatrix}_{n \times 1}  : p \in (\mathbb{R} \setminus \{0\}) \} \cup \{\vec{0}\}= \{
	\begin{bmatrix} 
	1 \\ 1 \\ \vdots \\ 1 \\ 1
	\end{bmatrix}_{n \times 1} :  p \in \mathbb{R}\} 
	\]
	Dobijeni izraz predstavlja ništa drugo do linearnu kombinaciju jednog vektora, tj. linearni vektorski omotač nad jednim vektorom.
	\[
	U_{\alpha_1} = S_{\alpha_1} \cup \{\vec{0}\}= \{
	\begin{bmatrix} 
	1 \\ 1 \\ \vdots \\ 1 \\ 1
	\end{bmatrix}_{n \times 1} :  p \in \mathbb{C}\} = L (\begin{bmatrix} 
	1 \\ 1 \\ \vdots \\ 1 \\ 1
	\end{bmatrix}_{n \times 1})
	\]

	Kako je sopstveni vektorski prostor jednak linearnom vektorskom omotaču nad jednim vektorom, 
	očito da se baza sopstvenog vektorskog prostora koji odgovara $\alpha_1$ se sastoji od jednog vektora,
	što nadalje znači da je geometrijska višestrukost $\alpha_1$ jednaka 1.
	Geometrijska višestrukost je uvijek manja ili jednaka algebarskoj višestrukosti sopstvene vrijednosti,
	pa dolazimo do zaključka da sopstvena vrijednost $\alpha_1$ ima algebarsku višestrukost 1.
	\[
	\alpha_1 = 0 \text{ ima algebarsku višestrukost 1 } \Rightarrow \alpha_1 \neq \alpha_2
	\]
	\[
	\alpha_1 = 0  \land \alpha_1 \neq \alpha_2 \land \alpha_1 \leq \alpha_2 \Rightarrow \alpha_2 > \alpha_1 \Rightarrow \alpha_2 > 0
	\]
	Ovime je dokaz završen.
	\end{custom_proof}
	
	\begin{theorem} 
	Geometrijska višestrukost sopstvene vrijednosti $\alpha_1 = 0$ Laplasove matrice predstavlja broj povezanih komponenti u grafu.
	\end{theorem}
	
	\begin{custom_proof}
	Pretpostavimo da graf $G = (V, E)$, gdje je $V = \{1, 2, \dots, n\}$, ima k povezanih komponenti. Neka su $G_i = \{V_i,E_i\}$, $i \in \{1,2, \dots n\}$ te povezane komponente.

	Za svaku od povezanih komponenti konstruišimo vektor $\vec{w_i}$ takav da njegove koordinate čiji se indeksi nalaze u skupu čvorova odgovarajuće povezane komponente imaju vrijednost 1, a u suprotnom 0.
	\[
	 (w_{i})_j =
	 \begin{cases}
	 1  & \text{ za } j \in V_i \\ 
	 0  & \text{ za } j \notin V_i
	 \end{cases}
	\]
	
	Posmatrajmo sada proizvod Laplasove matrice i vektora $w_i$.
	\[
	L_G\vec{w_i} = \vec{r}
	\]
	\[
	\vec{r}_j = L_{j.} \vec{w_i} = \sum_{ k  \in V} L_{jk}(\vec{w_{i}})_k 
	\]
	Kako su koordinate vektora $w_i$ čiji se indeksi ne nalaze u skupu čvorova povezane komponente jednake 0, možemo sumirati samo po čvorovima povezane komponente $G_i$. 
	\[
	\vec{r}_j = \sum_{ k  \in V_i} L_{jk}(\vec{w_{i}})_k = \sum_{ k  \in V_i} L_{jk} * 1 = \sum_{ k  \in V_i} L_{jk}
	\]

	Sagledajmo sada dva slučaja. Prvi slučaj je kada čvor j nije sadržan u komponenti povezanosti, tj. $j \notin V_i$.
	U tom slučaju ne postoji ni jedna grana koja povezuje čvor j sa nekim od čvorova iz $V_i$, pa samim tim za $k \in V_i$ važi $L_{jk} = 0$.
	S toga važi $\vec{r}_j = 0$ za $j \notin V_i$.
	Drugi slučaj je kada je čvor j sadržan u komponenti povezanosti, tj. $j \in V_i$.
	U tom slučaju ne postoji niti jedna grana koja povezuje čvor j sa nekim od čvorova van $V_i$, pa samim tim za $k \notin V_i$ važi $L_{jk} = 0$.
	S toga važi $ \sum_{k \in V_i}L_{jk} = \sum_{k \in V}L_{jk} = 0 \Rightarrow \vec{r}_j = 0$ za $j \in V_i$. Konačno, zaključujemo da je $\vec{r}$ nula vektor.
	
	Pošto je rezultantni vektor nula vektor, to nam ukazuje da su vektori $\vec{w_i}$ sopstveni vektori koji odgovaraju sopstvenoj vrijednosti $\alpha_1 = 0$ Laplasove matrice.
	Kako su skupovi čvorova komponenti povezanosti grafa disjunktni skupovi, jasno je i da su vektori $w_i$ $i \in \{1,2 \dots k\}$ linearno nezavisni vektori.
	Odredivši opštu formulu za sopstvene vektore koji odgovaraju sopstvenoj vrijednosti $\alpha_1 = 0$ Laplasove matrice grafa k komponenti povezanosti, mi smo odredili sve vektore u skupu $S_{\alpha_1}$.
	Sopstveni vektorski prostor koji odgovara $\alpha_1 = 0$ jednak linearnom vektorskom omotaču nad vektorima $w_i$ za $i \in \{1,2 \dots k\}$.
	\[
	U_{\alpha_1} = S_{\alpha_1} \cup \{\vec{0}\}= L (\{ w_i : i \in \{1,2 \dots k\}\})
	\]
	\[
	|U_{\alpha_1}| = k
	\]
	
	Odavde možemo zaključiti da je geometrijska višestrukost sopstvene vrijednosti $\alpha_1 = 0$ jednaka k, što odgovara broju povezanih komponenti grafa.
	\end{custom_proof}
	
	\begin{example}{Odrediti spektar Laplasove matrice kompletnog grafa $K_n$}.

	U kompletnom grafu sa $n$ čvorova $K_n =(V,E)$ svaki čvor je susjedan sa svim ostalim čvorovima, 
	tj. važi $deg(v)=n-1$ za $\forall v \in V$. To znači da su svi elementi na glavnoj dijagonali Laplasove matrice jednaki $n-1$,
	a svi ostali elementi jedanki $-1$.
	\[
	L_G =
	\begin{bmatrix}
	n-1 & -1 & -1 & \dots & -1 \\
	-1 & n-1 & -1 & \dots & -1 \\
	-1 & -1 & n-1 & \dots & -1 \\
	\vdots & \vdots & \vdots & \ddots & \vdots \\
	-1 & -1 & -1 & \dots & n-1 \\
	\end{bmatrix}_{n \times n}
	\]
	
	Posmatraćemo matricu $L_{K_n} - nI$. Motivacija za posmatranje ove matrice je njeno svojstvo da su svi njeni elementi jednaki $-1$, tj. $L_{K_n} - nI = [-1]_{n \times n}$,
	što bi nam trebalo omogućiti da lako odredimo njen spektar.
	
	Kako su sve vrste/kolone matrice $L_{K_n} - nI$ jednake, jasno je da $rank(L_{K_n} - nI) = 1$ i $det(L_{K_n} - nI) = 0$. 
	Ako uzmemo u obzir da 
	\[
	det(M) = 0 \Leftrightarrow \alpha = 0 \text{ je sopstvena vrijednost matrice M za } \forall M \in \mathbb{M}_{n \times n}
	\]
	dolazimo do zaključka je $\alpha = 0$ je sopstvena vrijednost matrice $L_{K_n} - nI$. 
	Sada je potrebno odrediti geometrijsku višestrukost sopstvene vrijednosti $\alpha$. Geometrijska višestrukost sopstvene vrijednosti jednaka je dimenziji sopstvenog vektorskog prostora koji joj odgovara.
	
	Da bi $\alpha'$ bila sopstvena vrijednost matrice $L_{K_n} - nI$, mora da važi:
	\[
	\begin{split}
	(L_{K_n} - nI)\vec{x} &= \alpha' \vec{x} \\
	(L_{K_n} - nI)\vec{x} &= \alpha' I \vec{x} \\
	(L_{K_n} - nI)\vec{x} - \alpha' I \vec{x} &= \vec{0}  \\
	[(L_{K_n} - nI) - \alpha' I] \vec{x} &= \vec{0}  \\
	[L_{K_n} - (n + \alpha') I] \vec{x} &= \vec{0}  \\
	\end{split}
	\]
	Na osnovu izraza, može se zaključiti da ako je $\alpha'$ sopstvena vrijednost matrice $L_{K_n} - nI$, onda je $n + \alpha'$ sopstvena vrijednost matrice $L_{K_n}$.

	Dimenzija sopstvenog vektorskog prostora koji odgovara $\alpha'$ jednaka je dimenziji nula prostora matrice $L_{K_n} - (n + \alpha') I$.
	Dakle, geometrijska višestrukost za  $\alpha$ jednaka je $\dim \ker(L_{K_n} - (n + \alpha') I)$.

	Ako iskoristimo sada Rank-Nullity teoremu koja kaže:
	$rank(M) + \dim ker(M) = n$ za $\forall M \in M_{n \times n}$, dobijamo je geometrijska višestrukost za $\alpha$ jednaka $n - rank(L_{K_n} - (n + \alpha') I)$.
	Za $\alpha' = \alpha = 0$ dobijamo vrijednost $n-1$.
	
	Na osnovu svega navedenog zaključujemo da je $n + \alpha' = n + \alpha = n + 0 = n$ sopstvena vrijednost matrice $L_{K_n}$ sa geometrijskom višestrukošću $n-1$. Ne zaboravimo da je $\alpha_1=0$ takođe sopstvena vrijednost Laplasove matrice, svakog, pa tako i ovog grafa.
	
	Drugi način da potvrdimo da je 0 sopstvena vrijednost matrice $L_{K_n} - n I$ jeste da, uočimo sljedeći vektor
	\[
	\vec{x} =
	\begin{bmatrix}
	 1-n \\ 
	1 \\
	1 \\
	\vdots \\
	1
	\end{bmatrix}_{n \times 1}
	\]
	dobijamo
	\[
	\begin{split}
	(L_{K_n} - nI) \vec{x} &= 
	\begin{bmatrix}
	-1 & -1 & -1 & \dots & -1 \\
	-1 & -1 & -1 & \dots & -1 \\
	-1 & -1 & -1 & \dots & -1 \\
	\vdots & \vdots & \vdots & \ddots & \vdots \\
	-1 & -1 & -1 & \dots & -1
	\end{bmatrix}
	\begin{bmatrix}
	-(n-1) \\
	1 \\
	1 \\
	\vdots \\
	1
	\end{bmatrix}
	 = 
	\begin{bmatrix}
	(n-1) + \sum_{i=1}^{n-1} (-1) \\
	(n-1) + \sum_{i=1}^{n-1} (-1) \\
	(n-1) + \sum_{i=1}^{n-1} (-1) \\
	\vdots \\
	(n-1) + \sum_{i=1}^{n-1} (-1)
	\end{bmatrix}  \\
	& =
	\begin{bmatrix}
	(n-1) + (n-1) * (-1) \\
	(n-1) + (n-1) * (-1) \\
	(n-1) + (n-1) * (-1) \\
	\vdots \\
	(n-1) + (n-1) * (-1) \\
	\end{bmatrix}
	=
	\begin{bmatrix}
	(n-1) - (n-1) \\
	(n-1) - (n-1) \\
	(n-1) - (n-1) \\
	\vdots \\
	(n-1) - (n-1) \\
	\end{bmatrix}
	 =
	\begin{bmatrix}
	0 \\
	0 \\
	0 \\
	\vdots \\
	0 \\
	\end{bmatrix}
	= 0 *
	\begin{bmatrix}
	-(n-1) \\
	1 \\
	1 \\
	\vdots \\
	1 \\
	\end{bmatrix}
	\end{split}
	\]
	
	\[
	\begin{split}
	(L_{K_n} - nI) 
	\begin{bmatrix}
	1-n \\
	1 \\
	1 \\
	\vdots \\
	1
	\end{bmatrix}_{n \times 1}
	= 0 *
	\begin{bmatrix}
	1-n \\
	1 \\
	1 \\
	\vdots \\
	1 \\
	\end{bmatrix}_{n \times 1} 
	\end{split}
	\]
	Odavde dobijamo da je 0 sopstvena vrijednost matrice $L_{K_n} - nI$. 
	\end{example}
	
	\begin{example}
		Odrediti spektar Laplasove matrice ciklusa sa n čvorova $C_n$.
	
		Ono što odmah možemo reći za graf $C_n$ jeste da je ovo regulan graf sa stepenom regularnosti 2. 
	Laplasova matrica ciklusa data je sa 
	\[
	\begin{bmatrix}
	
	2 & 1 & 0 & 0 & \dots & \dots & \dots & 1 \\
	1 & 2 & 1 & 0 & \dots & \dots & \dots & 0 \\
	0 & 1 & 2 & 1 & \dots & \dots & \dots & 0 \\
	0 & 0 & 1 & 2 & \dots & \dots & \dots & 0 \\
	0 & 0 & 0 & 1 & \dots & \dots & \dots & 0 \\
	\vdots  & \vdots  & \vdots  & \vdots  & \ddots & \vdots & \vdots & \vdots  \\
	0 & 0 & 0 & 0 & \dots & 2 & 1 & 0 \\
	0 & 0 & 0 & 0 & \dots & 1 & 2 & 1 \\
	1 & 0 & 0 & 0 & \dots & 0 & 1 & 2 \\
	\end{bmatrix}
	\]
	
	Odmah se vidi da vrste ove matrice predstavljaju ciklične permutacije.
	Upravo je ovakva priroda Laplasove matrice povod da prilikom određivanja njenog spektra posmatramo sljedeći vektor 
	\[ \vec{x} =
	\begin{bmatrix}
	1 \\
	w \\
	w^2\\
	\vdots \\
	w^{n-1}
	\end{bmatrix}
	\]
	gdje je $w^n = 1 \land w \in \mathbb{C}$. Izračunajmo proizvod $L_c \vec{x}$
	\[
	\begin{split}
	  	L_C \vec{x} & = 
		\begin{bmatrix}
		2 & 1 & 0 & 0 & \dots & \dots & \dots & 1 \\
		1 & 2 & 1 & 0 & \dots & \dots & \dots & 0 \\
		0 & 1 & 2 & 1 & \dots & \dots & \dots & 0 \\
		0 & 0 & 1 & 2 & \dots & \dots & \dots & 0 \\
		0 & 0 & 0 & 1 & \dots & \dots & \dots & 0 \\
		\vdots  & \vdots  & \vdots  & \vdots  & \ddots & \vdots & \vdots & \vdots  \\
		0 & 0 & 0 & 0 & \dots & 2 & 1 & 0 \\
		0 & 0 & 0 & 0 & \dots & 1 & 2 & 1 \\
		1 & 0 & 0 & 0 & \dots & 0 & 1 & 2 \\
		\end{bmatrix}
		 \begin{bmatrix}
		1 \\
		w \\
		w^2 \\
		w^3 \\
		w^4 \\
		\vdots \\
		w^{n-3} \\
		w^{n-2} \\
		w^{n-1} \\
		\end{bmatrix} 
		=
		\begin{bmatrix}
		 2 + w + w^{n-1} \\
		1 + 2w + w^2 \\
		w + 2w^2 + w^3 \\
		w^2 + 2w^3 + w^4 \\
		w^3 + 2w^4 + w^5 \\
		\vdots \\
		w^{n-4} + 2w^{n-3} + w^{n-2} \\
		w^{n-3} + 2w^{n-2} + w^{n-1} \\
		1 + w^{n-2} + 2w^{n-1} \\
		\end{bmatrix} = \\
	 	 & =  
		\begin{bmatrix}
		 w^{n-1} + 2 * 1 + w * 1 \\
		1 + 2w + w^2       \\
		w + 2w^2 + w^3       \\
		w^2 + 2w^3 + w^4  \\
		w^3 + 2w^4 + w^5  \\
		\vdots \\
		w^{n-4} + 2w^{n-3} + w^{n-2} \\
		w^{n-3} + 2w^{n-2} + w^{n-1} \\
		1 + 2w^{n-1} + w^{n-2} \\
		\end{bmatrix}
		= 
		\begin{bmatrix}
		w^{n-1} + 2 w^n + w * w^{n} \\
		1 + 2w + w^2       \\
		w + 2w^2 + w^3       \\
		w^2 + 2w^3 + w^4  \\
		w^3 + 2w^4 + w^5  \\
		\vdots \\
		w^{n-4} + 2w^{n-3} + w^{n-2} \\
		w^{n-3} + 2w^{n-2} + w^{n-1} \\
		w^n + 2w^{n-1} + w^{n-2} \\
		\end{bmatrix}
		= 
		\begin{bmatrix}
		w^{n-1} * ( 1+ 2 w + w^2) \\
		1 + 2w + w^2       \\
		w * (1 + 2w + w^2)       \\
		w^2 * (1 + 2w + w^2)       \\
		w^3 * (1 + 2w + w^2)       \\
		\vdots \\
		w^{n-4} * (1 + 2w + w^2)       \\
		w^{n-3} * (1 + 2w + w^2)       \\
		w^{n-2} * (1 + 2w + w^2)       \\
		\end{bmatrix} = \\
		& = (1 + 2w + w^2)
		\begin{bmatrix}
		w^{n-1}  \\
		1        \\
		w        \\
		w^2      \\
		w^3      \\
		\vdots   \\
		w^{n-4}  \\
		w^{n-3}  \\
		w^{n-2}  \\
		\end{bmatrix}
		= 
		w (w^{-1} + 2 + w)
		\begin{bmatrix}
		w^{n-1}  \\
		1        \\
		w        \\
		w^2      \\
		w^3      \\
		\vdots   \\
		w^{n-4}  \\
		w^{n-3}  \\
		w^{n-2}  \\
		\end{bmatrix}
		= 
		(w^{-1} + 2 + w)
		\begin{bmatrix}
		w^n      \\
		w        \\
		w^2      \\
		w^3      \\
		w^4      \\
		\vdots   \\
		w^{n-3}  \\
		w^{n-2}  \\
		w^{n-1}  \\
		\end{bmatrix} = \\
		& = 
		(w^{-1} + 2 + w)
		\begin{bmatrix}
		1      \\
		w        \\
		w^2      \\
		w^3      \\
		w^4      \\
		\vdots   \\
		w^{n-3}  \\
		w^{n-2}  \\
		w^{n-1}  \\
		\end{bmatrix}
	\end{split}
	\]
	\[
	 L_C 
	 \begin{bmatrix}
		1      \\
		w        \\
		w^2      \\
		w^3      \\
		w^4      \\
		\vdots   \\
		w^{n-3}  \\
		w^{n-2}  \\
		w^{n-1}  \\
		\end{bmatrix} 
	= 
	(w^{-1} + 2 + w)
		\begin{bmatrix}
		1      \\
		w        \\
		w^2      \\
		w^3      \\
		w^4      \\
		\vdots   \\
		w^{n-3}  \\
		w^{n-2}  \\
		w^{n-1}  \\
		\end{bmatrix}
	\]
	
	Dobijamo da je $w^{-1} + 2 + w$ sopstvena vrijednost, a $\vec{x}$ odgovarajući sopstveni vektor matrice.
	U skupu $\mathbb{R}$ jendačina $w^n = 1$ ima samo jedno rešenje $w=1$. Međutim, ništa nas ne sprečava 
    da jednačinu riješimo u skupu $\mathbb{C}$.
	\[
	\begin{split}
	w^n  & = 1 \\
	(x + iy)^n & = 1 \\
	|w|(\cos\varphi + i \sin \varphi)^n & = 1
	\end{split}
	\]
	Iskoristimo Muavrovu formulu za stepenovanje kompleksnih brojeva u trigonometrijskom zapisu
	\[
		\begin{split}
			|w|^n (\cos n\varphi + i\sin n\varphi) & = 1 + i * 0 \\
			\big(|w|^n \cos n \varphi = 1 \big) \land \big((|w|)^n \sin n \varphi\big) & = 0  
		\end{split}
		\]
		Jasno je da $|w| \neq 0$ jer u tom slučaju $w=0$. S toga, 
	
	\[
	\begin{split}
		\sin n \varphi &= 0 \\
		\sin n \varphi &= 2 k \pi \\
		\varphi &= \frac{2 k \pi}{n} \\
		\varphi \in [0,2\pi] &\Rightarrow k \in \{0,1, \dots n-1\} \\
		|w|^n \cos n \varphi & = 1 \\
		|w|^n \cos n 2 k \pi & = 1 \\
		|w|^n  & = 1 \\
	\end{split}
	\]
	Kako $|w| \in \mathbb{R}$ dobijamo $|w|=1$.
	\[
	\begin{split}
		w_k = |w| (\cos \varphi + i \sin \varphi) = \cos \frac{2 k \pi}{n} + i \sin \frac{2 k \pi}{n}
	\end{split}
	\]

	Već smo konstatovali da su sve sopstvene vrijednosti Laplasove matrice realni brojevi, ali sada smo naizgled dobili kompleksan broj. 
	Izračunajmo $\alpha_k$:
	\[
	\begin{split}
		\alpha_k & = w_k^{-1} + 2 + w_k \\
		\alpha_k & = \cos \frac{2 k \pi}{n} + i \sin \frac{2 k \pi}{n} + 2 + \cos \frac{-2 k \pi}{n} + i \sin \frac{-2 k \pi}{n} \\
		\alpha_k & = (\cos \frac{2 k \pi}{n} + \cos \frac{-2 k \pi}{n}) + i(\sin \frac{2 k \pi}{n} + \sin \frac{-2 k \pi}{n}) + 2 \\
		\alpha_k & = (\cos \frac{2 k \pi}{n} + \cos \frac{2 k \pi}{n}) + i(\sin \frac{2 k \pi}{n} - \sin \frac{2 k \pi}{n}) + 2 \\
		\alpha_k & = 2\cos \frac{2 k \pi}{n} + 2 \\
		\alpha_k & = 2 (\cos \frac{2 k \pi}{n} + 1)
	\end{split}
	\]
	Iako je w komplekan broj, vidimo da $\alpha_k \in \mathbb{R}$ za $\forall k \in \{0,1,2, \dots n-1\}$.
	Razmotrimo vrijednost $\alpha_{n-k}$.
	\[
	\begin{split}
		\alpha_{n-k} & = 2 (\cos \frac{2 (n-k) \pi}{n} + 1) \\
		\alpha_{n-k} & = 2 (\cos \frac{2n\pi - 2k\pi}{n} + 1) \\
		\alpha_{n-k} & = 2 (\cos ( 2\pi - \frac{2k\pi}{n}) + 1) \\
		\alpha_{n-k} & = 2 (\cos \frac{2k\pi}{n} + 1) \\ 
		\alpha_{n-k} & = \alpha_k 
	\end{split}
	\]
	Odavde vidimo su sopstvene vrijednosti simetrične oko $n/2$, što ukazuje da ih ima $\lceil \frac{n}{2}\rceil$ 
	Sopstveni vektor koji odgovara sopstvenoj vrijednosti $\alpha_k$ je dat sa 
	\[
	\begin{bmatrix}
	1 \\
	\cos \frac{2 k \pi}{n} + i \sin \frac{2 k \pi}{n} \\
	\cos \frac{4 k \pi}{n} + i \sin \frac{6 k \pi}{n} \\
	\vdots \\
	\cos \frac{(n-2) k \pi}{n} + i \sin \frac{(n-2) k \pi}{n} \\
	\cos \frac{(n-1) k \pi}{n} + i \sin \frac{(n-1) k \pi}{n} 
	\end{bmatrix}
	\]

	\end{example}

		\begin{example}
		Posmatrajmo sljedeci graf G
		\begin{figure}[h]
			\centering
			\includegraphics[width=1\textwidth]{Figure_1.png}
		\end{figure}
		\[
			A_G =
			\begin{bmatrix}
				0  &  1  &  1  &  1  &  0  & 0  &  0  &  0  &  0  &  0\\
				1  &  0  &  0  &  1  &  0  & 0  &  0  &  0  &  0  &  0\\
				1  &  0  &  0  &  0  &  0  & 0  &  0  &  0  &  0  &  0\\
				1  &  1  &  0  &  0  &  0  & 0  &  0  &  0  &  0  &  0\\
				0  &  0  &  0  &  0  &  0  & 1  &  1  &  0  &  0  &  0\\
				0  &  0  &  0  &  0  &  1  & 0  &  1  &  0  &  0  &  0\\
				0  &  0  &  0  &  0  &  1  & 1  &  0  &  0  &  0  &  0\\
				0  &  0  &  0  &  0  &  0  & 0  &  0  &  0  &  1  &  0\\
				0  &  0  &  0  &  0  &  0  & 0  &  0  &  1  &  0  &  0\\
				0  &  0  &  0  &  0  &  0  & 0  &  0  &  0  &  0  &  0\\
			\end{bmatrix}
			\]
			
			Jasno je da se graf sastoji od 3 komponente povezanosti. Formirajmo njegovu Laplasovu matricu i oderedimo njen spektar.
			
			\[
				L_G =
				\begin{bmatrix}
					3   &  -1  &  -1  &  -1  &  0  &  0   &  0   &  0   &  0   &  0\\
					-1  &  2   &  0   &  -1  &  0  &  0   &  0   &  0   &  0   &  0\\
		-1  &  0   &  1   &  0   &  0  &  0   &  0   &  0   &  0   &  0\\
		-1  &  -1  &  0   &  2   &  0  &  0   &  0   &  0   &  0   &  0\\
		0   &  0   &  0   &  0   &  2  &  -1  &  -1  &  0   &  0   &  0\\
		0   &  0   &  0   &  0   &  -1 &  2   &  -1  &  0   &  0   &  0\\
		0   &  0   &  0   &  0   &  -1 &  -1  &  2   &  0   &  0   &  0\\
		0   &  0   &  0   &  0   &  0  &  0   &  0   &  1   &  -1  &  0\\
		0   &  0   &  0   &  0   &  0  &  0   &  0   &  -1  &  1   &  0\\
		0   &  0   &  0   &  0   &  0  &  0   &  0   &  0   &  0   &  0\\
	\end{bmatrix}
	\]
	Numeričkim metodama dolazimo do spektra ove matrice.
	\[
		\begin{split}
			\alpha_1=0 \text{  } & \vec{e_1}^T = \begin{bmatrix} -0.5  &   -0.5  &  -0.5  &  -0.5  &  0  &  0  &  0  &  0  &  0  & 0      \end{bmatrix} \\ 
			\alpha_2=0 \text{  } & \vec{e_2}^T = \begin{bmatrix} 0  &   0  &  0  &  0  &  -0.58  &  -0.58  &  -0.58  &  0  &  0  & 0      \end{bmatrix} \\ 
			\alpha_3=0 \text{  } & \vec{e_3}^T = \begin{bmatrix} 0  &  0  &  0  &  0  &  0  &  0  &  0  &  0.71  &  0.71  &  0            \end{bmatrix} \\
			\alpha_4=0 \text{  } & \vec{e_4}^T = \begin{bmatrix} 0  &  0  &  0  &  0  &  0  &  0  &  0  &  0  &  0  &  1                  \end{bmatrix} \\
			\alpha_5=1 \text{  } & \vec{e_5}^T = \begin{bmatrix} 0  &  0.41  &  -0.82  &  0.41  &  0  &  0  &  0  &  0  &  0  &  0        \end{bmatrix} \\
			\alpha_6=2 \text{  } & \vec{e_6}^T = \begin{bmatrix} 0  &  0  &  0  &  0  &  0  &  0  &  0  &  0.71  &  -0.71  &  0           \end{bmatrix} \\
			\alpha_7=3 \text{  } & \vec{e_7}^T = \begin{bmatrix} 0  &  0.71  &  0  &  -0.71  &  0  &  0  &  0  &  0  &  0  &  0           \end{bmatrix} \\
			\alpha_8=3 \text{  } & \vec{e_8}^T = \begin{bmatrix} 0  &  0  &  0  &  0  &  0.82  &  -0.41  &  -0.41  &  0  &  0  &  0       \end{bmatrix} \\
			\alpha_9=3 \text{  } & \vec{e_9}^T = \begin{bmatrix} 0  &  0  &  0  &  0  &  0  &  0  &  0.29  &  -0.81  &  0.51  &  0        \end{bmatrix} \\
			\alpha_1=4 \text{  } & \vec{e_1}^T = \begin{bmatrix} 0.87  &  -0.29  &  -0.29  &  -0.29  &  0  &  0  &  0  &  0  &  0  &  0  \end{bmatrix}
		\end{split}
		\]
		Pogledajmo kakve vrijednosti sosptveni vektori koji odgovaraju $\alpha_1$ dodjeljuju čvorovima. 
		\begin{figure}[h]
			\centering
			\includegraphics[width=0.8\textwidth]{plot.png}
		\end{figure}
		
		Za očekivati je bilo da sopstvena vrijednost nula ima višestrukost 4, jer se graf sastoji od 4 komponente povezanosti, ali zanimljivo je
		da svaki sopstveni vektor koji odgovara ovoj sopstvenoj vrijednosti ističe čvorove koji pripadaju jednoj od komponenti povezanosti.
		Svaki od ovih sopstvenih vektora dodjeljuje čvorovima jedne komponente povezanosti istu vrijednost, a svim ostalim čvorovima vrijednost nula. 
		
		Najmanja nenula sopstvena vrijednost je mala, što ukazuje da graf nije čvrsto povezan, odnosno da je
        jako blizu da ima 5 povezanih komponenti. To se može postići uklanjanjem samo jedne grane.
	\end{example}

	\section{Interpretacija vektora i matrica u domenu grafova}

	Već nam je jasno da pomoću pojedinih matrica na različite načine možemo predstaviti isti graf, pri tome zadržavajući sve informacije o njegovoj strukturi. 
    Možemo predstaviti susjednost čvorova putem matrice susjedstva ili Laplasove matrice, ili pak incidentnost čvorova i grana putem matrice incidentnosti.
	Često su u izrazima sa ovim matricama konstituisali i vektori, naročiti sopstveni vektori tih matrica. 
	Postavlja se jedno logično pitanje, a to je koja je uloga tih vektora? Koji je njihov smisao u domenu grafova?
	
	Vidjeli smo da ukoliko posmatramo graf sa n čvorova, vektori koje koristimo su dimenzije $n \times 1$. 
	Pojavljivanje broja čvorova u njihovoj dimenziji nije slučajnost, jer se ti vektori mogu posmatrati kao diskretne funkcije koje
	u pojedinim tačkama imaju odgovarajuće vrijednosti. 
	Te tačke u kojima je definisana vrijednosti funkcije su upravo čvorovi grafa.
	Drugim riječima, vektori dimenzije $n \times 1$ se mogu posmatrati kao funkcije $\vec{x} = f(v)$ koje svakom čvoru grafa dodjeljuju neku vrijednosti na način da vektor $\vec{x}$ čvoru grafa G sa indeksom i dodjeljuje
	vrijednost $\vec{x}_i$, tj. $\vec{x}^T = [ f(v_1) f(v_2) \dots f(v_n)]$. Ove funkcije se nazivaju graf funkcije.
	Vrijednost dodijeljena čvoru može da ima različita značenja, to može biti njegova važnost, vrijedost koja se koristi za klasifikaciju čvorova grafa, možda i njegova koordinata. 
	Kako su i sopstveni vektori Laplasove i matrice susjedstva takođe vektori dimenzije $n \times 1$, i oni se mogu posmatrati kao funkcije. 
	Sopstveni vektori predstavjaju posebne funkcije koje se nazivaju sopstvene funkcije (eigenfunction).

	Često se u izrazima sreće množenje matrice susjedstva ili Laplasove matrice sa nekim vektorom. Posmatrajmo čemu su jednaki ti proizvodi.
	\[
	A_G \vec{x} = 
	\begin{bmatrix}
		\sum_{\{v_1,v_j\} \in E} \vec{x}_j \\
		\sum_{\{v_2,v_j\} \in E} \vec{x}_j \\
		\vdots \\
		\sum_{\{v_n,v_j\} \in E} \vec{x}_j \\
	\end{bmatrix}_{n \times 1} =
	\begin{bmatrix}
		\sum_{\{v_1,v_j\} \in E} f(v_j) \\
		\sum_{\{v_2,v_j\} \in E} f(v_j) \\
		\vdots \\
		\sum_{\{v_n,v_j\} \in E} f(v_j) \\
	\end{bmatrix}_{n \times 1} 
	\]
	Rezultat množenja je vektor dimenzije $n \times 1$ kod kojeg koordinata sa indeksom i predstavlja sumu vrijednosti koje vektor $\vec{x}$ dodijeljuje susjedima čvora sa indeksom i.
	
	\[
		\begin{split}
		L_G \vec{x} & =
		\begin{bmatrix}
			deg(v_1) * \vec{x_1} - \sum_{\{v_1,v_j\} \in E} \vec{x}_j \\
			deg(v_2) * \vec{x_2} - \sum_{\{v_2,v_j\} \in E} \vec{x}_j \\
			\vdots \\
			deg(v_n) * \vec{x_n} - \sum_{\{v_n,v_j\} \in E} \vec{x}_j \\
		\end{bmatrix} =
		\begin{bmatrix}
			\sum_{\{v_1,v_j\} \in E} \vec{x}_1-\vec{x}_j \\
			\sum_{\{v_2,v_j\} \in E} \vec{x}_2-\vec{x}_j \\
			\vdots \\
			\sum_{\{v_2,v_j\} \in E} \vec{x}_n-\vec{x}_j \\
		\end{bmatrix} \\
		& = 
		\begin{bmatrix}
			\sum_{ j \neq 1}  \vec{x}_1-\vec{x}_j \\
			\sum_{ j \neq 2}  \vec{x}_2-\vec{x}_j \\
			\vdots \\
			\sum_{ j \neq n}  \vec{x}_n-\vec{x}_j \\
		\end{bmatrix} =
		\begin{bmatrix}
			\sum_{ j \neq 1}  f(v_1) - f(v_j) \\
			\sum_{ j \neq 2}  f(v_2) - f(v_j) \\
			\vdots \\
			\sum_{ j \neq n}  f(v_n) - f(v_j) \\
		\end{bmatrix} =
		\end{split}
	\]

	Rezultat množenja Laplasove matrice i proizvoljnog vektora dimenzije $n \times 1$ je ponovo vektor dimenzije $n \times 1$
	kod kojeg je vrijednost koordinate sa indeksom i jednaka sumi razlika vrijednosti dodijeljenih čvoru sa indeksom i i njegovim susjedima.

	I u domenu grafova, kao i u linearnoj algebri, matrice nisu ništa drugo do reprezentacije operatora. 
	S toga, na osnovu takve prirode matrica i na osnovu dobijenih izraza možemo zaključiti sljedeće:
	\begin{itemize}
		\item Matrica susjedstva odgovara operatoru koji računa sumu vrijednosti dodijeljenih svim susjedima čvorova, tj. njihovoj okolini.
		\item Laplasova matrica odgovara operatoru koji računa razliku između vrijednosti dodijeljene čvoru i vrijednosti dodiljenih
		čvorovima u njegovoj okolini. 
		Ovaj operator pokazuje koliko vrijednost koju funkcija $\vec{x}$ dodijeli nekom čvoru odstupa od vrijednosti dodijeljenih njegovim susjedima.
	\end{itemize}

	Izraz $\sum_{ j \neq i}  f(v_i) - f(v_j)$ ukazuje na analogiju Laplasove matrice i istoimenog operatora.
	Pokušajmo sada otkriti povezanost između njih koristeći se osobinama Laplasovog operatora.

	Naime, Laplasov operator $\Delta f(p) = \nabla^2 f(p)$ funkcije f u tački p predstavlja vrijednost divergencije gradijenta funkcije f u tački p.
	Gradijent u Euklidskom prostoru nije ništa drugo do vektor čije su koordinate parcijalni izvodi funkcije f po svim promjenjivim, tj. geometrijski gledano po svim pravcima.
	
	Već znamo da bi diskretna funkcija u domenu grafova bila vektor, ali šta je njen gradijent. 
	Analogija izvoda kontinualne funkcije u diskretnom domenu je razlika, a analogija pravca u domenu grafa jeste grana. 
	Tako da bi vrijednost parcijalnog izvoda, tj. gradijenta u odnosu na granu $e=\{v_1,v_2\}$ diskretne funkcije $f(v) = \vec{x}$ u domenu grafa u čvoru $v_1$ bila $\nabla  f(v_1)|_e = \vec{x}_{v_1} - \vec{x}_{v_2}$.
	Vrijednost gradijenta u čvoru v bi bila $\nabla f(v_1) = (f(v_1) - f(v_2),f(v_1) - f(v_3),f(v_1) - f(v_4), \cdots)$, gdje su sa $v_i$ za $i>1$ predstavljeni susjedi čvora v.
	Ispostavlja se da važi $\nabla \vec{x} = I^T \vec{x}$, tj. do gradijenta funkcije u domenu grafova se može doći množenjem sa transponovanom matricom incidencije (ali onom koja je formirana na osnovu proizvoljno dodijeljene orjentacije granama!).
	Gradijent se može posmatrati i kao funkcija $g(e)$ koja svakoj grani grafa dodjeljuje vrijednost.
	
	Divergencija vektorskog polja, geometrijski gledano, pokazuje koliko se vektori šire iz neke tačke, tj. predstavlja izlazni fluks polja u nekoj tački.
	U domenu grafova, svakom čvoru grafa se može pridružiti vektor gradijenta graf funkcije u tom čvoru, pa se divergencija može dobiti kao zbir koordinata datog vektora.
	S toga, važi $\nabla^2 f(v_1) = \sum_{ \{v_1,v_1\} \in E} f(v_1) - f(v_i)$.
	Ispostavlja se da je $\nabla^2 f(v) = \nabla g(e) = \nabla I^T f(v) = I I^T f(v) = L_G f(v)$, što nam sada jasno ukazuje na povezanost između Laplasovog operatora i Laplasove matrice.

	Prema geometrijskoj intepretaciji, Laplasov operator pokazuje kolika je devijacija vrijednosti funkcije u tački p u odnosu na prosječnu vrijednost u njenoj okolini.
	Sličnu funkciju ima i množenje Laplasove matrica i vektora $\vec{x}$. Rezultantni vektor pokazuje koliko vrijednost vektora $\vec{x}$ u čvoru v odstupa od vrijednosti
	vektora u njegovim susjednim čvorovima (mada ne od prosječne vrijednosti), ponašajući se kao "diskretna" verzija Laplasovog operator. 
	Može se reći i da Laplasova matrica akumulira odstupanja vrijednosti u okolini, jer računa njihovu sumu po svim susjedima čvora.

	Laplasov operator funkcije mjeri koliko je funkcija glatka, tj. koliko se ona mijenja za bliske vrijednosti promjenjivih.
	S toga, Laplasova matrica mjeri koliko su graf funkcije, tj. vektori glatki i koliko odstupaju vrijednosti koje se dodjeljuju susjednim čvorovima.
	Kao što Laplasov operator ne mari za odstupanjem funkcije u tačkama koje su udaljene jedna od druge, tako ni Laplasova matrice ne mari za odstupanjem vrijednosti koje se dodjeljuju čvorovima koji nisu susjedni.
	
	\section{Primjena spektralne teorije grafova}

	    \subsection{Spectral graph drawing}

	Razmotrimo sada kako se spektralna teorema grafova može iskoristiti za njihovo prikazivanje. 
	Obično se loša reprezentacija grafa povezuje sa velikim brojem grana koje se sijeku.
	Iako znamo da je presijecanje grana u velikom broju slučajeva neizbježno, iz razloga što mnoštvo grafova nije planarno,
	trebamo težiti tome da broj grana koje se sijeku svedemo na minimum.
    Da bismo to postigli, potrebno je da strateški pozicioniramo čvorove.
	Kao logična ideja nameće se to da susjedne čvorove smjestimo što bliže.
	Na ovaj način grane koje ih spajaju će biti što kraće, a samim tim smanjiće se i vjerovatnoća da ih neka druga grana grafa siječe.
    
	Ukoliko graf prikazujemo u ravni koordinatnog sistema, blisko pozicioniranje susjednih čvorova se svodi na dodjeljivanje bliskih vrijednosti koordinata susjednim čvorovima.
    Razlika između koordinata susjednih čvora treba da bude što manja, tj. potrebno je da suma $\sum_{\{v,i\} \in E} (\vec{x}_v - \vec{x}_i)^2$ bude što manja za $\forall v \in V$.
	Posmatramo kvadrat razlike x koordinata samo zato da bismo izbjegli negativne vrijednosti.
    Ovaj izraz nam je već poznat, radi se o vrijednosti kvadratne forme Laplasove matrice, tj. $\vec{x}^T L_G \vec{x}$. Ponovo, Laplasova matrica se nameće kao potencijalno rješenje.

    Jasno je da nam ne odgovara da ova suma bude jednaka nuli, jer ipak želimo da razlikujemo susjedne čvorove, a ne da ih sve smjestimo u jednu tačku. S toga, vrijednost sume treba da bude što manja, ali različita od nule.
    Sada se postavlja pitanje kako odrediti vektor $\vec{x}$ koji svakom čvoru dodjeljuje vrijednost x koordinate imajući u vidu navedeno. 
	Zašto ne bismo pokušali sa sopstvenim vektorima Laplasove matrice?
    Do njih možemo (relativno) lako doći, a i znamo da $\vec{e_i}^T L_G \vec{e_i} = \alpha_i$.
	Fiedler-ov vektor se ponovo ukazuje kao potencijalno rješenje našeg problema jer upravo on od svih sopstvenih vektora Laplasove matrice minimizuje vrijednost kvadratne forme, tj. navedene sume.
    Treba napomenuti, da Fiedler-ov vektor nije uvijek najbolji izbor za dodjeljivanje koordinata čvorovima, ali je njegova prednost što po prirodi odgovara našem problemu i što znamo kako do njega doći za bilo koji graf.
    
	Sada znamo kako čvorovima dodijeliti x koordinatu, ali šta je y koordinate? 
	Ako bismo pokušali da svakom čvoru dodijelimo istu x i y koordinatu, došli bismo do toga da bi se svi čvorovi našli na pravoj $y = x$.
	Lako se zaključuje da bi u tom slučaju sve grane ležale na pomenutoj pravoj. Dakle, treba nam drugi pristup. 
    Šta ukoliko bismo y koordinatu samo skalirali za neko $p \in \mathbb{R}$? U tom slučaju svi čvorovi bi ležali na pravoj $y=px$, što nam opet ne odgovara iz istog razloga.
    Ukoliko sada to posmatramo u domenu vektora, ne odgovara nam ni jedna varijanta u kojoj je vektor $\vec{y}$ linearno zavisan sa vektorom $\vec{x}$.
    Sada bar znamo gdje da tražimo vektor $\vec{y}$ - u skupu vektora linerno nezavisnih sa Fiedler-ovim vektorom, a koji istovremeno nastoje da minimizuju kvadratnu formu Laplasove matrice.
    Upravo takav vektor je sopstveni vektor Laplasove matrice koji odgovara narednoj najmanjoj sopstvenoj vrijednosti. 
	Ukoliko pak, ovaj vektor ne daje najbolju reprezentaciju grafa, može se pokušati i sa ostalim sopstvenim vektorima.

    Za prikaz grafa u prostoru, iskoristili bismo još jedan od sopstvenih vektora Laplasove matrice putem kojeg bismo dodijelili z koordinate svim čvorovima.
    U nastavku ćemo vidjeti kako sve ovo izgleda na djelu.

    Prethodno opisanim pristupom i motivacijom se vodio i Hall kada je došao do pristupa prikazivanju grafa upotrebom spektra Laplasove matrice koji je nazvan "Spectral Graph Drawing". 

	\begin{example}
		U ravni prikazati ciklus sa 20 cvorova. Jasno da je idealan prikaz ovog grafa pravilan dvadesetougao.
        
		Laplasova matrica ciklusa sa 20 cvorova je data sa
		\[
			\begin{pmatrix}
				2 & -1 & 0 & 0 & \cdots & 0 & -1 \\
				-1 & 2 & -1 & 0 & \cdots & 0 & 0 \\
				0 & -1 & 2 & -1 & \cdots & 0 & 0 \\
				0 & 0 & -1 & 2 & \cdots & 0 & 0 \\
				\vdots & \vdots & \vdots & \vdots & \ddots & \vdots & \vdots \\
				0 & 0 & 0 & 0 & \cdots & 2 & -1 \\
				-1 & 0 & 0 & 0 & \cdots & -1 & 2
            \end{pmatrix}	
            \]
            
            Dio spektra Laplsove matrice je dat u nastavku
            \[
                \begin{split}
                    \alpha_1=-0   \text{  } \vec{e}_1 = \begin{bmatrix} -0.22 \\ -0.22 \\ -0.22 \\ -0.22 \\ -0.22 \\ -0.22 \\ -0.22 \\ -0.22 \\ -0.22 \\ -0.22 \\ -0.22 \\ -0.22 \\ -0.22 \\ -0.22 \\ -0.22 \\ -0.22 \\ -0.22 \\ -0.22 \\ -0.22 \\ -0.22 \end{bmatrix}  
                    \alpha_2=0.1  \text{  } \vec{e}_2 = \begin{bmatrix} -0.32 \\ -0.3 \\ -0.26 \\ -0.19 \\ -0.1 \\ 0.0 \\ 0.1 \\ 0.19 \\ 0.26 \\ 0.3 \\ 0.32 \\ 0.3 \\ 0.26 \\ 0.19 \\ 0.1 \\ 0.0 \\ -0.1 \\ -0.19 \\ -0.26 \\ -0.3 \end{bmatrix} 
                 \alpha_3=0.1  \text{  } \vec{e}_3 = \begin{bmatrix} -0.0 \\ 0.09 \\ 0.18 \\ 0.25 \\ 0.3 \\ 0.32 \\ 0.3 \\ 0.26 \\ 0.19 \\ 0.1 \\ 0.0 \\ -0.09 \\ -0.18 \\ -0.25 \\ -0.3 \\ -0.32 \\ -0.3 \\ -0.26 \\ -0.19 \\ -0.1 \end{bmatrix}
              \end{split}
              \]
              
              Iscrtajmo sada graf u ravni tako da su x i y koordinate cvorova data sa odgovarajucom vrijednosti sopstvenih vektora koji odgovara najmanjoj nenula sopstvenoj vrijednosti.
              
            \end{example}
            
            \begin{figure}[h]
                \centering
                \includegraphics[width=1\textwidth]{circle_graph.png}
            \end{figure}
            
    \begin{example}
        Predstaviti 'bucky ball' graf u ravni. 
        
        Radi se o grafu koji je matematicka reprezentacija molekula $c_{60}$. Molekul se sastoji od 60 atoma ugljenika koji forimiraju icosahedron.
        U pitanju je graf sa stepenom regularnosti 3.
        
        Kako su matrice pridruzene voom grafu jako velike i nisu pogodne za prikaz, u nastavku cemo dati samo dio liste susjednosti, podrazumijevajuci da je skup cvorova dat Sa
        $V = \{1,2, \dots 60\}$
        \[
            \begin{tabular}{c|l}
                \textbf{Vertex} & \textbf{Adjacent Vertices} \\
                \hline
                1 & 2, 3, 4 \\
                2 & 1, 55, 56 \\
                3 & 1, 58, 60 \\
                4 & 1, 57, 59 \\
                5 & 8, 13, 14 \\
                6 & 8, 12, 15 \\
                7 & 8, 11, 16 \\
                8 & 5, 6, 7 \\
                9 & 13, 15, 25 \\
                10 & 14, 16, 26 \\
                11 & 7, 12, 24 \\
                12 & 6, 11, 23 \\
                13 & 5, 9, 18 \\
        \end{tabular}
        \]

        Nakon izracunavanja spektra Laplasove matrice ovog grafa, vec nailazimo na prvi problem. Rekli smo da su nam potrebna 2 sopstvena vektora koja odgovarajaju najmanjin nenula sopstvenim 
        vrijednostima. Najmanja nenula sopstvena vrijednost je $\alpha2 = 0.24$ i ima geometrijsku visestrukost 3. Nemamu nikakvu indikaciju koja dva od moguca 3 vektora su najbolja za prikaz, pa cemo pokusati sve moguce kombinacije.

        \begin{figure}[h]
            \centering
            \includegraphics[width=1\textwidth]{bucky_ball_1.png}
        \end{figure}
        \begin{figure}[h]
            \centering
            \includegraphics[width=1\textwidth]{bucky_ball_2.png}
        \end{figure}
        \begin{figure}[h]
            \centering
            \includegraphics[width=1\textwidth]{bucky_ball_3.png}
        \end{figure}

        Zaista, sva tri prikazu su pristojna, ali se drugi naizgled cini kao najbolji.
        S druge strane, cini se kao da im nedostoje jos jedna dimenzija za bolji prikaz. I zaista, kao dati graf predstavimo u prostoru dobijamo zaista dobar prikaz.
        Da li je visestrukost najmanje nenula sopstvene vrijednosti upucivala da je graf potrebno predstaviti u trodimenzijalnom prostoru.

        \begin{figure}[h]
            \centering
            \includegraphics[width=1\textwidth]{bucky_ball_3d.png}
        \end{figure}
        
    \end{example}
    
    \begin{example}
        Prikazati pseudo-slucajno generisani graf sa 25 cvorova u ravni.
        Uz pomoc odgovarajuceg softvera generisan je sljedeci graf.
        
        \begin{figure}[h]
            \centering
            \includegraphics[width=1\textwidth]{random_graph_3.png}
        \end{figure}
        
        Pogledajmo sada do kakve reprezentacije grafa se dolazi upotrebom opisane metode.
        
        \begin{figure}[h]
            \centering
            \includegraphics[width=1\textwidth]{random_graph_4.png}
        \end{figure}

        Sa slike se vidi da he graf gotovo planaran, tj. prikaz sadrzi samo 3 grane koje se sijeku, sto je i bila nasa ideja prilikom prikazivanja grafa.
    \end{example}

	\begin{example}
		Posmatrajmo sada drugi graf.
		\begin{figure}[h]
			\centering
			\includegraphics[width=0.8\textwidth]{Figure_2.png}
		\end{figure}
		Odogvarajuca Laplasova matrica je data sa
		\[ 
		L_G =
		\begin{bmatrix}
		3 & -1 & 0 & 0 & -1 & -1 & 0 & 0 \\
		-1 & 2 & -1 & 0 & 0 & 0 & 0 & 0 \\
		0 & -1 & 2 & -1 & 0 & 0 & 0 & 0 \\
		0 & 0 & -1 & 2 & -1 & 0 & 0 & 0 \\
		-1 & 0 & 0 & -1 & 2 & 0 & 0 & 0 \\
		-1 & 0 & 0 & 0 & 0 & 3 & -1 & -1 \\
		0 & 0 & 0 & 0 & 0 & -1 & 2 & -1 \\
		0 & 0 & 0 & 0 & 0 & -1 & -1 & 2
		\end{bmatrix}
		\]
        \[
            \begin{split}
                \alpha_1=0.0 \text{      } & \vec{e_1}^T = \begin{bmatrix} 0.35  &   0.35  &  0.35  &  0.35  &  0.35  &  0.35  &  0.35  &  0.35      \end{bmatrix} \\ 
                \alpha_2=0.31 \text{     } & \vec{e_2}^T = \begin{bmatrix} 0.06  &   0.26  &  0.37  &  0.37  &  0.26  &  -0.34  &  -0.49  &  -0.49  \end{bmatrix} \\ 
                \alpha_3=1.38 \text{     } & \vec{e_3}^T = \begin{bmatrix} 0  &  0.6  &  0.37  &  -0.37  &  -0.6  &  0  &  0  &  0                  \end{bmatrix} \\
                \alpha_4=1.67 \text{     } & \vec{e_4}^T = \begin{bmatrix} -0.56  &  -0,31  &  0,46  &  0,46  &  -0,31  &  -0,13  &  0,19  &  0,19   \end{bmatrix} \\
                \alpha_5=3 \text{        } & \vec{e_5}^T = \begin{bmatrix} 0  &  0  &  0  &  0  &  0  &  0  & -0,71  &  0,71                         \end{bmatrix} \\
                \alpha_6=3.33 \text{     } & \vec{e_6}^T = \begin{bmatrix} 0,34  &  -0,37  &  0,16  &  0,16  &  -0,37  &  0,64  &  -0,27  &  -0.27   \end{bmatrix} \\
                \alpha_7=3.62 \text{     } & \vec{e_7}^T = \begin{bmatrix} 0  &  0.37  &  -0.6  &  0.6  &  -0.37  &  0  &  0  &  0            \end{bmatrix} \\
                \alpha_8=4.69 \text{     } & \vec{e_8}^T = \begin{bmatrix} -0,67  &  0,28  &  -0,07  &  -0,07  &  0.28  &  0.58  &  0.16  &  -0,16        \end{bmatrix}
            \end{split}
            \]
            Kako je dati graf povezan bilo je i za ocekivati da sopstveni vektori koji odgovara sopstvenoj vrijednosti nula svakom cvoru dodjeljuje istu vrijednost.
            Zanimljiv je sopstveni vektor koji odgovara najmanjoj pozitivnoj sopstvenog vrijednosti. Ukoliko pogledamo vrijednosti koje on dodjeljuje cvorovima grafa, uvidjamo da su pozitivne 
            vrijednosti dodijeljene prvih pet cvorova. Ovo nam ukazuje da se prilikom kreiranja klastera, ovi cvorovi trebaju biti u istom klasteru.
	\end{example}


    \subsection
    Lema. Za graf G stepena d i njegovu najvecu sopstvenu vrijdnost Laplasove matrice vazi $\alpha_n \geq d$.

    Dokaz.

    Kako bismo dokazali navedenu lemu upotrijebicemo Courrant-Fischer teoremu

    \begin{theorem}[Courrant-Fischer]
        Neka je data simetricna kvadratna matrica $M_{n \times n}$. Njene sopstvene vrijednosti $\alpha_1 \leq \alpha_2 \leq \dots \alpha_{n-1} \geq \alpha_n$.
        mozemo izracunati putem sljedeceg izraza
        \[
            \alpha_i = \min_{\vec{x} \neq \vec{0} \land \vec{x} \perp \vec{x}_i \forall i \in \{1,2, \dots, i-1\}} \frac{\vec{x}^T M \vec{x}}{\vec{x}^T\vec{x}} 
        \]
        gdje je $x_i$ sopstveni vektor koji odogvara $\alpha_1$. 
    \end{theorem}

    Courrant-Fischer teorema na govori da se sopstvena vrijednost $\alpha_i$ moze dobiti kao minimum Rayleough qoutiten po svim vektorima 
    ortogonalnim sa sopstvenim vektorima koji odogvaraju manjim sopstvenim vrijednostima. 

    Koristeci navedenu teoremu i steceno znanje o kvadratnoj formi Laplasove matrice dobijamo
    \[
        \alpha_n = \max_{\vec{x} \neq \vec{0}} \frac{\vec{x}^T M \vec{x}}{\vec{x}^T\vec{x}} = \max_{\vec{x} \neq \vec{0}} \frac{ \sum_{\{u,v\} \in E} (\vec{x}_u - \vec{x}_v)^2 }{ \sum_{i \in \{1,2 \dots n\}} \vec{x}_i^2}
    \]
        
    Ukoliko sad uzmemo da je $\vec{x}$ jedan od vektora iz standardne baze, tj da vazi za neko $j \in \{1,2, \dots n\} \vec{x}_j = 1$, a za $i \neq j$ vazi $\vec{x}_i = 0$ dobijamo
    \[
        \alpha_n > \frac{ d (\vec{x}_j - 0)^2 }{1} = d \times 1  
    \]
    \[
        \alpha_n > d  
    \]

    Razmislimo sada kako bismo mogli utvrditi koji dio grafa je najmanje povezan sa ostatkom grafa. Idealno bi bilo kada bismo mogli definisati mjeru povezanosti podskupa cvorova S grafa G sa ostatkom cvorova.
    Jasno je da cemo posmatrati grane koje povezuju neki od cvorova podskupa sa cvorom van tog skupa. Na ovaj nacin dolazimo do izraza za provodnost grafa
    \[ \partial(S) = \{ \{v_1,v_2\}: v_1 \in S \land v_2 \in (V \setminus S)\} \]
    Provodnost skupa podskupa cvorova S grafa G predstavlja skup grana koje povezuju cvorove iz skupa S sa ostalim cvorovima.


    Kada bismo kao mjeru povezanosti podskupa svorova S uzeli kardinalost provodnosti, to zasigurno ne bi predstavljalo dobru mjeru, iz razloga sto kardinalost
    posmatrnog skupa utice i na kardinalnost provodnosti, tj. za ocekivati da veci skup cvorova S ima i vecu provodnosti. S toga, kao jedna od ideja namece se da izvrsimo normalizajiju sa brojem 
    cvorova u posmatranom skupu. Na taj nacin dolazimo da definicije izometrijskog odnosa skupa S.
    \[
        h(S) = \frac{|\partial(S)|}{|S|}
    \]
    Jasno je da manja vrijednost izometrijskog odnosa $h(S)$ ukazuje na vecu izolovanosti skupa cvorova S.
    Pored izometrijskog odnosa podskupa cvorova definise se i izomterijski odnosr grafa kao
    \[
         h(G) = \min_{|S| <= |V| / 2} h(S)
    \]
    Izometrijski odnos grafa G predstavlja izometrijski odnos njenogovj najizolovanijeg podgrafa.
    
    Bilo jos bolje kada bismo mogli utvrditi donju granicu za izometrijski odnos. Upravo to nam omogucava sljedeca teorema.

    \begin{theorem}[Lower bound of Cheegar]
        Neka je dat povezan graf $G=(V,E)$ i neka je $S \subset V$. Tada vazi
        \[ h(S) \geq \alpha_2 (1 - \frac{|S|}{|V|})\]
        gdje je $\alpha_2$ sopstvena vrijednost Laplasove matrice grafa.

        Jasno je da je za povezan graf ocito koji podskup cvorova je najizolovaniji.
    \end{theorem}

    \begin{custom_proof}
        Prema Courant-Fischer teoremi imamo
        \[
        \alpha_2 = \min_{\vec{x} \perp \vec{1}} \frac{\vec{x}^T L_G \vec{x}}{\vec{x}^T \vec{x}} 
        = \min_{\vec{x} \perp \vec{1}} \frac{ \sum_{\{u,v\} \in E} (\vec{x}_u - \vec{x}_v)^2}{\vec{x}^T \vec{x}}
        \]

        Posmatrajmo sada vektor cije su koordinate date Sa
        \[
            \vec{v} = 
            \begin{cases}
                1 - \frac{|S|}{|V|} & i \in S\\ 
                - \frac{|S|}{|V|} & i \in V \setminus S
            \end{cases}
        \]

        Motivacija za posmatranje ovog vektora je vise, prije svega ovaj vektor je ortogonalna jedinicnom vektoru.

        \[
        \begin{split}
            \vec{v}^T\vec{1} & = \sum_{i \in \{1,2, \dots n\}} \vec{v}_i * 1 \\
            & = \sum_{i \in \{1,2, \dots n\}} \vec{v}_i \\
            & = |S| * (  1 - \frac{|S|}{|V|} ) + ( -\frac{|S|}{|V|}) * |V \setminus S| \\
            & = |S| - |S| * \frac{|S|}{|V|} - \frac{|S|}{|V|} * |V \setminus S| \\
            & = |S| - \frac{|S|}{|V|} * (|V \setminus S| + |S|) \\
            & = |S| - \frac{|S|}{|V|} * |V| \\
            & = |S| - |S| = 0 
        \end{split}
        \]

        Uocima da je $\vec{v}_u - \vec{v}_v = 0$ ukoliko oba cvora pripadaju ili ne pripadaju skupu S, a da je $\vec{v}_u - \vec{v}_v = 1$ ukoliko tacno jedan od cvorova u i v pripada skupu S.
         
        \[
        \begin{split}
            \alpha_2  & \leq \frac{ \sum_{\{u,v\} \in E} (\vec{v}_u - \vec{v}_v)^2}{\vec{v}^T \vec{v}} \\
            \alpha_2  & \leq \frac{ |\partial(S)|}{\sum \vec{v}_i^2} 
        \end{split}
        \]
        
        Odredimo sada modul vektora v
        \[
            \begin{split}
                \vec{v}^T\vec{v} & = \sum_{i \in \{1,2, \dots n\}} v_i \\ 
                \vec{v}^T\vec{v} & = |S| * (1 - \frac{|S|}{|V|})^2 + |V \setminus S| (-\frac{|S|}{|V|})^2 \\ 
                \vec{v}^T\vec{v} & = |S| - 2 * |S| * \frac{|S|}{|V|} + |S| * (\frac{|S|}{|V|})^2 + |V \setminus S| (\frac{|S|}{|V|})^2 \\ 
                \vec{v}^T\vec{v} & = |S| - 2 * |S| * \frac{|S|}{|V|} + (|S| + |V \setminus S|) * (\frac{|S|}{|V|})^2 \\ 
                \vec{v}^T\vec{v} & = |S| - 2 * |S| * \frac{|S|}{|V|} + |V| * (\frac{|S|}{|V|})^2 \\ 
                \vec{v}^T\vec{v} & = |S| - 2 * \frac{|S|^2}{|V|} + (\frac{|S|^2}{|V|}) \\ 
                \vec{v}^T\vec{v} & = |S| - \frac{|S|^2}{|V|} \\ 
                \vec{v}^T\vec{v} & = |S| * ( 1 - \frac{|S|}{|V|}) \\ 
            \end{split}
            \]
            
            Nakon uvrstavanja dobijenog izraza, dobijamo
            \[
            \begin{split}
                \alpha_2  & \leq \frac{ \sum_{\{u,v\} \in E} (\vec{v}_u - \vec{v}_v)^2}{\vec{v}^T \vec{v}} \\
                \alpha_2  & \leq \frac{ |\partial(S)|}{|S| * ( 1 - \frac{|S|}{|V|})} \\ 
                \alpha_2 * |S| * ( 1 - \frac{|S|}{|V|}) & \leq |\partial(S)| \\
                \alpha_2 * ( 1 - \frac{|S|}{|V|}) & \leq \frac{|\partial(S)|}{|S|} \\ 
                \frac{|\partial(S)|}{|S|} & \geq  \alpha_2 * ( 1 - \frac{|S|}{|V|}) \\ 
                h(S) & \geq  \alpha_2 * ( 1 - \frac{|S|}{|V|}) \\ 
                h(G) & = \min_{|S| \leq |V| / 2}  \alpha_2 * ( 1 - \frac{|S|}{|V|}) \\ 
                h(G) & = \alpha_2  - \max_{|S| \leq |V| / 2}  \alpha_2 * \frac{|S|}{|V|} \\ 
                h(G) & = \alpha_2  - \alpha_2 * \frac{|V|/2}{|V|} \\ 
                h(G) & = \alpha_2  * (1 - 1/2) \\ 
                h(G) & = \alpha_2 / 2  
            \end{split}
            \]

            Dobijeni izraz je jos jedna potvrda cinjenice da najmanja pozitivna sopstvena vrijednost laplasove matrice pokazuje koliko dobro je graf povezan.
            Ukoliko je ova sopstvena vrijednmost veca, veci je i izomterijski odnos grafa, tj. njenogoj najizoolovanij dio je bolje povezan sa ostatkom grafa.

    \end{custom_proof}



    Spektralna teorema grafova je jako zastupljena u raynim naucnim olastima, narocito u racunarskim naukama.
    Laplasova matrica i njen spektar se koristi za spektralno praticionisanja grafova prilikom segmentacije slike rasporednvanja tranzistora na VLSI cipovima. Prilikom klasifikacije dokumanta na sonocu semnatickih pveyanistu izmedju rijeci.
    Prilikon analize podataka i spektralnog uparivanja.

    Fiedler value nad vector

    Graph cut problem se svodi na partcionisanje grafa tako da teyine grana unutar particije budu sto vece a da grane koje povezuju 2 particije budu sto manje

    Spectaln graph embedding on a line

    \begin{example}
        Regulatorna agencija na raspolaganju ima k radio frekvencija. Na konkurs za dodjelu frekvencije prijavilo se n radio stanica, pri cemu je $n \gg k$.
        Prostorno udaljenim radio stanicima ciji se reoni emitovanja ne poklapaju agencija moze dodijeliti istu frekvenciju ali to nije slucaj sa svim radio stanicama.
        Kako ce agencija odrediti kojoj radio stanici ce didijeliti koju radio frekvenciju?

        Kako bismo opravdali pojavljivanje ovoga zadatku u radu vezanom za grafove, naravno da cemo i ovaj put svesti problem na domen grafova.
        Radio stanice cemo predstavljati sa cvorovima a relaciju preklapanja reona emitovanja izmedju dvije stanice cemo modelovati sa granom izmedju cvorova.

        Vec na prvi pogled zadatak se svodi na bojenje ovako formiranog grafa sa k boja, odnosno dodjeljivanje boje (neke vrijednosti) svakom cvoru grafa tako da
        susjednim cvorovima nisu dodijeljene iste boje. Bojenje grafa nije nista drugo do odredjivanje funkcije koja slika skup cvorova u skup dostupnih boja $ c : V \to C$.
        Problem bojenja grafa sa k boja, nije uvijek rjesiv i zavisi od strukture grafa. Ukoliko je pak moguce obojiti dati graf sa k boja za graf G se
        kaze da je "k obojiv". Na osnovu definicije bojenja grafa lako se moze zakljuciti da graf nije "k obojiv" ukoliko bar jedan od cvorova ima vise od $k-1$ susjeda, tj.
        ukoliko vazi $\exists v \in V deg(v) \geq k-1$. Najmanje k za koje je G "k obojiv" naziva se hromatski broj grafa G u oznaci $\chi(G)$.

        Vec na pocetku, nismo sigurni ni da li je moguce rijesiti postavljeni zadatak, ali smanjimo kriterijum i zadovoljimo se time da nakon podjele frekvencija imamo sto manje preklapanja.
        
        Kao bismo rijesili postavljeni zadatak potrebno je odrediti vektor $\vec{c}$ takava da $\vec{c}_i$ ukazuje na to koju boju je potrebno dodijeliti radio stanici i.

        Prisjetimo se cemu je jednaka kvadratna forma Laplasove matrice formiranog grafa.
        \[
            \vec{x}^T L_G \vec{x} = \sum_{\{i,j\} \in E} (\vec{x}_i - \vec{x}_j)^2
        \]
        
        Ukoliko je pak $\vec{x}$ sopstveni vektor Laplasove matrice imamo da je 
        \[
            \vec{x}^T L_G \vec{x} = \sum_{\{i,j\} \in E} (\vec{x}_i - \vec{x}_j)^2 = \alpha
        \]
        
        Ukoliko odaberemo vektore koji odgovaraju najvecim sopstvenim vrijednostima Laplasove matrice, jasno je da u tom slucaju suma iz prethodnog izraza
        dostize svoj maksim, a upravo ona predstavlja sumu razlika izmedju vrijednosti dodijeljenih susjednim cvorovima. S toga, cvorovi kojima ovi sopstveni 
        vekori dodjeljuju bliske vrijednosti teze da budu sto udaljeniji u grafu.

        S toga, kao jedna od ideja namece spektralno particionaisanje ali upotrebom sopstvenih vektora koji odgovara 2 do 3 najvece sopstvene vrijednosti.
        Nakon izracunavanja sopstvenih vektora, njihove elemente je potrebno podijeliti u k grupa na osnovu njihove vrijednosti (upotrebom k-susjeda ili nekog drugog algoritma) i svakoj grupu dodijeliti boju.
        Element sopstvenog vektora $\vec{e}_i$ koji pripada j-toj grupi, kojoj je pridruzena boja $c_j$ ukazuje na to da bismo cvor $v_i$ trebali obojit sa $c_j$ bojom.

    \end{example}



    
	\end{document}
	