	% !TEX TS-program = pdflatex
	% !TEX encoding = UTF-8 Unicode
	
	\documentclass[11pt]{article}
	
	\usepackage[utf8]{inputenc}
	\usepackage{geometry}
	\usepackage{graphicx}
	\usepackage{booktabs}
	\usepackage{array}
	\usepackage{paralist}
	\usepackage{verbatim}
	\usepackage{subfig}
	\usepackage{amsthm}
	\usepackage{amsmath}
	\usepackage{amsfonts}
	\usepackage{graphicx}
	
	\geometry{a4paper}
	
	\usepackage{fancyhdr}
	\pagestyle{fancy}
	\renewcommand{\headrulewidth}{0pt}
	\lhead{}\chead{}\rhead{}
	\lfoot{}\cfoot{\thepage}\rfoot{}
	
	\usepackage{sectsty}
	\allsectionsfont{\sffamily\mdseries\upshape}
	
	\usepackage[nottoc,notlof,notlot]{tocbibind}
	\usepackage[titles,subfigure]{tocloft}
	\renewcommand{\cftsecfont}{\rmfamily\mdseries\upshape}
	\renewcommand{\cftsecpagefont}{\rmfamily\mdseries\upshape}
	
	\newtheorem{definition}{Definicija}
	\newtheorem{theorem}{Teorema}
	\newtheorem*{custom_proof}{Dokaz tvrdjenja}
	\newtheorem{consequence}{Posljedica}
	\newtheorem{example}{Primjer}
	
	\title{Spektralna teorija grafova}
	\author{Željko Tripić}
	\date{06.02.2025}
	
	\begin{document}
	\maketitle
	
	\section{Motivacija}
	Upotreba grafova u računarskim naukama je jako zastupljena. Bilo da se grafovi koriste za skladištenje podataka, 
	kreiranje grafova znanja za nepredna pretraživanje. Izazov prilikom rada sa grafovima u oblasti računarskih nauka je oduvijek
	bila njihova reprezentacija. Grafovi, zbog svoje strukture, nisu najpogodniji za procesiranje u računarima.
	Grafovi se najčešće reprezentuju u vidu matrica.

    ???
	
	\section{Uvod}
	Doći do svojstava grafa proučavanjem svojstava njemu pridruženih matrica. 

    ???

    dodati latinicna slova
	
	\section{Osnovni pojmovi teorije grafova}
	
		\begin{definition}[Graf]
		Graf je uredjen par $G = (V, E)$, gdje je $V$ skup elemenata koji se nazivaju čvorovi, a $E = \{ \{v_1, v_2\} : v_1, v_2 \in V \}$ je skup grana, tj. dvočlanih podskupova skupa cvorova $V$.
		\end{definition}
	
        \paragraph{}
        Primjetimo da se u definiciji grafa, implicitno podrazumijeva da je grana neuredjeni dvočlani skup čvorova. Ukoliko uzmemo u obzir svojstva neuredjenog skupa, dolazimo do zakljucka da $\{v_1,v_2\} = \{v_2, v_1\}$, tj. svaka grana grafa ima dva ekvivalentna zapisa.
    
        \paragraph{}
        Bez umanjenja opštosti, graf sa $n$ čvorova možemo predstavljati sa $G = (V, E)$, gdje je $V = \{1, 2, \dots, n\}$.
	
		\begin{definition}[Petlja]
		Petlja je grana koja spaja cvor grafa $G=(V,E)$ sa samim sobom, tj. grana definisana sa $\{v,v\} \in E$.
		\end{definition}
	
		\begin{definition}[Multigraf]
		Multigraf je graf kod kojeg dva proizvoljna čvora mogu biti povezana sa više grana, koje se u tom slučaju nazivaju višestruke grane.
		\end{definition}
	
		\begin{definition}[Težinski graf]
		Težinski graf $G =(V,E,w)$ je graf kod kojeg je svakoj grani dodijeljena težina, tj. realan pozitivan broj. Pored skupa čvorova i skupa grana, težinski graf je definisan i sa funkcijom $w: E \to \mathbb{R}^+$ koja grani grafa dodjeljuje težinu.
		\end{definition}
	
		\begin{definition}[Usmjeren graf]
		Usmjeren graf $G =(V,E)$ je graf kod kojeg grane imaju usmjerenje, tj. orjentaciju. $E$ je u tom slucaju skup uredjenih parova čvorova $E = \{ (v_1,v_2) : v_1, v_2 \in V\}$. 
		Ukoliko $(v_1,v_2) \in E$, to znaci da graf sadrži granu koja vodi od $v_1$ do $v_2$, ali ne i da postoji grana koja vodi od $v_2$ do $v_1$, sto je posljedica svojstva $(v_1,v_2) \neq (v_2,v_1).$
		\end{definition}
	
		\paragraph{}
		Ukoliko drugačije nije naglašeno, grafovi sa kojima ćemo se baviti u nastavku su podrazumijevano netežinski neusmjereni grafovi koji ne sadrže petlje i višestruke grane.
	
		\begin{definition}[Susjednost čvorova]
		Za čvorove $v_1$ i $v_2$ grafa $G = (V, E)$ kažemo da su susjedni ako u okviru grafa $G$ postoji grana koja povezuje ta dva čvora.
		\[\text{ čvorovi } v_1  \text{ i } v_2 \text{ su susjedni } \Leftrightarrow \{v_1, v_2\} \in E \]
		\end{definition}
		
		\begin{definition}[Incidentnost čvora i grane]
		Za čvor $v$ i granu $e$ grafa $G = (V, E)$ kažemo da su incidentni ako dati čvor predstavlja jedan kraj date grane.
		\[ \text{čvor } v \in V \text{ i grana }\ e \in E \text{ su incidentni} \Leftrightarrow v \in e \Leftrightarrow e = \{i,j\} \land ( v = i  \lor  v = j) \] 
		\end{definition}
	
		\begin{definition}[Okolina čvora]
		Okolina čvora $v$ u oznaci $N_G(v)$ je skup svih čvorova grafa $G$ koji su susjedni sa datim čvorom $v$.
		\[N_G(v) = \{ v_1 : \{v, v_1\} \in E\}\]
		\end{definition}
	
		\begin{definition}[Stepen čvora]
		Stepen (valencija) čvora $v$ u oznaci $\deg(v)$ je broj njemu susjednih čvorova, tj. kardinalnost njegove okoline.
		\[\deg(v) = |N_G(v)|\]
		\end{definition}
		
		\begin{definition}[Regularan graf]
		Graf $G=(V,E)$ je regularan sa stepenom regularnosti $x$ ako je stepen svakog njegovog čvora jednak $x$, tj ako važi 
		\[ deg(v) = x \text{ za } \forall v \in V \]
		\end{definition}
	
		\begin{definition}[Podgraf]
		Neka je dat graf $G = (V, E)$. Podgraf grafa $G$ je svaki graf čiji skup čvorova je podskup skupa čvorova grafa $G$, a skup grana je podskup skupa grana grafa $G$.
		\[ G_1 = (V_1, E_1) \text{ je podgraf od } G \Leftrightarrow V_1 \subseteq V \text{ i } E_1 \subseteq E \]
		\end{definition}
	
		\begin{definition}[Šetnja i-j]
		Neka je dat graf $G = (V, E), V = \{1, 2, \dots, n\}$. Šetnja od čvora $i$ do čvora $j$ u oznaci $i$-$j$ je podgraf $G_1 = (V_1, E_1)$ takav da važi 
		\[
		 	V_1 \subseteq V \land  i \in V_1 \land j \in V_1
		\]
		\[ 
		 	E_1 = \{\{i,v_1\},\{v_1,v_2\},\dots,\{v_{k-1},v_k\},\{v_k,j\}\} \land k \geq 0
		\] 
			\paragraph{}
			Alternativno, šetnja $i$-$j$ je niz čvorova $i, v_1, v_2, \dots, v_k, j \in V$ za koje važi 
			\[
				\{i,v_1\} \in E \land \{v_k,j\} \in E \land \{v_p,v_{p+1}\} \in E \text{ za } \forall p \in \{1,2, \dots, k-1\} 
			\]
			\paragraph{}
			Čvorovi $v_1,v_2 \dots v_k \in V_1 $ ne moraju biti različiti, a dužina šetnje je broj grana u datom podgrafu/nizu.
		\end{definition}
	
		\begin{definition}[Put i-j]
		Put $i$-$j$ je šetnja $i$-$j$ kod koje se su svi čvorovi u nizu različiti, osim možda $i$ i $j$.
		\end{definition}
		
		\begin{definition}[Ciklus i]
		Ciklus je put $i$-$j$ kod kojeg je $i=j$.
		\end{definition}
		
		\begin{definition}[Prost put i-j]
		Prost put $i$-$j$ je put kod kojeg vazi $i \neq j$.
		\end{definition}
	
		\begin{definition}[Povezanost čvorova]
		Za čvorove $v_1$ i $v_2$ grafa $G = (V, E)$ kažemo da su povezani ukoliko postoji $v_1$-$v_2$ šetnja u grafu $G$. 
		\end{definition}
	
		\begin{definition}[Komponenta povezanosti]
		Za podgraf $G_1 = (V_1, E_1)$ grafa $G = (V, E)$ kažemo da je njegova komponenta povezanosti ukoliko je svaki par čvorova podgrafa $G_1$ povezan, a niti jedan njegov čvor nije povezan sa čvorom grafa $G$ koji nije u sastavu podgrafa $G_1$. 
		\[
			\begin{split}
			&\text{podgraf } G_1 = (V_1, E_1) \text{ je komponenta povezanosti grafa } G = (V, E)  \Leftrightarrow \\
			& \forall v_1,v_2 \in V_1 \text{ važi } v_1 \text{ i } v_2 \text{ su povezani, }  \not \exists \{v_3, v_4\} \in E \text{ takva da } v_3 \in V_1 \land v_4 \in (V \setminus V_1)
			\end{split}
		\]
			\paragraph{}
			Iz navedenih definicija povezanosti čvorova i komponenti povezanosti, lako se uočava da je je relacija povezanosti u stvari relacije ekvivalencije na skupu čvorova grafa $G$. Relacija povezanosti čvorova razbija skup $V$ na klase ekvivalencije tj. na disjunktne skupove čvorova koji odgovaraju skupovima cvorova komponenti povezanosti grafa $G$. 
		\end{definition}
	
		\begin{definition}[Povezan graf]
		Povezan graf je graf $G=(V,E)$ koji ima samo jednu komponentu povezanosti, ako $\forall v_1,v_2 \in V $ postoji šetnja $v_1$-$v_2$
		\end{definition}
	
		\begin{definition}[Kompletan graf]
		Kompletan graf sa $n$ čvorova u oznaci $K_n$ je graf kod kojeg je svaki čvor povezan sa svim ostalim čvorovima grafa. 
		\[ 
			K_n = (V,E) \text{ gdje } |V| = n \text{ i } v_1,v_2 \in V \Rightarrow \{v_1,v_2\} \in E
		\]
		\end{definition}
		
		\begin{definition}[Put]
		Put sa $n$ cvorova je graf $P_n=(V,E)$ za koga važi $V = \{1,2, \dots ,n\}$, $E = \{\{i, i+1\} : i \in \{1,2, \dots , n-1\} \}$.
		\end{definition}
		
		\begin{definition}[Ciklus]
		Ciklus sa $n$ cvorova je graf $C_n=(V,E)$ za koga važi $V = \{1,2, \dots ,n\}$, $E = \{\{i, i+1\} : i \in \{1,2, \dots , n-1\} \} \cup \{\{1,n\}\}$.
		\end{definition}
	
		\begin{definition}[Stablo]
		Stalo je povezan graf koji ne sadrži ciklus.
		\end{definition}
	
		\begin{definition}[Razapinjujući graf]
		Razapinjujući graf grafa $G = (V,E)$ jeste njegov podgraf koji sadrži sve čvorove
		\[
			G' = (V, E_1) \text{ gdje je } E_1 \subseteq E \Leftrightarrow G' \text{ je razapinjujuci graf} 
		\]
		\end{definition}
	
		\begin{definition}[Razapinjujuće stablo]
		Razapinjujuće stablo je razapinjujući graf koji je ujedno i stablo.
		\end{definition}
	
		\begin{definition}[Kompleksnost grafa]
		Kompleksnost grafa $G$ u oznaci $k(G)$ je ukupan broj njegovih razapinjujućih stabala. 
		\end{definition}
	
	\section{Grafovi i njima pridružene matrice}
	Svakom grafu se mogu pridružiti matrica susjedstva, matrica incidencije, matrica stepeni i Laplasova matrica.
	Za spektralnu teoriju grafova od velikog značaja je Laplasova matrica.
	
		\subsection{Matrica susjedstva}
		Matrica susjedstva je najčešći vid reprezentacije grafa pomoću matrice. 
		Vrste i kolone matrice susjedstva su indeksirane čvorovima grafa $G$.  
		Matrica susjedstva je dimenzije $|V| \times |V|$ i element matrice susjedsta u $i$-tom redu i $j$-toj koloni nosi informaciju o susjedstvu čvorova sa indeksima $i$ i $j$.
	
			\begin{definition}[Matrica susjedstva] 
			Neka je dat graf $G = (V, E)$, gdje je $V = \{1, 2, \dots, n\}$. Matrica $A_G = [a_{ij}]_{n \times n}$ koja odgovara grafu $G$ i koja je definisana je sa
			 \[
				 a_{ij} =
				 \begin{cases} 
				 0 & \text{ako } \{i, j\} \notin E \\ 
				 1 & \text{ako } \{i, j\} \in E
				 \end{cases}
			 \]
			naziva se matrica susjedstva (eng. adjacency matrix) grafa $G$.
				\paragraph{}
				Suma svih elemenata u $i$-toj vrsti/koloni matrice susjedstva predstvlja stepen cvora sa indeksom $i$,
				tj. vazi $\sum_{k=1}^{n}a_{ik} = \sum_{k=1}^{n}a_{ki} = deg(i) \text{ za } \forall i \in \{1,2, \dots, n\}$.
				Iz definicije matrice susjedstva neusmjerenog grafa jasno je da se radi o realnoj simetricnoj matrici, sto je cini ermitskom matricom. 
			\end{definition}
	
			\begin{definition}[Matrica susjedstva tezinskog grafa] 
			Matrica susjedstva tezinskog grafa  $G = (V, E, W)$ je data matricom tezina grafa, tj. $A_G = W$.
			\end{definition}
	
		\subsection{Matrica incidencije}
		Matrica incidencije grafa $G$ je realna matrica dimenzije $|V| \times |E|$ kod koje su vrste indeksirane čvorovima, a kolone granama grafa $G$.
		Matrica incidencije grafa $G$ nosi informacije o incidentnosti čvorova i grana grafa. 
		Preciznije, element matrice incidencije u $i$-toj vrsti i $j$-toj koloni nosi informaciju o incidenciji čvora sa indeksom $i$ i grane sa indeksom $j$.
		U literaturi se mogu pronaci razlicite definicije matrice incidencije, ali mi cemo ovdje navesti samo jednu koja se ispostavila kao najpogodnija za dobijanje dodatnih saznanja o grafu $G$.
	
			\begin{definition}[Matrica incidencije] 
			Neka je dat graf $G = (V, E)$, gdje je $V = \{1, 2, \dots, n\}$. Matrica $I_G = [b_{ij}]_{n \times m}$, gdje je $n = |V|$ i $m = |E|$, koja odgovara grafu $G$ i koja je definisana sa
			 \[
				 b_{ij} =
				 \begin{cases}
				 1 & \text{ako } e_j = \{v_1,v_2\} \land ( i = v_1 \lor i=v_2) \text{, gdje je } e_j \in E \text{ grana sa indeksom j} \\
				 0 & \text{u suprotnom}
				 \end{cases}
			 \]
			naziva se matrica incidencije grafa $G$.
			\end{definition}
	
			\begin{definition}[Matrica incidencije] 
			Neka je dat usmjeren graf $G = (V, E)$, gdje je $V = \{1, 2, \dots, n\}$. Matrica $I_G = [b_{ij}]_{n \times m}$, gdje je $n = |V|$ i $m = |E|$, koja odgovara grafu $G$ i koja je definisana sa
			 \[
				 b_{ij} =
				 \begin{cases}
				 1 & \text{ako } e_j = (v_1,v_2) \land i = v_1 \text{, gdje je } e_j \in E \text{ grana sa indeksom j} \\
				-1 & \text{ako } e_j = (v_1,v_2) \land i = v_2 \text{, gdje je } e_j \in E \text{ grana sa indeksom j} \\
				 0 & \text{u suprotnom}
				 \end{cases}
			 \]
			naziva se matrica incidencije grafa $G$.
			\end{definition}
	
	
			Matrica incidencije nije najpogodnija za analizu putem metoda linearne algebre, jer nije kvadratna, niti simetricna matrica.
	
		\subsection{Matrica stepeni}
		Matrica stepeni (eng. degree matrix) grafa $G$ je realna dijagonalna matrice dimenzije $|V| \times |V|$ kod koje su vrste i kolone indeksirane čvorovima grafa $G$.
		Elementi na glavnoj dijagonali matrice stepeni imaju vrijednost stepena cvora koji odgovara indeksu vrste/kolone. 
	
			\begin{definition}[Matrica stepeni] 
			Neka je dat graf $G = (V, E)$, gdje je $V = \{1, 2, \dots, n\}$. Matrica $D_G = [d_{ij}]_{n \times n}$, gdje je $n = |V|$ , koja odgovara grafu $G$ i koja je definisana sa
			\[
				d[ij] = 
				 \begin{cases}
				 deg(i) &  \text{ ako } i =j  \\ 
				 0  & \text{ ako } i \neq j 
				 \end{cases}
			\]
			naziva se matrica stepeni grafa $G$.
			\end{definition}
	
		\subsection{Laplasova matrica}
	
			\begin{definition}[Laplasova matrica] 
			Neka je dat graf $G = (V, E)$, gdje je $V = \{1, 2, \dots, n\}$. Matrica $L_G = [l_{ij}]_{n \times n}$ grafa $G$, definisana sa $L_G = D_G - A_G$ naziva se Laplasova matrica grafa $G$.
			Laplasova matrica uspostavlja relaciju izmedju dvije prethodno definisane matrice, matrice susjedstva i matrice stepeni grafa $G$.
			Iz navedenog izraza se lako vidi da su elementi Laplasove matrice dati sa
			 \[
				 l_{ij} =
				 \begin{cases}
				 \deg(i)  & \text{ako } i = j \\
				 -1 & \text{ako } \{i, j\} \in E \\ 
				 0  & \text{u suprotnom}
				 \end{cases}
			 \]
			
			Na osnovu definicije Laplasove matrice mozemo zakljuciti da je matrica realna i simetricna.
			Zanimljivo svojstvo Laplasove matrice je da je zbir elemenata svake vrste/kolone Laplasove matrice jednak nuli.
			\end{definition}
	
			\paragraph{}
			Iako smo već dali definiciju Laplasove matrice, ispostavlja se da ta definicija nije najpogodnija za ispitivanje njenih svojstava. Iz tog razloga dacemo jos neke karakterizacije Laplasove matrice.
	
			\begin{definition}{Laplasova matrica}
			Neka je dat graf $G = (V, E)$, gdje je $V = \{1, 2, \dots, n\}$. 
			\[
				L_G = \sum_{\{v_1,v_2\} \in E} L_{\{v_1,v_2\}}
			\] 
			
			Matrica $L_{\{v_1,v_2\}} = [l_{ij}]_{n \times n}$ je matrica definisana sa
			\[
				 l_{ij} =
				 \begin{cases}
				 1  & \text{ako } i = j  \land i \in \{v_1,v_2\} \\
				 -1 & \text{ako } \{i,j\} = \{v_1,v_2\} \\ 
				 0  & \text{u suprotnom}
				 \end{cases}
			 \]
			i može se posmatrati kao Laplasova matrica koja odgovara grani $\{v_1.v_2\}$ grafa $G$. Ova matrica je definisana za svaku granu grafa $G$. Iz definicije matrice $L_{\{v_1,v_2\}}$ lako se vidi da je ona realna i 					simetricna.
			\end{definition}
			
			\begin{custom_proof}
			Nakon sumiranja matrica $L_{\{v_1,v_2\}}$ po svim granama grafa $G$ dobijamo:
				\begin{itemize}
				\item element na glavnoj dijagonali $a_{ii}$ ce imati vrijednost $p_i$ koja odgovara broju granu kojima je dati cvor incidentan, tj. vrijednost elementa $a_{ii} = deg(i)$
				\item element $a_{ij}$ van glavne dijagonale ($i \neq j$) matrice ce imati vrijednost -1 samo ako graf $G$ sadrzi granu $\{i,j\}$, u suprotnom vrijednost 0.  
				\end{itemize}
			\end{custom_proof}
		
			\paragraph{}
			U nastavku cemo dati jos jednu njenu karakterizaciju koju cemo kasnije iskoristiti za dokazivanje Kirhofove teoreme.
	
			\begin{definition}[Laplasova matrica] 
			Neka je dat graf $G = (V, E)$, gdje je $V = \{1, 2, \dots, n\}$, $|E| = m$. Svakoj grani grafa $G$ dodijelimo proizvoljnu orjentaciju i formirajmo matricu incidencije $I_G$ u odnosu na izabranu orijentaciju.
			Tada vazi
			 \[
				 L_G = I_G I_G^T
			 \]
			\end{definition}
	
			\begin{custom_proof}
			\[
				I_G I_G^T = I  I^T  = [b_{ij}]_{n \times n}
			\]
			\[
				 b_{ij} = \sum_{k=1}^{m} I_{ik} I_{kj}^T
			 \]
			Iskoristimo svojstvo transponovane matrice koje kaze $M_{ij}^T = M_{ji}$
			\[
				 b_{ij} = \sum_{k=1}^{m} I_{ik} I_{jk}
			 \]
			Element $I_{ik}$ nosi informaciju o incidentnosti cvora sa $i$ i grane $k$, a Element $I_{jk}$ nosi informaciju o incidentnosti cvora sa $j$ i grane $k$.
			\[
				 I_{ik}I_{jk} =
				 \begin{cases}
				1  & \text{ ako je } i = j \text{ incidentan sa granom k, tj. ako }  i = j \in k \\
				 -1  & \text{ ako su oba cvora i i j incidentni sa granom k, tj. ako } \{i,j\} \in E \\
				 0 & \text{ ako bar jedan od cvorova i i j nije incidentan sa granom k, tj, ako } i \notin k \lor j\notin k 
				 \end{cases}
			 \]
			Jasno je da $I_{ik}I_{jk} = -1$ za fiksno i i j, moguce samo za jedno $k$ jer graf ne sadrzi visestruke grane. S toga ,vazi
			\[
				\sum_{k=1}^{m} I_{ik}I_{jk} =
				 \begin{cases}
				deg(i)  & \text{ za } i = j \\
				 -1  & \text{ za } \{i,j\} \in E \\
				 0 & \text{ u suprotnom }
				 \end{cases}
			 \]
			Prethodni izraz predstavlja upravo definiciju elemenata Laplasove matrice. Uocimo da tvrdjenje vazi bez obzira koju orijentaciju grana grafa odaberemo.
			\end{custom_proof}
	
	\section{Prebrojavanje šetnji izmedju dva proizvoljna čvora grafa}
		\paragraph{}
		U nauci, ali i nasoj svakodnevnici cesti su problemi koji se mogu modelovati i rijesiti pomocu grafova. 
		Mnogi od tih problema se u domenu grafova svode na pitanje da li od nekog cvor $v_1$ mozemo doci do cvora $v_2$, i ukoliko je moguce, na koliko nacina je to moguce postici. Neki takvig problema su:
				\begin{itemize}
				\item Na koliko nacina je moguce poslati, tj. rutirati poruka kroz racunarsku mrezu od racunara A do racunara B? Racunari su modelovani cvorovima grafa, a direktne veze izmedju uredjaja sa granama grafa.
				\item Na koliko nacina DNK sekvenca A moze da mutira u DNK sekvencu B? DNK sekvence su modelovane cvorovima, a proste mutacije supstitucije gena granama grafa. 
				\item Na koliko nacina je u lancanoj hemijskoj reakciji od molekula A dobiti molekul B. Molekuli su modelovani cvorovima, a reakcije sa granama grafa.
				\end{itemize}
	
		Ukoliko sada jos podignemo kriterij i zahtijevamo rjesavanje problema u odredjenom broju koraka, npr. da $n$ uredjaja posreduje u razmjeni poruke u racunarskoj mrezi, 
		da se odigra tacno $n$ mutacija gena ili se 	ogranicimo na $n$ reakcija u lancu, lako se vidi, da je u pozadini problem odredjivanja broja $i$-$j$ setnji duzine $n$ u grafu.
	
		\paragraph{}
		Kako bismo odredili broj šetnji dužine $n$ izmedju dva proizvoljna čvora grafa, pogodno je graf $G = (V, E)$ predstaviti u obliku matrice susjedstva. 
		Naredna teorema daje relaciju izmedju elemenata matrice susjedstva i broja šetnji izmedju čvorova grafa.
	
		\begin{theorem}
		Neka je dat graf $G = (V, E), V = \{1, 2, \dots, n\}$. Element matrice susjedstva $A_G$ u presjeku i-te vrste i j-te kolone označićemo sa $a_{ij}$, 
		a element m-tog stepena matrice susjedstva $A_G^m$ u presjeku i-te vrste i j-te kolone označićemo sa $a_{ij}^m$.
		
		\paragraph{}
		Element $a_{ij}^m$ matrice $A_G^m$ predstavlja broj $i$-$j$ šetnji dužine $m$. 
		\end{theorem}
	
		\begin{custom_proof}
			Datu teoremu ćemo dokazati putem matematičke indukcije.  Radi jednostavnosti zapisa koristićemo $A_G = A$.
			
			\paragraph{Baza indukcije}
			Za $n=1$, jasno je da $a_{ij}^1 = a_{ij}$ predstavlja broj šetnji dužine 1 izmedju čvorova, jer to odgovara broju grana kojima su povezani čvorovi $i$ i $j$. 
			Kod grafa koji nije multigraf, broj grana izmedju čvorova $i$ i $j$ jednak je vrijednosti elementa $a_{ij}$ matrice $A_G$, tj. 0 ukoliko čvorovi nisu susjedni ili 1 ukoliko su čvorovi susjedni. 
			Nije moguće imati šetnju $i$-$j$ dužine 1 ukoliko čvorovi $i$ i $j$ nisu susjedni, tj. direktno povezani granom grafa. 
			
			\paragraph{Induktivna pretpostavka}
			Pretpostavimo da tvrdjenje važi za $k>1$.  Tada, $a_{ij}^k$ predstavlja broj šetnji dužine $k$ izmedju čvorova $i$ i $j$ u grafu $G$ za $\forall i,j \in V$. 
			
			\paragraph{Induktivni korak}
			Pokažimo sada da tvrdjenje važi i za $k+1$.
			Iskoristimo pravila operacije množenja matrica, da bi smo odredili vrijednost elementa $a_{ij}^{k+1}$ matrice $A_G^{k+1}$.  
			\[
				A_G^{k+1} = A_G^k * A_G
			\]
			\[
				a_{ij}^{k+1} = A_{i.}^k * A_{.j} 
			\]
			\begin{equation}
			a_{ij}^{k+1} = \sum_{p=1}^{n} a_{ip}^k * a_{pj}
			\label{eq:element}
			\end{equation}
		
			\paragraph{}
			Vrijednost elementa $a_{ij}^{k+1}$ matrice $A_G^{k+1}$ se dobija kao proizvod i-te vrste matrice $A_G^k$ i j-te kolone matrice $A_G$. 
			
			\paragraph{}
			Svaku $i$-$j$ šetnju dužine veće od 1 možemo predstaviti kao $i$-$p$ šetnju na koju nadovežemo granu $\{p,j\} \in E$ (što je ekvivalentno šetnji dužine 1), gdje je $p$ proizvoljan čvor grafa $G$. 
			Zakljucujemo da se broj $i$-$j$ šetnji dužine $k+1$ može dobiti tako što za svaki čvor $p$ grafa $G$ prebrojimo $i$-$p$ šetnje dužine $k$ i dodamo ih u konačnu sumu ako graf $G$ sadrži granu $\{p,j\}$,
			tj. ako su čvorovi $p$ i $j$ susjedni.
		
			\paragraph{}
			Ako sada uzmemo u obzir da prema induktivnoj pretpostavci $a_{ip}^k$ predstavlja broj $i$-$p$ šetnji dužine $k$ i da prema definiciji matrice susjedstva  $a_{pj}=1$ ako graf $G$ sadrži granu $\{p,j\}$, 
			izraz $a_{ip}^k * a_{pj}$ predstavlja ukupan broj $i$-$j$ šetnji dužine k+1 koje koje su konstruisane na prethodno izlozeni nacin. 
			Sumiranjem izraza po svim čvorovima grafa $G$ dobija se ukupan broj $i$-$j$ šetnji dužine $k+1$ predstavljen izrazom $\sum_{p=1}^{n} a_{ip}^k * a_{pj}$. 
			Vrijednost ove sume, prema jednakosti (1), jednaka je elementu $a_{ij}^{k+1}$ matrice $A_G^{k+1}$ cime dokaz je završen.
		\end{custom_proof}
	
		\paragraph{}
		Sada, iako znamo kako da prebrojimo $i$-$j$ setnje u zadatom grafu, nailazimo na problem koji proizilazi iz same prirode setnji koja im dozvoljava da vise puta posjete neke od cvorova grafa.
		Na primjer, naredne setnje su u uracunate u konacnu sumu dobijenu putem prethodne teoreme a nisu prosti putevi:
		\[
			\begin{split}
			i \rightarrow v_1 \rightarrow i \\
			i \rightarrow  j \rightarrow v_1 \rightarrow j \\
			i \rightarrow j \rightarrow v_1 \rightarrow v_2 \rightarrow j \\
			i \rightarrow v_1 \rightarrow j \rightarrow v_2 \rightarrow j \\
			i \rightarrow v_1 \rightarrow i \rightarrow v_2 \rightarrow j \\
			i \rightarrow v_1 \rightarrow v_2 \rightarrow i \rightarrow j 
			\end{split}
		\]
		 Bilo bi korisno kada bismo na slican nacin mogli prebrojati proste puteve $i$-$j$ duzine $n$ u zadatom grafu. Ovo ogranicenje je sasvim prirodno, jer mozda, u kontekstu rutiranja poruke kroz racunarsku mrezu,
		ne zelimo da isti uredjaj vise puta posreduje u prenosu jedne poruke.  U nastavku cemo po ugledu na prethodnu primjenu, pokusati da iskoristimo matricu susjedstva $A_G$ kako bismo prebrojali proste puteve $i$ - $j$.
		
		\subsection{Prebrojavanje prostih puteva izmedju dva proizvoljna čvora grafa}
	
			\begin{theorem}
			Neka je dat graf $G = (V, E), V = \{1, 2, \dots, n\}$. Element $s_{ij}^{(2)}$ matrice $S_G^{(2)}$ predstavlja broj prostih puteva $i$-$j$ dužine 2. Matrica $S_G^{(2)}$ je definisana sa
			\[
			S_G^{(2)} = A_G^{2} - D_G
			\]  
			gdje je $A_G$ matrica susjedstva, a $D_G$ matrica stepeni grafa $G$.
			Uociti da $S_G^{(2)} \neq S_G^{2}$, tj. $S_G^{(2)}$ ne predstavlja drugi stepen matrice $S_G$.
			\end{theorem}
		
			\begin{custom_proof}
			Kada su u pitanju setnje $i$-$j$ duzine 2, problematicne su samo one setnje koje pocinju i zavrsavaju u istom cvoru grafa, tj. one koje predstavljaju ciklus. 
		 	Na primjer sve setnje oblika $i \rightarrow v \rightarrow i$ gdje $v \in (V \setminus \{i\})$ su ciklusi a ukljucene u konacan broj. Jasno je da $v \neq i$ jer razmatramo grafove bez petlji. 
			Kod ovakvih setnji polazimo iz cvora $i$, idemo do njegovog susjeda $v$ in onda se vracamo nazad u cvor $i$ istom granom.
			Kako su u pitanju setnje $i$-$i$, njihov broj je sadrzan u vrijednostima elemenata $a[ii]^2$ matrice $A_G^2$ (elementi na glavnoj dijagonali matrice). 
			Jasno je da je broj ovakvih setnji jednak broju susjeda cvora i, tj. $deg(i)$. Takodje, jasno je i da su  
			S obzirom da znamo da je broj problematicnih setnji $deg(i)$ i da znamo da njihov broj utice samo na elemente na glavnoj dijagonali matrice $A_G^2$, 
			dolazimo do zakljucka da je broj $i$-$j$ prostih puteva duzine 2 dat izrazom
				\[
				s_{ij}^{(2)} = 
				 \begin{cases}
				 0 = a_{ij} - deg(i) &  \text{ ako } i =j  \\ 
				 a_{ij} = a_{ij} - 0   & \text{ ako } i \neq j 
				 \end{cases}
				\]  
		
			Matrica definisana sa prethodnim izrazom je upravo matrica $S_G^{(2)}$ data u teoremi, cime je tvrdnja dokazana.
			\end{custom_proof}
	
			\begin{theorem}
			Neka je dat graf $G = (V, E), V = \{1, 2, \dots, n\}$. Element $s_{ij}^{(3)}$ matrice $S_G^{(3)}$ predstavlja broj i-j šetnji dužine 3. Matrica $S_G^{(3)}$ je definisana sa
			\[
			S_G^{(3)} = S_G^{(2)} A_G - A_G(D_G - I) - Diag(S_G^{(2)}A_G) 
			\]  
			gdje $Diag([a_{ij}]) $ definisana sa
			\[
			diag_{ij} = 
			 \begin{cases}
			a_{ij} &  \text{ ako } i = j  \\ 
			 0   & \text{ ako } i \neq j 
			 \end{cases}
			\]
			\end{theorem}
	
			\begin{custom_proof}
			Ideja je opet ista, prost put duzine 3 predstaviti kao prost put duzine 2 na kojeg je nadovezana jos jedna grana grafa. Posmatrajmo prost put $i$-$v$ duzine 2 na koga je nadovezana grana $\{v,j\}$. 
			Naravno $v \neq j$ jer u tom slucaju graf bi sadrzavao petlju $\{j,j\}$ i $v \neq i$, jer u tom slucaju $i$-$v$ nije prost put, sto je nasa pretpostavka. 
			Konstrukcija ove setnje je moguca samo ako su cvorovi $v$ i $j$ susjedni. Broj tako konstruisanih setnji, gdje je $v$ fiksan cvor, je $s_{iv}^{(2)} * a_{vj}$.
			Nakon sumiranja po svim cvorovima grafa, osim $i$ i $j$, dobijamo $\sum_{v \in (V \setminus \{i,j\})} (s_{iv}^{(2)} * a_{vj}) $.
			Da bi ovako konstruisana setnja $i$-$j$ bila prost put mora da vazi 	da prost put $i$-$v$ ne posjecuje cvor $j$. 
			Da bismo odredili broj prostih puteva $i$ - $j$ duzine 3 moramo da eliminisemo sve setnje oblika $i \rightarrow j \rightarrow v \rightarrow j$. 
			Ukoliko je moguce konstruisati ovakve setnje njihov broj odgovara broju grana $\{j,v\}$ gdje $v \in (V \setminus \{i,j\})$. Jasno je da 
			\[
			% card\{\{j,v\}:  v \in (V \setminus \{i,j\})\} =
			%  \begin{cases}
			%  deg(j)      & \text{ ako } \{i,j\} \notin E  \\ 
			%  deg(j) - 1 & \text{ ako } \{i,j\} \in E   
			%  \end{cases} 
			\]
		
			Uocimo da ukoliko vazi $\{i,j\} \notin E$ onda setnja $i \rightarrow j \rightarrow v \rightarrow j$ ni ne postoji.
			S obzirom na sve navedeno potrebno je iskljuciti $a_{ij} * (deg(j) - 1)$ setnji iz konacnog broja.
		
			\[
			s_{ij}^{(3)} =  \sum_{v \in (V \setminus \{i,j\})} (s_{iv}^{(2)} * a_{vj}) - (a_{ij} * (deg(j)  - 1))
				       = \sum_{v \in (V \setminus \{i,j\})} (s_{iv}^{(2)} * a_{vj}) - (a_{ij} * deg(j)) - a_{ij}
			\] 
		
			Iako smo rekli da $v \neq j$, mozemo sumirati i za vrijednost $v=j$ jer u tom slucaju $a_{vj} = a_{jj} = 0$.
			Takodje, smo rekli da $v \neq i$, ali mozemo sumirati i za vrijednost $v=i$ jer u tom slucaju $s_{iv}^{(2)} = s_{ii}^{(2)} = 0$.
			
			\[
			s_{ij}^{(3)} =  \sum_{v \in V} (s_{iv}^{(2)} * a_{vj}) - (a_{ij} * (deg(j)  - 1))
				       = \sum_{v \in V} (s_{iv}^{(2)} * a_{vj}) - (a_{ij} * deg(j)) - a_{ij}
			\] 
		 
			Uocima da izraz $a_{ij} * deg(j)$ predstavlja mnozenje svih elemenata u istoj koloni matrice susjedstva sa istim skalarom, sto odgovara mnozenju matrice susjedstva sa dijagonalnom matricom, 
			u ovom konkretnom slucaju, matricom stepeni.
			Prethodni izraz prestavlja broj prostih puteva $i-j$ duzine 3 za $i \neq j$. Za $i = j$ broj prostih puteva je nula pa mora da vazi $s_{ii} = 0$ za $\forall i \in V$. 
			Da bismo to postigli za elemente na glavnoj dijagonali moramo anulirati vrijednost $\sum_{v \in E} (s_{iv}^{(2)} * a_{vj}) - a_{ij} * (deg(j)  - 1)$.
			Kako je $a_{ij} = 0$ za elemente na glavnoj dijagonali matrice susjedstva, ostaje nam da anuliramo vrijednost $\sum_{v \in E} (s_{iv}^{(2)}*a_{vj}) = \sum_{v = 1}^{n} (s_{iv}^{(2)}*a_{vj}) = S_{i.}^{(2)} A_{.j}$.
			To mozemo postici tako sto cemo od izraza oduzeti matricu ciji elementi na glavnoj dijagonali imaju vrijednost odgovarajuceg elementa matrice $S_G^{(2)} A_G$, a ostali elementi su jednaki 0.
			Kada prethodni izraz prikazemo u matricnom zapisu dobijamo
			\[
				\begin{split}
				S_G^{(3)} & = S_G^{(2)} A_G - A_G D_G - A_G  - Diag(S_G^{(2)} A_G) \\
					      & = S_G^{(2)} A_G - A_G D_G - A_G I - Diag(S_G^{(2)} A_G)  \\
					      & = S_G^{(2)} A_G - A_G (D_G - I) - Diag(S_G^{(2)} A_G)
				\end{split}
			\]
			\end{custom_proof}
	
			\paragraph{}
			Razmotrimo i broj prostih puteva i-j duzine 4. Polazimo od iste ideja, a to je predstaviti prost put duzine 4 preko prostog puta duzine 3 na koji je nadovezana jos jedna grana grafa.
			Posmatrajmo prosti put $i$-$v$ duzine 3 na koga je nadovezana grana $\{v,j\}$. Naravno $v \neq j$ jer u tom slucaju graf bi sadrzavao petlju $\{j,j\}$ i $v \neq i$, jer u tom slucaju $i$-$v$ nije prost put, 
			sto je nasa pretpostavka.  Konstrukcija ove setnje je moguca samo ako su cvorovi $v$ i $j$ susjedni. Broj tako konstruisanih setnji, gdje je $v$ fiksan cvor, je
			\[
				\sum_{v \in  (V \setminus \{i,j\})} s_{iv}^{(3)} * a_{vj}
			\]
			Ponovo iz istih razloga mozemo dopustiti sumiranje i po cvoru $j$.
			\[
				\sum_{v \in (V \setminus \{i\})} s_{iv}^{(3)} * a_{vj}
			\]
			\paragraph{}
			Konstruisane setnje se mogu predstaviti sa $i \rightarrow v_1 \rightarrow v_2 \rightarrow v \rightarrow j$. Naravno da bi $i$ - $v$ bio prost put mora da vazi $i \neq v_1 \neq v_2 \neq v$,
			a da bi ovako konstruisana setnja $i$-$j$ bila prost put mora da vazi da prosti put $i$-$v$ ne posjecuje cvor $j$.
			To znaci da moramo eliminisati setnje koje oblika $i \rightarrow j \rightarrow v_2 \rightarrow v \rightarrow j$ i $i \rightarrow v_1 \rightarrow j \rightarrow v \rightarrow j$, tj. iz prosti puteva $i$-$v$ moramo izuzeti 
			$i \rightarrow j \rightarrow v_2 \rightarrow v$ i $i \rightarrow v_1 \rightarrow j \rightarrow v$.
		
			\paragraph{}
			Prost put $i \rightarrow j \rightarrow v_2 \rightarrow v$ su moguc samo ako su $i$ i $j$ susjedni cvorovi i za fiksan cvor $v$ ima ih koliko i prostih puteva $j \rightarrow v_2 \rightarrow v$. 
			S tim da moramo da pazimo da $v_2 \neq i$, tj. moramo izuzeti setnju $j-i-v$. Ne moramo da vodimo racuna da li je $v=i$ jer ne vrsimo sumiranje u tom slucaju.
			\[
				\sum_{v \in (V \setminus \{i\})} \{a_{vj} * [s_{iv}^{(3)} - a_{ij} * (s_{jv}^{(2)} - a_{ij}a_{jv})]\}
			\]
		
			\paragraph{}
			Sa druge strane, prostih puteva $i \rightarrow v_1 \rightarrow j \rightarrow v$ za fiksan cvor $v$ ima koliko i prostih puteva $i \rightarrow v_1 \rightarrow j$. 
			S tim da moramo da pazimo da $v_1 \neq v$. Ove setnje su moguce samo ako su $j$ i $v$ susjedni cvorovi.
			\[
				\sum_{v \in (V \setminus \{i\})} \{a_{vj} * [s_{iv}^{(3)} - a_{ij} * (s_{jv}^{(2)} - a_{ij}a_{jv}) - a_{jv} * (s_{ij}^{(2)} -  a_{iv}a_{vj})]\}
			\]
			Jasno je da $a_{xy}a_{xy}=a_{xy}$, tako da mozemo ukloniti $a_{ij}$ i $a_{jv}$ iz malih zagrada, kao i $a_{jv}$ iz srednje zagrade, cime dobijamo konacan izraz
			 \[
				s_{ij}^{(4)} = \sum_{v \in (V \setminus \{i\})} \{a_{vj} * [s_{iv}^{(3)} - a_{ij} * (s_{jv}^{(2)} - a_{jv}) - (s_{ij}^{(2)} -  a_{iv})]\}
			 \]
			Zbog kompleksnosti samog racuna ovdje cemo se zaustaviti.
			Evidentno je da se povecanjem duzine prostog puta povecava i kompleksnost izraza.
		
			\paragraph{}
			Maksimum broja prostih puteva izmedju dva cvora u grafu mozemo izracunati na osnovu znanja koja imamo o kompletnom grafu sa $n$ cvorova $K_n$.
			Potrebno je da izaberemo $k-1$ cvor koji ce pored $v_0$ i $v_k$ cvorova biti sadrzan u prostom putu. Prvi od $k-1$ cvorova moemo odabrati na $n-2$ nacina, 
			jer ne mozemo odabrati cvorove $v_0$ i $v_k$. Naredni cvor pored pored $v_0$ i $v_k$, ne moze biti ni prethodno izabrani cvor. Istom srategijom biramo cvorove sve dok ne odaberemo svih $k-1$. 
			Ispostavlja se da je u kompletnom grafu broj prostih puteva $i$-$j$ jednak 
			\[
				\begin{split}
				s_{v_0 v_k}^{(k)} & = (n-2)(n-2-1) \dots (n-2-(k-2)) \\
				                  & = (n-2)(n-3) \dots (n-k) \\
				                  & = \prod_{i=0}^{k-2} (n-2-i) \\ 
				                  & = \frac{(n-2)!}{(n-k-1)!}
				\end{split}
			\]
	\section{Prebrojavanje razapinjujucih stabala grafa}
	
		\paragraph{}
		Prisjetimo se rjesavanja strujnih kola upotrebom Kirhofovih zakona. Po drugom Kirhofovom zakonu zbir padova napona u zatvorenoj konturi jednak je nuli. Kako bismo uopste primijenili ovaj zakon, 
		potrebno je odrediti odgovarajuce zatvorene konture. Jedan od nacina da uvijek izaberemo prave konture jeste da posmatramo strujno kola kao graf u kome su provodnici grane, 
		a mjesta racvanja provodnika cvorovi i da formiramo razapinjujuce stablo takvog grafa. Nakon formiranja razapinjujuceg stabla, zatvorene konture konstruisemo tako sto na stablo nadovezemo jos jednu granu,
	        tj. provodnik, koji nije u njegovom sastavu.
	
		\paragraph{}
	        S druge strane, posmatrajmo situaciju u kojoj zelimo da dizajniramo racunarsku mrezu koja ce spajati sve vece gradove u nekoj drzavi. 
		Cilj je da upotrebom brzih, ali i skupih kablova uspostavimo direktnu ili indirektnu vezu izmedju svih tih gradova.
		Izmedju pojedinih gradova nije moguce uspostaviti direktnu mrezu iz razloga sto reljef ne omogucava postavljanje kablova, dok je kod nekih udaljenost prevelika. Jasno je da ce optimalna mreza imati strukturu
		razapinjujuceg stabla grafa kojeg cine gradovi. Takodje, jasno je i da razapinjujuce stablo ne mora biti jedinstveno.
	
		\paragraph{}
		Vidimo da je odredjivanje razapinjujuceg stabla problem koji se jako cesto javlja. Kako razapinjujuce stablo vrlo cesto nije moguce jedinstveno odrediti, 
		potrebno je pronaci sva takva stabla, pa onda odrediti koje nam najvise odgovara vodeci se nekim kriterijumom odabira.  Ukoliko je vec moguce formirati vise od jednog razapinjujuceg stabla, 
		korisno bi bilo imati informaciju o njihovom broju.
		
		\paragraph{}
		Kako bismo konacno opravdali naslov rada koji glasi spektralna teorija grafova, krajnje je vrijeme da uspostavimo vezu izmedju spektra neke od grafu pridruzenih 	     
		matrica i nekog njegovog svojstva, u ovom slucaju broj razapinjujucih stabala. 
		S toga, navodimo narednu teoremu koja nam daje relaciju izmedju sopstvenih vrijednosti Laplasove matrice grafa i broja njegovih razapinjujucih stabala.
	
		\begin{theorem}[Kirhofova matrica-stablo teorema]
		Neka je dat povezan graf $G= (V,E) $ sa n cvorova, njegova Laplasova matrica $L_G$ i njene nenula sopstvene vrijednosti $\alpha_2, \alpha_3, \dots, \alpha_n$. Vazi
		\[
			k(G) = |V|^{-1}\alpha_2 \alpha_3 \dots \alpha_n
		\]
		Kompleksnost povezanog grafa $G$, tj. broj njegovih razapinjujucih stabala jednak je kolicniku proizvoda nenula sopstvenih vrijednosti Laplasove matrice i broja cvorova. 
		Jasno je da se za nepovezan graf ne moze formirati razapinjujuce stablo, tj. u tom slucaju $k(G) = 0$. 
		\end{theorem}
	
		\begin{custom_proof}
	
		Neka je $M$ proizvoljna matrica dimenzije $a \times b$. Sa $M_{R,S}$ oznacavacemo redukovanu matricu koja se od matrice $M$ dobije tako sto se zadrze samo vrste iz skupa $R$ i kolone is skupa $S$. 
		$M_{R,:}$ je matrica koja se dobije od matrice $M$ tako sto se zadrze samo vrste iz skupa $R$, a $M_{:,S}$ je matrica koja se dobije od matrice $M$ tako sto se zadrze samo kolone iz skupa $S$.
		Dobijena matrica je dimenzije $|R| \times |S|$.
	
		Kako svako razapinjujuce stablo sadrzi sve cvorove grafa, tj. njih n
		i $n$ grana, jasno je da treba odrediti koliko podskupova skupa grana sa $n-1$ elemenata zadovoljavaju uslove razapinjujuceg stabla. Kako je u pitanju povezan graf, jasno je i da $|E| \geq (n-1)$
		Matrica incidencije je matrica koja najneposrednije sadrzi informacije o granama grafa, s toga se ona namece kao prirodan izbor prilikom formiranja razapinjujuceg stabla grafa.
	        Iz grafa je potrebno je ukloniti odgovarajuce grane, odnosno iz matrice incidencije potrebno je ukloniti odgovarajuce kolone kako bi se formiralo razapinjujuce stablo. 
		Tako dobijenu matricu incidencije razapinjujuceg stabla mozemo predstaviti sa $I_{:,S}$ gdje je $S \subseteq E$ poskup skupa grana koje ulaze u sastav stabla. Dobijena matrica je dimenzije $n \times (n - 1)$.
	
		Posmatrajmo slucaj u kome matrica $I_{:,S}$ odgovara podgrafu koji nije stablo. U tom slucaju podgraf sadrzi ciklus. Drugim rijecima u okviru skupa $S$ postoji niz grana koje formiraju ciklus.
		Uocimo jedan takav ciklus i formirajmo redukovanu matricu $I_{V \setminus \{q\}:,S}$ uklanjanjem vrste koja predstavlja proizvoljan cvor $q$ van uocenog ciklusa.
	       Jasno je da takav cvor postoji, jer bi u suprotnom znacilo da su svi cvorovi sadrzani u uocenom ciklusu a to nije moguce, jer bi nam u tom slucaju trebalo $n$ grana, dok ih mi imamo dostupno samo $n-1$. 
		Bez umanjenja opstosti, mozemo predpostaviti da se radi o posljednjoj vrsti u matrici, jer je indeksiranje cvorova i grana prilikom formiranja matrice incidencije uradjeno na proizvoljan nacin, 
		sto znaci da je uvijek moguce namjestiti da uoceni cvor $q$ bude predstavljen posljednjom vrstom matrice incidencije. 
	       Formirana matrica je kvadratna dimenzije $n -1$. Krecimo se sad po uocenom ciklusu u grafu i pri tome vrsimo elementarne transformacije kolona matrice $I_{V \setminus \{q\}:,S}$,
	       na nacin da prolaskom granom $(i,j)$ u smjeru od i ka j, mnozimo kolonu koja odgovara grani $\{i,j\}$ sa 1, a u suprotnom sa -1. Dobijenu matricu oznacimo sa $I_{V \setminus \{q\}:,S}'$.
	       Jasno je da vazi $det(I_{V \setminus \{q\}:,S}') = \pm det(I_{V \setminus \{q\}:,S}$). Jasno je da svaki cvor sadrzan u ciklusu je incidentan sa dvije grane,
		i to na nacin da jedna grana izlazi iz datog cvora, a druga ulazi u dati cvor. Ukoliko sada saberemo kolone koje odgovaraju svim granama sadrzanim u uocenom ciklusu, doivecemo da je njihov zbir nula. 
		Drugim rijecima, matrica $I_{V \setminus \{q\}:,S}'$ ima netrivijalnu sumu kolona koja je jednaka nuli, tj. ima linerano zavisne kolone sto na ukazuje da vrijedi
		$det(I_{V \setminus \{q\}:,S}') = det(I_{V \setminus \{q\}:,S}) = 0$.
	
		Posmatrajmo slucaj u kome matrica $I_{:,S}$ odgovara podgrafu koji jeste stablo. Formirajmo redukovanu matricu $I_{V \setminus \{q\}:,S}$ uklanjanjem posljednje vrste matrice $I_{:,S}$ koja odgovara cvoru $q$. 
	        Kako je u pitanju stablo, jasno je da $q$ incidentan sa bar jednom granom. Posto smo uklonili vrstu koja odogovara cvoru $q$ iz matrice, sada imamo slucaj da je bar jedna kolona matrice $I_{V \setminus \{q\}:,S}$
		ima jedan element jednak $\pm 1$, a sve ostale jednake 0. Iskoristimo Laplasov metode racunanja determinante i izvrsimo razvoj po uocenoj koloni, 
		dobijamo $det(I_{V \setminus \{q\}:,S}) = \pm det(I_{V \setminus \{q,w\}:,S})$. Cvor $w$ odgovara vrsti koju smo uklonili prilikom razvoja determinante. Jasno je da su $q$ i $w$ bili susjedni cvorovi. 
		Kako nije moguce da su q i w bili samo medjusobno povezani, mora postojati jos jedan cvor koji je bio susjedan sa jednim od njih, tj. u matrici $I_{V \setminus \{q,2\}:,S}$ mora postojati kolona koja ima jedan element
		jednak $\pm 1$, a sve ostale jednake 0. Ponovo na isti nacin, upotrebom Laplasovog razvoja racunamo $det(I_{V \setminus \{q,w\}:,S})$. 
		Ovaj iterativni postupak ponavljamo sve dok ne dodjemo do determinante koja odgovara matrici dimenzije $2 \time 2$. Nije moguce da dva preostala cvora budu povezana samo oizmedju sebe, tj. bar jedan od njih 
		je u izvornom grafu susjedan sa nekim od uklonjenjih cvorova. Ukoliko preostala sva cvora nisu susjedna, onda i drugi cvor mora biti susjedan sa nekim od uklonjenih cvorova. Na osnovu prethodno, dolazimo do 				zakljucka da je rezultantna matrica oblika $\begin{bmatrix} \pm 1 &  0 \\ \mp 1 & \pm 1 \end{bmatrix}$ ili $\begin{bmatrix} \pm 1 &  \pm 1 \\ \mp 1 & 0 \end{bmatrix}$ ili $\begin{bmatrix} \pm 1 &  \pm 1 \\ 0 &  \mp 1 		\end{bmatrix}$ ili $\begin{bmatrix} 0 & \pm 1 \\ \pm 1 & \mp 1 \end{bmatrix}$ ili $\begin{bmatrix} 0 &  \pm 1 \\ \pm 1 & 0 \end{bmatrix}$ ili $\begin{bmatrix} \pm 1 &  0 \\ 0 & \pm 1 \end{bmatrix}$. 
		Jasno je da dva preostala cvora ne mogu biti istovremeno susjedni i pri tome oba biti povezana sa prethodno uklonjenim covrovima, jer bi u tom slucaju imali ciklus. Determinanta svih navednim matrica je $\pm 1$, 
		sto znaci i da je det $det(I_{V \setminus \{q\}:,S}) = \pm 1$.
	
		Kod oba slucaja orijentacija grana grafa je proizvoljna.
	
		Sumirajmo
		\[
		 det(I_{V \setminus \{q\}:,S} = 
		\begin{cases}
		\pm 1 \text{ ako skup S odgovara skupu grana koje formiraju razapinjujuce stablo grafa G}\\
		0 u suprotnom
		\end{cases} 
		\]
		gdje je $q$ cvor predstavljen sa posljednom vrstom matrice $I_G$.
	
		Kako bismo se rijesili $\pm 1$, kvadrirajmo determinantu:
		\[
		 det(I_{V \setminus \{q\}:,S})^2 = 
		\begin{cases}
		1 \text{ ako skup S odgovara skupu grana koje formiraju razapinjujuce stablo grafa G}\\
		0 \text{ u suprotnom}
		\end{cases} 
		\]
	
		Ukoliko sada izracunamo sumu po svim skupovima grana $S$ koji sadrze tacno $n-1$ granu i formiraju razapinjujuce stablo grafa dobijamo
		\[
		\sum_{S \subseteq E \land |S| = n -1}   det(I_{V \setminus \{q\}:,S})^2 = k(G)
		\]
	
		Ukoliko sada iskoristimo cinjenicu da $det(M^T) = det(M)$, da $M_{R,S} = (M_{:,S})_{R,:} = (M_{R,:})_{:,S}$ i da $(M_{R,S})^T =(M^T)_{S,R}$, dobijamo
		\[
		\begin{split}
		\sum_{S \subseteq E \land |S| = n -1}   det(I_{V \setminus \{q\}:,S})^2  & =
		\sum_{S \subseteq E \land |S| = n -1}   det(I_{V \setminus \{q\}:,S}) * det\big((I_{V \setminus \{q\}:,S})^T\big) \\
		&  = \sum_{S \subseteq E \land |S| = n -1}   det\big((I_{V \setminus \{q\},:})_{:,S}\big) * det\Big(\big((I_{V \setminus \{q\},:})_{:,S}\big)^T\Big)\\
		&  = \sum_{S \subseteq E \land |S| = n -1}   det\big((I_{V \setminus \{q\},:})_{:,S}\big) * det\Big(\big((I_{V \setminus \{q\},:})^T\big)_{S,:}\Big)\\
		\end{split}
		\]
	
		Prijsetimo se Cauchy-Binet formule koja kaze da ako su date matrice $A_{p \times q}$ i $B_{q \times p}$ i ako je $q \geq p$ onda vazi
		\[
			det(AB) = \sum_{|S| = q} det(A_\{:,S\}) det(B_\{S,:\})
		\], gdje se sumiranje vrsi po svim podskupovima S skupa kolona matrice A i skupa vrsta matrice B sa kardinalnosti q.
		
		Prema ovoj formuli, za matricu $A = I_{V \setminus \{q\},:}$ dimenzija $(n - 1) \times |E|$ i $B = (I_{V \setminus \{q\},:})^T\big)$ dimenzija $|E| \times (n - 1)$, pri cemu je sigurno $|E| \geq (n-1)$, jer u pitanju
		povezan grafi, vazi
		\[
		det \big(I_{V \setminus \{q\},:} (I_{V \setminus \{q\},:})^T\big) = \sum_{|S| = n - 1} det\big((I_{V \setminus \{q\},:})_{:.S}\big) det\Big(\big((I_{V \setminus \{q\},:})^T\big)_{S,:}\Big)
		\] 
		Odavde imamo
		\[
		det \big(I_{V \setminus \{q\},:} (I_{V \setminus \{q\},:})^T\big) = k(G)
		\] 
		\end{custom_proof}
	
		Posmatrajmo proizvoljnu matricu 
	
		Ukoliko sada uzmemo u obzir jednu od karakterizacija Laplasove matrice $L_G = I_G I_G^T$ (pri cemu je za svaku granu odabrana proizvoljna orijentacija).
		
		\[
		L_{ij} = \sum_{k=1}^{|E|} I_{ik}I_{k,j}
		\]
		Ukoliko pak posmatramo redukovnu matricu incidencije $I_{V \setminus \{q\},:}$, imamo
		\[
		\big(I_{V \setminus \{q\},:}(I_{V \setminus \{q\},:})^T\big)_{ij} = \sum_{k=1}^{|E|} I_{ik}I_{k,j} \text{ pri cemu } i,j \in \{1,2, \dots n-1\}
		\]
	, tj. kao rezultat dobijamo redukovanu Laplasovu matricu koja je dobijena izostavljanjem posljednje vrste i kolone Laplasove matrice. Oznacimo tu matricu sa $L_{V \setminus \{n\}, E \setminus \{m\}}$, gdje n odgovara posljednoj vrsti a m posljednoj koloni $L_G$.
		Sada imamo
	\[
		det(L_{V \setminus \{n\}, E \setminus \{m\}}) = k(G)
	\]
		Kako n odgovara posljednoj vrsti, a m posljednkoj koloni matrice $L_G$ jasno je da $det(L_{V \setminus \{n\}, E \setminus \{m\}}) = L_{nn}$ gdje je $L_{nn}$ minor koji odgovara elementu $l_{nn}$ matrice $L_G$.
	
	-	Sada je jos potrebno uspostaviti vezu izmedju minora $L_{nn}$ i sopstvenih vrijednosti matrice $L_G$. Kako su sopstvene vrijednosti matrice sadrzane u okviru $det(L_G-xI)$, sasvim je prirodna da podje od matrice 
		$L_G-xI$.
	
		Posmatrajmo sada matricu $L_G - xI_{n \times n}$, gdje $x \in \mathbb{R}$.
		\[
				 (L_G - xI_{n \times n})_{ij} =
				 \begin{cases}
				 deg(i) - x  & \text{ako } i = j  \land i \in \{v_1,v_2\} \\
				 -1 & \text{ako } \{i,j\} = \{v_1,v_2\} \\ 
				 0  & \text{u suprotnom}
				 \end{cases}
		\]
	
		Kako je $det(L_G - xI)$ u stvari karakteriscni polinom matrice $L_G$, imamo $det(L_G - xI) = (x - \alpha_1) \dots (\alpha - \alpha_n)$. Uzmimo u obzir da je $\alpha_1 = 0$ za svaki graf.
		Sada dobijamo $det(L_G - xI) =\bigg(\prod_{i = 1}^{n}\big((-1)^{i + i} (-1)\bigg) x (x - \alpha_2) \dots (x - \alpha_n) = \big(\prod_{i = 1}^{n}(-1)^{2i + 1}\big) x (x - \alpha_2) \dots (x - \alpha_n) 
		= (\prod_{i = 1}^{n}-1) x (x - \alpha_2) \dots (x - \alpha_n) = (-1)^n x (x - \alpha_2) \dots (x - \alpha_n) = (-1)^n(-1)^{n-1} \alpha_2 \dots \alpha_n x + \dots = (-1)^{2n-1} \alpha_2 \dots \alpha_n x + \dots
		= (-1) \alpha_2 \dots \alpha_n x + \dots$.
		 Jasno je da je koeficijent uz $x$ dat sa $- \alpha_2 \dots \alpha_n$. 
	
		S druge strane posmatrajmo matricu $M = L_G - xI_{n \times n})$.Kako je zbir elemenata u svakoj vrsti/koloni matrice $L_G$ jednak nuli, onda je zbir elemenata u svakoj vrsti/koloni matrice M jednak -x.
		Ukoliko sada izvrsimo elementarnu transformaciju vrsta i prvih $n-1$ vrsta saberemo u n-tu vrstu matrice dobijamo  $m_{nj} = x$ .
		Uradimo sada isto i sa kolonama. Nakon sumiranja prvih $n-1$ kolona u posljednju dobijamo $m_{in} = x$ za $i \in \{1,2, \dots n-1\}$ i $m_{nn} = -nx$.
		Kako svi elementi u n-toj vrsti imaju faktor $-x$, vazi $det(M)=-x det(M')$, gdje je M' matrica jednaka M osim sto je $m'_{nj} = 1$ za $j \neq n$, $m'_{nn} = n$ i $m'_{in} = -x$ za $i \neq n$. Ukoliko sada iskoristimo 
		Laplasov razvoj determinante $det(M')$ po posljednjoj koloni, dobijamo $det(L_G - xI) = det(M) = -x det(M') = -x \big( (-1)^{n+n}n M'_{nn} + \sum_{k=1}^{n-1}(-1)^{n+k}(-1)xM'_nk \big) = 
		-x \big( (-1)^{n+n} nM'_{nn} + x\sum_{k=1}^{n-1}(-1)^{n+k+1}M'_nk \big) = (-1)^{2n+1} xnM'_{nn} + x^2 \sum_{k=1}^{n-1}(-1)^{n+k+1}M'_nk $. Sa $M'_{ij}$ predstavlljen je minor koji odgovara elementu $m_{ij}$.
		Uocima sada da za minore vazi $M'_{nn} = (L_G - xI)_{nn}$, jer se matrice razlikuju iskljucivo u posljednjoj n-toj vrsti i koloni. Dobijamo $det(L_G - xI) = (-1)^{2n+1} xn(L_G- xI)_{nn} + x^2 \sum_{k=1}^{n-1}(-1)^{n+k+1}M'_nk $. $(L_G- xI)_{nn}$ je neki novi polinom stepena $n-1$, tj. $(L_G- xI)_{nn} = a_{n-1}x^{n-1} \dots a_0$. Uvrstavanjem dobijamo $det(L_G - xI) = (-1)^{2n+1} xn ( a_{n-1}x^{n-1} \dots a_0) + x^2 \sum_{k=1}^{n-1}(-1)^{n+k+1}M'_nk =  (-1)^{2n+1} n ( a_{n-1}x^n a_{n-2}x^{n-1 }\dots a_0x) + x^2 \sum_{k=1}^{n-1}(-1)^{n+k+1}M'_nk $
	Odavde je jasno da koeficijent uz x jednak $ (-1)^{2n+1} n a_0$, gdje je $a_0$ slobodni clan polinom $(L_G- xI)_{nn}$ koga dobijamo za $x=0$, tj $a_0 = (L_G- 0I)_{nn} = (L_G- O)_{nn} = (L_G)_{nn}$. Konacno
	koeficijent uz x jednak $ (-1)^{2n+1} n (L_G)_{nn} = - n (L_G)_{nn}$.
	
	
		Poredjenjem dva dobijena izraza za $det(L_G - xI)$ dobijamo $- (L_G)_{nn} = - \alpha_2 \dots \alpha_n$, tj. $(L_G)_{nn} = L_{nn} = n^{-1} \alpha_2 \dots \alpha_n$. 
		Ovime je dokaz zavrsen.
		
	\section{Ispitivanje povezanosti grafa}
	\paragraph{}
	Kako bismo došli do informacija o povezanosti grafova $G$, potrebno je ispitati odgovarajuću Laplasovu matricu $L_G$. U nastavku ćemo detaljnije proučiti svojstva odgovrajauće Laplasove matrice, a zatim dati nekoliko teorema koje daju relaciju izmedju odredjenih svojstava Laplasove matrice i povezanosti grafa $G$.
	
	\subsection{Svojstva Laplasove matrice}
	
	\begin{theorem} Laplasova matrica $L_G$ je ermitska matrica.
	\[
	L_G = L_G^*
	\] 
	\end{theorem}
	
	\begin{custom_proof}
	\[
	\forall i,j \in \{1, 2, \dots, n\} \text{ vazi } a_{ij} \in (\mathbb{N}_0 \cup \{-1\}) \Rightarrow L_G \text{ je realna matrica}
	\]
	\[
	 L_G \text{ je realna matrica} \Rightarrow \overline{L_G} = L_G 
	\]
	\[
	\text{Iz definicije Laplasove matrice se vidi } \forall i,j \in \{1, 2, \dots, n\} \text{ vazi } a_{ij} = a_{ji} \Rightarrow L_G^T = L_G 
	\]
	\[
	  L_G^T = L_G \Rightarrow L_G \text{ je simetrična matrica}
	\]
	\[
	\overline{L_G} = L_G \land L_G^T = L_G \Rightarrow L_G^* = \overline{L_G^T} = \overline{L_G} = L_G 
	\]
	\[
	L_G^* = L_G \Rightarrow L_G\text{ je ermitska matrica}
	\]
	\end{custom_proof}
	
	\begin{theorem} Laplasova matrica $L_G$ je pozitivno poluodredjena matrica.
	$G = (V, E)$, gdje je $V = \{1, 2, \dots, n\}$ i njegova Laplasova matrica $L_G$. 
	\[
	  \vec{x}^TL_G\vec{x} \geq 0 \text{ za } \forall \vec{x} \in \mathbb{R}^n
	\] 
	\end{theorem}
	\begin{custom_proof}
	Neka je $\vec{x}$ proizvoljan vektor is vektorskog prostora $\mathbb{R}^n$. Koristicemo definiciju Laplasove matrice u vidu sume matrica $L_{\{v_1,v_2\}}$.
	\[
	\vec{x} = 
	\begin{bmatrix} 
	x_1 \\ x_2 \\ \dots \\ x_{n-1} \\ x_n
	\end{bmatrix}
	\]
	\[
	  \vec{x}^TL_G\vec{x} = \vec{x}^T (\sum_{\{v_1,v_2\} \in E} L_{\{v_1,v_2\}}) \vec{x}  =  \sum_{\{v_1,v_2\} \in E} \vec{x}^TL_{\{v_1,v_2\}} \vec{x}
	\]
	\[
	  \vec{x}^TL_G\vec{x} =\vec{x}^T (\sum_{\{v_1,v_2\} \in E} L_{\{v_1,v_2\}}) \vec{x} 
	\]  
	
	Posmatrajmo proizvod $\vec{x}^T L_{\{v_1,v_2\}}$. Rezultantna matrica je matrica vrsta dimenzije $1 \times n$.
	\[
	M = \vec{x}^T L_{\{v_1,v_2\}} = [m_{1j}]_{1 \times n}
	\]
	\[
	m_{1j} = \vec{x}^T L_{\{v_1,v_2\} . j}
	\]
	Sve kolone matrice $L_{\{v_1,v_2\}}$ su nula kolone osim kolona sa indeksom $v_1$ i $v_2$.
	\[
	 m_{1j} =
	 \begin{cases}
	 \vec{x} L_{\{v_1,v_2\} . v_1}  & \text{ako } j = v_1 \\ 
	 \vec{x} L_{\{v_1,v_2\} . v_1}  & \text{ako } j = v_2  \\ 
	 0  & \text{u suprotnom}
	 \end{cases}
	\]
	Kolona matrice $L_{\{v_1,v_2\}}$ indeksom $v_1$ ima vrijednost 1 na poziciji $v_1$, tj. $a_{v_1 v_1} = 1$, a vrijednost -1 na poziciji $v_2$, tj. $a_{v_2 v_1} = -1$ .
	Kolona matrice $L_{\{v_1,v_2\}}$ indeksom $v_2$ ima vrijednost 1 na poziciji $v_2$, tj. $a_{v_2 v_2} = 1$, a vrijednost -1 na poziciji $v_1$, tj. $a_{v_1 v_1} = -1$ .
	Svi ostali elementi matrice su 0.
	\[
	 m_{1j} =
	 \begin{cases}
	 (\vec{x}_{v_1} * a_{v_1 v_1}) + (\vec{x}_{v_2} * a_{v_2 v_1}) = \vec{x}_{v_1} - \vec{x}_{v_2}   & \text{ako } j = v_1 \\ 
	 (\vec{x}_{v_2} * a_{v_2 v_2}) + (\vec{x}_{v_2} * a_{v_1 v_2}) = \vec{x}_{v_2} - \vec{x}_{v_1}  & \text{ako } j = v_2  \\ 
	 0  & \text{u suprotnom}
	 \end{cases}
	\]
	\[
	M = 
	\begin{bmatrix} 
	0 & \dots &  \vec{x}_{v_1} - \vec{x}_{v_2} & \dots &  \vec{x}_{v_2} - \vec{x}_{v_1} & \dots
	\end{bmatrix}
	_{1 \times n}
	\]
	\[
	  \vec{x}^TL_G\vec{x} = M \vec{x} =\begin{bmatrix} 
	0 & \dots &  \vec{x}_{v_1} - \vec{x}_{v_2} & \dots &  \vec{x}_{v_2} - \vec{x}_{v_1} & \dots
	\end{bmatrix}
	_{1 \times n} \vec{x}  
	\]
	
	Posmatrajmo proizvod $M \vec{x}$. Rezultantna matrica je matrica vrsta dimenzije $1 \times 1$, tj. skalarna vrijednost. Matrica vrsta $M$ ima nenula vrijednost samo u kolonama $v_1$ i $v_2$.
	\[
	M \vec{x} = m_{1 v_1} * \vec{x}_{v_1} + m_{1 v_2} * \vec{x}_{v_2} =  (\vec{x}_{v_1} - \vec{x}_{v_2}) * \vec{x}_{v_1} + (\vec{x}_{v_2} - \vec{x}_{v_1}) * \vec{x}_{v_2} 
	\] 
	\[
	M \vec{x} =  \vec{x}_{v_1}^2 - \vec{x}_{v_1} * \vec{x}_{v_2} + \vec{x}_{v_2}^2 - \vec{x}_{v_1} * \vec{x}_{v_2} = \vec{x}_{v_1}^2 - 2 *(\vec{x}_{v_1} * \vec{x}_{v_2}) \vec{x}_{v_2}^2 = (\vec{x}_{v_1} - \vec{x}_{v_2})^2   
	\] 
	Kada dobijeni rezultat uvrstimo u pocetni izraz dobijamo:
	\[
	  \vec{x}^TL_G\vec{x} = \sum_{\{v_1,v_2\} \in E} (\vec{x}_{v_1} - \vec{x}_{v_2})^2
	\]
	\[
	  (\vec{x}_{v_1} - \vec{x}_{v_2})^2 \geq 0 \text{ za } \forall v_i \in \mathbb{R} \Rightarrow \vec{x}^TL_G\vec{x} \geq 0 \text{ za } \forall \vec{x} \in \mathbb{R}^n
	\]
	Pokazali smo da je proizvod $\vec{x}^TL_G\vec{x}$ nenegativan, cime smo dokazali da je Laplasova matrica $L_G$ pozitivno poludodredjena.
	\end{custom_proof}
	
	\begin{theorem} Sve sopstvene vrijednosti Laplasove matrice $L_G$ su realni brojevi.
	Neka je data Laplasova matrica $L_G \in  \mathbb{M}_{n \times n}(\mathbb{R})$.
	\[
	 L_G\vec{x} = \alpha \vec{x} \text{ gdje je } \vec{x} \in ( \mathbb{R}^n \setminus \{\vec{0}\}) \text{ nenula vektor} \Rightarrow \alpha \in \mathbb{R}
	\]
	\end{theorem}
	
	\begin{custom_proof}
	Iako je data teorema direktna posljedica teoreme koja govori da su sve sopstvene vrijednosti ermitske matrice realni brojevi, u nastavku ce biti izlozen dokaz ovoga tvrdjenja.
	
	Pretpostavimo da je $\alpha \in \mathbb{C}$ proizvoljna sopstvena vrijednost matrice $l_G$ i da je $\vec{x} \in (\mathbb{R}^n \setminus \{\vec{0}\})$ odgovarajuci sopstveni vektor. Tada vazi:
	\[
		L_G\vec{x} = \alpha \vec{x}
	\]
	Pomnozimo izraz s lijeva vektorom $\vec{x}^*$.
	\[
		\vec{x}^*L_G\vec{x} = \vec{x}^*\alpha \vec{x}
	\]
	\[
		\vec{x}^*L_G\vec{x} = (\vec{x}^* \vec{x}) \alpha
	\]
	\[
		\vec{x}^*L_G\vec{x} = \sum_{i=1}^{n}(Re(x_i)^2 + Im(x_i)^2 ) \alpha =  \sum_{i=1}^{n}|x_i|^2 \alpha= |\vec{x}|^2 \alpha  
	\]
	\begin{equation}
		 \alpha =   \frac{\vec{x}^*L_G\vec{x}}{|\vec{x}|^2}
	\label{eq:alpha}
	\end{equation}
	Izraz $\vec{x}^* \vec{x}$ prestavlja kvadrat modula vektora $\vec{x}$, a modul vektora je uvijek nenegativan realan broj,tj. vazi
	\[
		\vec{x}^* \vec{x} = |\vec{x}|^2 \land |\vec{x}| \in (\mathbb{R}^+ \cup \{0\})  
	\]
	U nasem slucaju $\vec{x}$ je nenula vektor pa vazi jos stroziji uslov $|\vec{x}| \ge 0$, koji nam je omogucio da se modul vektora $\vec{x}$ nadje u imeniocu razlomka.  \\
	Posmatrajamo sada lijevu stranu jednakosti, tj. proizvod matrica $\vec{x}^*L_G\vec{x}$ i pokusajmo izracunati adjugovanu vrijednost rezultantne matrice.
	\[
		\overline{\vec{x}^*L_G\vec{x}} = ?
	\] 
	Kako je desna strana jednakosti $|\vec{x}| * \alpha$ proizvod realnog i kompleksnog broja, odatle se lako zakljucuje da je rezultat proizvoda na desnoj strani jednakosti kompleksan broj tj. matrica iz vektorskog prostora  $\mathbb{M}_{1 \times 1}(\mathbb{C})$. Transponovanje matrice dimenzija $1 \times 1$ nema nikakav uticaj na matricu, sto znaci
	\[
	(\vec{x}^*L_G\vec{x})^T = \vec{x}^*L_G\vec{x}
	\]
	Ukoliko sada to uvrstimo i uzmemo u obzir da je $M^*=\overline{M^T} =\overline{M}^T$, gdje je $M$ proizvolja matrica, dobijamo
	\[
		\overline{\vec{x}^*L_G\vec{x}} = \overline{(\vec{x}^*L_G\vec{x})^T} = (\vec{x}^*L_G\vec{x})^* = \vec{x}^*L_G^*(\vec{x}^*)^*= \vec{x}^*L_G^*\vec{x}
	\] 
	Ukoliko sada iskoristimo tvrdjenje da je Laplasova matrica $L_G$ ermitska, tj. da vazi $L_G^*=L_G$. dobijamo
	\[
		\overline{\vec{x}^*L_G\vec{x}} = \vec{x}^*L_G\vec{x}
	\] 
	
	Iskoristimo sada tvrdjenje vezano za kompleksne brojeve. Neka je $x \in \mathbb{C}$.
	\[
	 |x| = x \Rightarrow x \in \mathbb{R}
	\]
	Odatle dobijamo da  $\vec{x}^*L_G\vec{x} \in \mathbb{R}$ \\
	Kako su i imenilac i brojilac na desnoj strani jednakosti \ref{eq:alpha} realni brojevi i lijeva strana jednakosti mora biti realan broj, tj. $\alpha \in \mathbb{R}$. Ovime je dokaz zavrsen.
	\end{custom_proof}
	
	\begin{theorem} Sopstvene vrijednosti Laplasove matrice $L_G$ su nenegativni realni brojevi.
	Neka je data Laplasova matrica $L_G \in  \mathbb{M}_{n \times n}(\mathbb{R})$.
	\[
	 L_G\vec{x} = \alpha \vec{x} \text{ gdje je } \vec{x} \in ( \mathbb{R}^n \setminus \{\vec{0}\}) \text{ nenula vektor} \Rightarrow \alpha \in (\mathbb{R}^+ \cup \{0\})
	\]
	\end{theorem}
	
	\begin{custom_proof}
	Iako je data teorema direktna posljedica teoreme koja govori da su sopstvene vrijednosti matrice nenegativni realni brojevi ako i samo ako je matrica pozitivno poluodredjena, u nastavku ce biti izlozen dokaz ovoga tvrdjenja ali samou jednom smjeru. \\
	
	Neka je $\alpha$ proizvoljna sopstvena vrijednost matrice $L_G$ i da je $\vec{x} \in (\mathbb{R}^n \setminus \{\vec{0}\})$ odgovarajuci sopstveni vektor. Tada vazi
	
	\[
		L_G\vec{x} = \alpha \vec{x}
	\]
	Pomnozimo izraz s lijeva vektorom $\vec{x}^T$.
	\[
		\vec{x}^TL_G\vec{x} = \vec{x}^T\alpha \vec{x} = \vec{x}^T \vec{x} \alpha = \vec{x}^2 \alpha 
	\]
	Kako je $\vec{x}$ je nenula vektor, mozemo pisati
	\[
	 \alpha = \frac{\vec{x}^TL_G\vec{x}}{|\vec{x}|^2}
	\]
	Ukoliko sada iskoristimo cinjenicu da je $|\vec{y}| > 0$ i da je Laplasova matrica pozitivno poluodredjena, tj. da vazi $\vec{y}^TL_G\vec{y}$, gdje je $\vec{y} \in \mathbb{R}^n$, dolazimo dolazimo do zakljucka da lijeva strana jednakosti mora biti nenegativna jer je takva desna strana, tj. da vazi
	\[
		L_G\vec{x} = \alpha \vec{x} \text{ gdje je } \vec{x} \in (\mathbb{R}^n \setminus \{\vec{0}\}) \Rightarrow \alpha \geq 0
	\] 
	\end{custom_proof}
	
	\begin{consequence}
	S obzirom da su sve sopstvene vrijednosti matrice $L_G$ dimenzije $n \times n$ neneagativni realni brojevi i da sopstvenih vrijednosti ima maksimalno $n$ (jer je minimum dimenzija Laplasove matrica upravo $n$), mozemo uspostaviti sljedecu relaciju izmedju sopstvenih vrijednosti matrice $L_G$ 
	\[
	0 \leq \alpha_1 \leq \alpha_2 \leq \dots \leq \alpha_{n-1} \leq \alpha_n
	\]
	\end{consequence}
	
	\begin{theorem} Najmanja sopstvena vrijednost Laplasove matrice $L_G$ je $\alpha_1=0$
	\[
	L_G \text{ je Laplasova matrica proizvoljnog grafa } G \Rightarrow \alpha_1=0
	\]
	\end{theorem}
	
	\begin{custom_proof}
	Posmatrajmo proizvod Laplasove matrice $L_G$ i vektora $\vec{x} \in \mathbb{R}^n$.
	\[
	\vec{x} = 
	\begin{bmatrix} 
	1 \\ 1 \\ \dots \\ 1 \\ 1
	\end{bmatrix}
	_{n \times 1}
	\]
	\[
	L_G\vec{x} = [a_{ij}]_{n \times 1}
	\]
	\[
		a_{ij} = \sum_{i=1}^{n}L_{i.}\vec{x} = \sum_{j=1}^{n}\sum_{i=1}^{n}L_{ij} \vec{x}_j 
	\]
	Ukoliko sada iskoristimo cinjenicu da su sve kordinate vektora $\vec{x}$ jednake 1, dobijamo
	\[
		a_{ij} = \sum_{i=1}^{n}L_{i.}\vec{x} = \sum_{j=1}^{n}\sum_{i=1}^{n}L_{ij} * 1 = \sum_{j=1}^{n}\sum_{i=1}^{n}L_{ij}
	\]
	Ukoliko sada iskoristimo cinjenicu da je zbir elemenata svake vrste/kolone Laplasove matrice jednak 0, dobijamo
	\[
		a_{ij} = \sum_{j=1}^{n}\sum_{i=1}^{n}L_{ij} = \sum_{j=1}^{n} 0 = 0
	\]
	Sada se vidi da je svaka koordinata rezultujuce vrsta matrica jednaka 0, tj. je je rezultat mnozena Laplasove matrice i vektora $\vec{x}$ nula vektor.
	\[
	L_G\vec{x} = \vec{0} = 0 * \vec{x}
	\]
	S obzirom da je proizvod bilo kog vektora i 0 jednak nula vektoru $0 * \vec{y} = \vec{0}$, imamo
	\[
	L_G\vec{x} = 0 * \vec{x}
	\]
	Kako vektor $\vec{x}$ nije nula vektor zakljucujemo da je $\alpha = 0$ sopstvena vrijednost proizvoljne Laplasove matrice. Uzimajuci u obzir da smo vec uspostavili relaciju izmedju sopstvenih vrijednosti Laplasove matrice, imamo
	\[
		0 \leq \alpha_1 \leq \alpha_2 \leq \dots \leq \alpha_{n-1} \leq \alpha_n \land 0 in \{\alpha_1, \alpha_2, \dots, \alpha_n\} \Rightarrow \alpha_1 = 0
	\]
	Ovime tvrdnja je dokazana.
	\end{custom_proof}
	
	\begin{theorem} Za sopstvenu vrijednost $\alpha_2$ Laplasove matrice $L_G$ koja odgovara povezanom grafu $G$ vazi $\alpha_2>0$
	\paragraph{}
	Navedena tvrdja je ekvivalentna tvrdnji da je algebarska visetrukost 0 kao sopstvene vrijednosti Laplasove matrice $L_G$ proizvoljnog poveyanog grafa $G$ jednaka 1.
	\end{theorem}
	
	\begin{custom_proof}
	Neka je dat graf $G=(V,E).$ Neka je $\vec{z} \in \mathbb{R}^n$ sopstveni vektor koji odgovara sopstvenoj vrijednosti $\alpha = 0$ Laplasove matrice $L_G$.
	Pokusajmo odrediti neka svojstva vekora $\vec{z}$
	\[
		L_G\vec{z} = 0 * \vec{z} = \vec{0}
	\]
	Pomnozimo izraz s lijeva vektorom $\vec{z}^T$.
	\[
		\vec{z}^TL_G\vec{z} = \vec{x}^T * \vec{0} = 0
	\]
	Iz dokaza prethodnih teorema imamo 
	\[
	\vec{z}^TL_G\vec{z} = \sum_{\{v_1,v_2\} \in E} (z_{v_1} - z_{v_2})
	\]
	
	Iz datog izraza se vidi da su koordinate sopstvenog vektora $\vec{z}$, koje odgovaraju susjednim cvorovima grafa $G$, jednake. S obzirom da je graf $G$ povezan, tj. da po definiciji povezanog grafa postoji put izmedju bilo koja dva cvora u grafu, dolazimo do zakljucka da su sve koordinate sopstvenog vektora $\vec{z}$, koji odgovara sopstvenoj vrijednosti 0, jednake. 
	
	\[
	\begin{split}
	 \vec{z} \text{  je sopstveni vektor koji odgovara sopstvenoj vrijednosti } & \alpha = 0 \text{ matrice}  L_G \Rightarrow \\
	 \vec{z} =p *
	\begin{bmatrix} 
	1 \\ 1 \\ \vdots \\ 1 \\ 1
	\end{bmatrix}_{n \times 1}
	\text{ gdje je } p \in (\mathbb{C} \setminus \{\vec{0}\})
	\end{split}
	\]
	Odredimo sada sopstveni prostor Laplasove matrice $L_G$ koji odgovara sopstvenoj vrijednosti $\alpha = 0$
	Kako smo odredili opsti izraz sopstvenog vektora Laplasove matrice koji odgovara sopstvenoj vrijednosti $\alpha = 0$, mi smo pronasli sve sopstvene vektore koji odgovaraju sopstvenoj vrijednosti $\alpha = 0$.
	
	\[
	S_{\alpha} = \{p *
	\begin{bmatrix} 
	1 \\ 1 \\ \vdots \\ 1 \\ 1
	\end{bmatrix}_{n \times 1} : p \in (\mathbb{C} \setminus \{0\}) \} 
	\]
	Ukoliko dozvolimo da $p=0$, tj. odradimo uniju skupa sopstvenih vektora $S_{\alpha}$ i skupa koji sadrzi nula vektor, dobicemo sopstveni vektorski prostor koji odgovara $\alpha = 0$.
	\[
	U_{\alpha} = S_{\alpha} \cup \{\vec{0}\}= \{
	\begin{bmatrix} 
	1 \\ 1 \\ \vdots \\ 1 \\ 1
	\end{bmatrix}_{n \times 1}  : p \in (\mathbb{C} \setminus \{0\}) \} \cup \{\vec{0}\}= \{
	\begin{bmatrix} 
	1 \\ 1 \\ \vdots \\ 1 \\ 1
	\end{bmatrix}_{n \times 1} :  p \in \mathbb{C}\} 
	\]
	Dobijeni izraz predstavlja nista drugo do linearnu kombinaciju jednog vektora, tj. linearni vektorski omotac nad jednim vektorom.
	\[
	U_{\alpha} = S_{\alpha} \cup \{\vec{0}\}= \{
	\begin{bmatrix} 
	1 \\ 1 \\ \vdots \\ 1 \\ 1
	\end{bmatrix}_{n \times 1} :  p \in \mathbb{C}\} = L (\begin{bmatrix} 
	1 \\ 1 \\ \vdots \\ 1 \\ 1
	\end{bmatrix}_{n \times 1})
	\]
	Kako je sopstveni prostor jednak linearnom vektorskom omotacu nad jednim vektorom, ocito da se baze sopstvenog vektorskog prostora koji odgovara $\alpha = 0$ se sastoji od jednog vektora, sto nadalje znaci da je geometrijska visestrukost $\alpha = 0$ jednaka 1.
	Ukoliko sada iskoristimo cinjenicu da je geometrijska visestrukost manja ili jednaka algebarskoj visestrukosti sopstvene vrijednosti, dolazimo do zakljucka da sopstvene vrijednost $\alpha = 0 $ ima algebarsku visestrukost 1. Iz relacije koju smo uspostavili izmedju sopstvenih vrijednosti imamo $\alpha_1 \leq \alpha_2$.
	\[
	\alpha_1 = 0 \text{ ima algebarsku visestrukost 1 } \Rightarrow \alpha_1 \neq \alpha_2
	\]
	\[
	\alpha_1 = 0  \land \alpha_1 \neq \alpha_2 \land \alpha_1 \leq \alpha_2 \Rightarrow \alpha_2 \ge 0
	\]
	Ovime dokaz je zavrsen.
	\end{custom_proof}
	
	\begin{theorem} 
	Geometrijska visestrukost sopstvene vrijednosti $\alpha_1 = 0$ Laplasove matrice $L_G$ predstavlja broj povezanih komponenti u grafu $G$
	\end{theorem}
	
	\begin{custom_proof}
	Pretpostavimo da graf  $G = (V, E)$, gdje je $V = \{1, 2, \dots, n\}$, ima $k$ povezanih komponenti. Neka su $G_i = \{V_i,E_i\} i \in \{1,2, \dots n\}$ te povezane komponente grafa $G$.
	
	Za svaku od povezanih komponenti konstruisimo vektor $\vec{w_i}$ takav da njegove koordinate ciji je indeks se ne nalazi u skupu cvorova odgovarajuce povezane komponente ima vrijednost 1, u suprotnom 0.
	\[
	 w_{i_j} =
	 \begin{cases}
	 1  & \text{ako } j \in V_i \\ 
	 0  & \text{ako } j \notin V_i
	 \end{cases}
	\]
	
	Posmatrajmo sada proizvod Laplasove matrice $L_G$ i vektora $w_i$.
	\[
	L_G\vec{w_i} = \vec{r}
	\]
	\[
	\vec{r}_j = L_{j.} \vec{w_i} = \sum_{ k  \in V} L_{jk}\vec{w_{i_k}} 
	\]
	Kako su koordinate vektora $w_i$ ciji se indeks ne nalazi u skupu cvorova povezane komponente jdnake 0, mozemo sumirati po cvorovima poveyane komponente $G_i$. 
	\[
	\vec{r}_j = \sum_{ k  \in V_i} L_{jk}\vec{w_{i_k}} = \sum_{ k  \in V_i} L_{jk} * 1 = \sum_{ k  \in V_i} L_{jk}
	\]
	Ponovo, iskoristimo cinjenicu da je suma elemenata po kolonama/vrstama Laplasove matrice jednaka 0.
	\[
	\vec{r}_j = \sum_{ k  \in V_i} L_{jk}\vec{w_{i_k}} = \sum_{ k  \in V_i} L_{jk} * 1 = \sum_{ k  \in V_i} L_{jk}
	\]
	\[
	\vec{r} = \vec{0}
	\]
	
	Posto je rezultantni vektor nula vektor, to nam ukazuje da su vektori $\vec{w_i}$ sopstveni vektori koji odgovaraju sopstvenoj vrijednosti  $\alpha_1 = 0$ Laplasove matrice.
	Kako su skupovi cvorova pojedinih povezanih komponenti grafa disjunktni skupovi, jasno je i da su vektoru $w_i i \in \{1,2 \dots k\}$. Kako smo odredili opstu formulu za sopstvene vektore koji  odgovaraju sopstvenoj vrijednosti  $\alpha_1 = 0$ Laplasove matrice grafa $G$ sa $k$ povezanih komponenti, mi smo odredili sve vektore u skupu $S_{\alpha_1}$. Odatle, imamo da je sopstveni vektorski prostor koji  odgovara $\alpha_1 = 0$ jednak linearnom vektorskom omotacu vektora $w_i$ za $i \in \{1,2 \dots k\}$
	\[
	U_{\alpha_1} = S_{\alpha_1} \cup \{\vec{0}\}= L (\{ w_i : i \in \{1,2 \dots k\}\})
	\]
	
	Odavde mozemo zakljuciti da je geometerijska visestrukost sopstvene vrijednosti $\alpha_1 = 0$ jednaka $k$, sto odgovara broju povezanih komponenti grafa $G$, cime je tvrdnja dokazana.
	\end{custom_proof}
	
	\begin{example}{Odrediti spektar Laplasove matrice kompletnog grafa $K_n$}
	U kompletnon grafu sa $n$ cvorova $K_n =(V,E)$ svaki cvor je susjedan sa svim ostalim cvorovima, tj. vazi $deg(v)=n-1$ za $\forall v \in V$. To znaci da su svi elementi na glavnoj dijagonali Laplasove matrice jednaki $n-1$, a ostali elementi jedanki $-1$.
	\[
	L_G =
	\begin{bmatrix}
	n-1 & -1 & -1 & \dots & -1 \\
	-1 & n-1 & -1 & \dots & -1 \\
	-1 & -1 & n-1 & \dots & -1 \\
	\vdots & \vdots & \vdots & \ddots & \vdots \\
	-1 & -1 & -1 & \dots & n-1 \\
	\end{bmatrix}_{n \times n}
	\]
	
	Posmatracemo matricu $L_{K_n} - nI$. Motivacija za posmatranje ove matrice je ta sto je ovo matrica kod koje su svi elementi jednaki $-1$, tj. $L_{K_n} - nI = [-1]_{n \times n}$, sto bi nam trebalo omoguciti da lako pronadjemo njene sopstvene vrijednosti.
	
	Kako su sve vrste/kolone matrice $L_{K_n} - nI$ jednake, jasno je da $rank(L_{K_n} - nI) = 1$ i $det(L_{K_n} - nI) = 0$. Ako uzmemo u obzir da 
	\[
	det(M) = 0 \Leftrightarrow \alpha = 0 \text{ je sopstvena vrijednost matrice} M \text{ za } \forall M \in \mathbb{M}_{n \times n}
	\]
	dolazimo do zakljucka $\alpha = 0$ je sopstvena vrijednost matrice $L_{K_n} - nI$. 
	Sada je potrebno odrediti geometrijsku visestrukost sopstvene vrijednosti $\alpha = 0$. Geometrijska visestrukost sopstvene vrijednosti matrice $M$  je jednaka dimenziji sopstvenog vektorskog prostora koji joj odgovara.
	Da bi $\alpha'$ bila sopstvena vrijednost matrice $L_{K_n} - nI$, mora da vazi
	\[
	\begin{split}
	(L_{K_n} - nI)\vec{x} &= \alpha' \vec{x} \\
	(L_{K_n} - nI)\vec{x} &= \alpha' I \vec{x} \\
	(L_{K_n} - nI)\vec{x} - \alpha' I \vec{x} &= \vec{0}  \\
	[(L_{K_n} - nI) - \alpha' I] \vec{x} &= \vec{0}  \\
	[L_{K_n} - (n + \alpha') I] \vec{x} &= \vec{0}  \\
	\end{split}
	\]
	Dimenzijia sopstvenog vektorskog prostora koji joj odgovara $\alpha = 0$ jednaka je dimenziji nula prostora matrice $L_{K_n} - (n + \alpha') I$.
	Ako sada uvrstitmo $\alpha = 0$, dobijamo da je geometrijska visestrukost $\alpha = 0$ je $dim N(L_{K_n} - (n + 0) I) = dim N(L_{K_n} - n I)$. Ako iskoristimo sada rank nullity teoremo koja kaze: $rank(M) + N(m) = n$ za $\forall M \in M_{n \times n}$, dobijamo da je  geometrijska visestrukost $\alpha = 0$ jednala $n - rank(L_{K_n} - n I) = n - 1$.
	
	Posmatrajmo sada sopstvene vrijednosti matrice $L_G$. Za njih mora da vazi \[
	\begin{split}
	L_{K_n} \vec{x} &= \alpha \vec{x} \\
	L_{K_n} \vec{x} &= \alpha I \vec{x} \\
	L_{K_n} \vec{x} - \alpha I \vec{x} &= \vec{0}  \\
	(L_{K_n} - \alpha I) \vec{x} &= \vec{0}  \\
	\end{split}
	\]
	
	Poredeci izraze $[L_{K_n} - (n + \alpha') I] \vec{x} = \vec{0}$ i $(L_{K_n} - \alpha I) \vec{x} = \vec{0}$, zakljucujemo da ako je $\alpha$ sopstvena vrijednost matrice $L_G$ onda je $n + \alpha'$ sopstvena vrijednost te matrice, tj $\alpha = n - \alpha'$. Odatle dobijamo sa ke $\alpha = n $ sopstvena vrijednost matrice $L_G$ da algebarskom vissetrukoscu $n-1$.
	  
	
	Drugi nacin da potvrdimo da je 0 sopstvena vrijednost matrice $L_{K_n} - n I$ jeste da, uocim sljedeci vektor
	\[
	\vec{x} =
	\begin{bmatrix}
	 n-1 \\ 
	1 \\
	1 \\
	\vdots \\
	1
	\end{bmatrix}_{n \times 1}
	\]
	dobijamo
	\[
	\begin{split}
	(L_{K_n} - nI) \vec{x} &= 
	\begin{bmatrix}
	-1 & -1 & -1 & \dots & -1 \\
	-1 & -1 & -1 & \dots & -1 \\
	-1 & -1 & -1 & \dots & -1 \\
	\vdots & \vdots & \vdots & \ddots & \vdots \\
	-1 & -1 & -1 & \dots & -1
	\end{bmatrix}
	\begin{bmatrix}
	n-1 \\
	1 \\
	1 \\
	\vdots \\
	1
	\end{bmatrix}
	 = 
	\begin{bmatrix}
	(n-1) + \sum_{i=1}^{n-1} (-1) \\
	(n-1) + \sum_{i=1}^{n-1} (-1) \\
	(n-1) + \sum_{i=1}^{n-1} (-1) \\
	\vdots \\
	(n-1) + \sum_{i=1}^{n-1} (-1)
	\end{bmatrix}  \\
	& =
	\begin{bmatrix}
	(n-1) + (n-1) * (-1) \\
	(n-1) + (n-1) * (-1) \\
	(n-1) + (n-1) * (-1) \\
	\vdots \\
	(n-1) + (n-1) * (-1) \\
	\end{bmatrix}
	=
	\begin{bmatrix}
	(n-1) - (n-1) \\
	(n-1) - (n-1) \\
	(n-1) - (n-1) \\
	\vdots \\
	(n-1) - (n-1) \\
	\end{bmatrix}
	 =
	\begin{bmatrix}
	0 \\
	0 \\
	0 \\
	\vdots \\
	0 \\
	\end{bmatrix}
	= 0 *
	\begin{bmatrix}
	n-1 \\
	1 \\
	1 \\
	\vdots \\
	1 \\
	\end{bmatrix}
	\end{split}
	\]
	
	\[
	\begin{split}
	(L_{K_n} - nI) 
	\begin{bmatrix}
	n-1 \\
	1 \\
	1 \\
	\vdots \\
	1
	\end{bmatrix}_{n \times 1}
	= 0 *
	\begin{bmatrix}
	n-1 \\
	1 \\
	1 \\
	\vdots \\
	1 \\
	\end{bmatrix}_{n \times 1} 
	\end{split}
	\]
	Odavdje dobijamo da je 0 sopstvena vrijednost matrice $L_{K_n} - nI$. 
	\end{example}
	
	\begin{example}
		Odrediti spektar Laplasove matrice grafa ciklusa $C_n$
	Ono sto odmah mozemo reci za graf $C_n$ jeste da je ovo regulan graf sa stepenom regularnosti 2. 
	Laplasova matrica ciklusa je data sa
	\[
	\begin{bmatrix}
	
	2 & 1 & 0 & 0 & \dots & \dots & \dots & 1 \\
	1 & 2 & 1 & 0 & \dots & \dots & \dots & 0 \\
	0 & 1 & 2 & 1 & \dots & \dots & \dots & 0 \\
	0 & 0 & 1 & 2 & \dots & \dots & \dots & 0 \\
	0 & 0 & 0 & 1 & \dots & \dots & \dots & 0 \\
	\vdots  & \vdots  & \vdots  & \vdots  & \ddots & \vdots & \vdots & \vdots  \\
	0 & 0 & 0 & 0 & \dots & 2 & 1 & 0 \\
	0 & 0 & 0 & 0 & \dots & 1 & 2 & 1 \\
	1 & 0 & 0 & 0 & \dots & 0 & 1 & 2 \\
	\end{bmatrix}
	\]
	
	
	Lako se vidi da vrste ove matrice predstavljaju ciklicne permutacije. Upravo je ovakva priroda Laplasove matrice povod da prilikom odredjivanja njeno spektra posmatramo sljedeci vektor 
	\[ \vec{x} =
	\begin{bmatrix}
	1 \\
	w \\
	w^2\\
	\vdots \\
	w^{n-1}
	\end{bmatrix}
	\]
	gdje je $w^n = 1$. Izracunajmo proizvod $L_c \vec{x}$
	\[
	\begin{split}
	  	L_C \vec{x} & = 
		\begin{bmatrix}
		2 & 1 & 0 & 0 & \dots & \dots & \dots & 1 \\
		1 & 2 & 1 & 0 & \dots & \dots & \dots & 0 \\
		0 & 1 & 2 & 1 & \dots & \dots & \dots & 0 \\
		0 & 0 & 1 & 2 & \dots & \dots & \dots & 0 \\
		0 & 0 & 0 & 1 & \dots & \dots & \dots & 0 \\
		\vdots  & \vdots  & \vdots  & \vdots  & \ddots & \vdots & \vdots & \vdots  \\
		0 & 0 & 0 & 0 & \dots & 2 & 1 & 0 \\
		0 & 0 & 0 & 0 & \dots & 1 & 2 & 1 \\
		1 & 0 & 0 & 0 & \dots & 0 & 1 & 2 \\
		\end{bmatrix}
		 \begin{bmatrix}
		1 \\
		w \\
		w^2 \\
		w^3 \\
		w^4 \\
		\vdots \\
		w^{n-3} \\
		w^{n-2} \\
		w^{n-1} \\
		\end{bmatrix} 
		=
		\begin{bmatrix}
		 2 + w + w^{n-1} \\
		1 + 2w + w^2 \\
		w + 2w^2 + w^3 \\
		w^2 + 2w^3 + w^4 \\
		w^3 + 2w^4 + w^5 \\
		\vdots \\
		w^{n-4} + 2w^{n-3} + w^{n-2} \\
		w^{n-3} + 2w^{n-2} + w^{n-1} \\
		1 + w^{n-2} + 2w^{n-1} \\
		\end{bmatrix} = \\
	 	 & =  
		\begin{bmatrix}
		 w^{n-1} + 2 * 1 + w \\
		1 + 2w + w^2       \\
		w + 2w^2 + w^3       \\
		w^2 + 2w^3 + w^4  \\
		w^3 + 2w^4 + w^5  \\
		\vdots \\
		w^{n-4} + 2w^{n-3} + w^{n-2} \\
		w^{n-3} + 2w^{n-2} + w^{n-1} \\
		1 * 1 + 2w^{n-1} + w^{n-2} \\
		\end{bmatrix}
		= 
		\begin{bmatrix}
		w^{n-1} + 2 w^n + w \\
		1 + 2w + w^2       \\
		w + 2w^2 + w^3       \\
		w^2 + 2w^3 + w^4  \\
		w^3 + 2w^4 + w^5  \\
		\vdots \\
		w^{n-4} + 2w^{n-3} + w^{n-2} \\
		w^{n-3} + 2w^{n-2} + w^{n-1} \\
		w^n + 2w^{n-1} + w^{n-2} \\
		\end{bmatrix}
		= 
		\begin{bmatrix}
		w^{n-1} * ( 1+ 2 w + w^2) \\
		1 + 2w + w^2       \\
		w * (1 + 2w + w^2)       \\
		w^2 * (1 + 2w + w^2)       \\
		w^3 * (1 + 2w + w^2)       \\
		\vdots \\
		w^{n-4} * (1 + 2w + w^2)       \\
		w^{n-3} * (1 + 2w + w^2)       \\
		w^{n-2} * (1 + 2w + w^2)       \\
		\end{bmatrix} = \\
		& = (1 + 2w + w^2)
		\begin{bmatrix}
		w^{n-1}  \\
		1        \\
		w        \\
		w^2      \\
		w^3      \\
		\vdots   \\
		w^{n-4}  \\
		w^{n-3}  \\
		w^{n-2}  \\
		\end{bmatrix}
		= 
		w (w^{-1} + 2 + w)
		\begin{bmatrix}
		w^{n-1}  \\
		1        \\
		w        \\
		w^2      \\
		w^3      \\
		\vdots   \\
		w^{n-4}  \\
		w^{n-3}  \\
		w^{n-2}  \\
		\end{bmatrix}
		= 
		(w^{-1} + 2 + w)
		\begin{bmatrix}
		w^n      \\
		w        \\
		w^2      \\
		w^3      \\
		w^4      \\
		\vdots   \\
		w^{n-3}  \\
		w^{n-2}  \\
		w^{n-1}  \\
		\end{bmatrix} = \\
		& = 
		(w^{-1} + 2 + w)
		\begin{bmatrix}
		1      \\
		w        \\
		w^2      \\
		w^3      \\
		w^4      \\
		\vdots   \\
		w^{n-3}  \\
		w^{n-2}  \\
		w^{n-1}  \\
		\end{bmatrix}
	\end{split}
	\]
	\[
	 L_C 
	 \begin{bmatrix}
		1      \\
		w        \\
		w^2      \\
		w^3      \\
		w^4      \\
		\vdots   \\
		w^{n-3}  \\
		w^{n-2}  \\
		w^{n-1}  \\
		\end{bmatrix} 
	= 
	(w^{-1} + 2 + w)
		\begin{bmatrix}
		1      \\
		w        \\
		w^2      \\
		w^3      \\
		w^4      \\
		\vdots   \\
		w^{n-3}  \\
		w^{n-2}  \\
		w^{n-1}  \\
		\end{bmatrix}
	\]
	Sada je jasno da je $w^{-1} + 2 + w$ sopstvena vrijednost a $\vec{x}$ odgovarajuci sopstveni vektro.
	Jasno je da u skupu $\mathbb{R}$ jendacina $w^n = 1$ ima samo jedno resenje $w=1$. Medjutim nista nas ne sprecava 
    jednacinu ne rijesimo u skupu $\mathbb{C}$. U tom slucaju
	\[
	\begin{split}
	w^n  & = 1 \\
	(x + iy)^n & = 1 \\
	|w|(\cos\varphi + i \sin \varphi)^n & = 1
	\end{split}
	\]
	Iskoristimo Muavrovu formulu za stepenovanje kompleksnih brojeva u trigonometrijskom zapisu
	\[
		\begin{split}
			|w|^n (\cos n\varphi + i\sin n\varphi) & = 1 + i * 0 \\
			\big(|w|^n \cos n \varphi = 1 \big) \land \big((|w|)^n \sin n \varphi\big) & = 0  
		\end{split}
		\]
		Jasno je da $|w| \neq 0$ jer u tom slucaju $w=0$
	
	\[
	\begin{split}
		|w|^n  & = 0	\lor \sin n \varphi = 0 \\
		\sin n \varphi &= 2 k \pi \\
		&\varphi = \frac{2 k \pi}{n} \\
		& \varphi \in [0,2\pi] \Rightarrow k \in \{0,1, \dots n-1\}
	\end{split}
	\]
	\[
	\begin{split}
		|w|^n \cos n \varphi & = 1 \\
		|w|^n \cos n 2 k \pi & = 1 \\
		|w|^n  & = 1 \\
	\end{split}
	\]
	Kako $|w| \in \mathbb{R}$ dobijamo $|w|=1$ i
	\[
	\begin{split}
		w_k = |w| (\cos \varphi + i \sin \varphi) = \cos \frac{2 k \pi}{n} + i \sin \frac{2 k \pi}{n}
	\end{split}
	\]

	Vec smo utvrdili da su sve sopstvene vrijednosti Laplasove matrice realni brojevi, ali sada smo naizgled dobili kompleksan broj. Medjutim izracunajmo $\alpha_k$
	\[
	\begin{split}
		\alpha_k & = w_k^{-1} + 2 + w_k \\
		\alpha_k & = \cos \frac{2 k \pi}{n} + i \sin \frac{2 k \pi}{n} + 2 + \cos \frac{-2 k \pi}{n} + i \sin \frac{-2 k \pi}{n} \\
		\alpha_k & = (\cos \frac{2 k \pi}{n} + \cos \frac{-2 k \pi}{n}) + i(\sin \frac{2 k \pi}{n} + \sin \frac{-2 k \pi}{n}) + 2 \\
		\alpha_k & = (\cos \frac{2 k \pi}{n} + \cos \frac{2 k \pi}{n}) + i(\sin \frac{2 k \pi}{n} - \sin \frac{2 k \pi}{n}) + 2 \\
		\alpha_k & = 2\cos \frac{2 k \pi}{n} + 2 \\
		\alpha_k & = 2 (\cos \frac{2 k \pi}{n} + 1)
	\end{split}
	\]
	Sada je jasno da $\alpha_k \in \mathbb{R}$
	Razmotrimo sad $\alpha_{n-k}$
	\[
	\begin{split}
		\alpha_{n-k} & = 2 (\cos \frac{2 (n-k) \pi}{n} + 1) \\
		\alpha_{n-k} & = 2 (\cos \frac{2n\pi - 2k\pi}{n} + 1) \\
		\alpha_{n-k} & = 2 (\cos ( 2\pi - \frac{2k\pi}{n}) + 1) \\
		\alpha_{n-k} & = 2 (\cos \frac{2k\pi}{n} + 1) \\ 
		\alpha_{n-k} & = \alpha_k 
	\end{split}
	\]
	Odavde vidimo su sopstvene vrijednosti simetricne oko $n/2$, sto ukazuje da ih ima $\lceil \frac{n}{2}\rceil$ 
	Sopstveni vektor koji odgovara sopstvenoj vrijednosti $\alpha_k$ je dat sa 
	\[
	\begin{bmatrix}
	1 \\
	\cos \frac{2 k \pi}{n} + i \sin \frac{2 k \pi}{n} \\
	\cos \frac{4 k \pi}{n} + i \sin \frac{6 k \pi}{n} \\
	\vdots \\
	\cos \frac{(n-2) k \pi}{n} + i \sin \frac{(n-2) k \pi}{n} \\
	\cos \frac{(n-1) k \pi}{n} + i \sin \frac{(n-1) k \pi}{n} 
	\end{bmatrix}
	\]

	\end{example}

	\section{Interpretacija vektora i matrica u domenu grafova}

	Vec nam je do sada jasno da pomoocu pojedinih matrica na razlicite nacine mozemo predstaviti grafove, bilo da predsavljamo susjednost cvorova putem matrice
	susjedstva ili Laplasove matrice ili incidentnost cvorova i grana putem matrice susjedstva. Koja je uloga vektora? Koji je njihov smisao u domenu grafova?
	
	Vec smo vidjeli da ukoliko posmatramo graf sa n cvorova, vektori koje koristimo su dimenzije $n \times n$. Ti vektori se mogu posmatrati kao diskretne funkcije koje
	u pojedinim tackama imaju odgovarajuce vrijednosti. Te tacke u kojima je definisana vrijednosti funkcije su upravo cvorovi grafa. Drugim rijecima vektori dimenzije
	$n \times 1$ se mogu posmatrati kao funkcije koje svakom cvororu grafa dodjeljuju neku vrijednosti na nacina da vektor $\vec{x}$ cvoru grafa $G$ sa indeksom i dodjeljuje
	vrijednost $x_i^T$, tj. $\vec{x} = [ f(v_1) f(v_2) \dots f(v_n)]$. Vrijednost dodjeljena cvoru moze da ima razlicita znacenja, to moze biti njegova vaznost ili pak 
	vrijedost koja se koristi za klasifikaciju cvorova grafa. Kako su i sopstveni vektori Laplasove matrice takodje vektori dimenzij $n \times 1$, i oni se mogu posmatrati
	kao funkcije. Sopstveni vektori Laplsove matrice predstavjaju posebne funkcije koje se nazivaju sopstvene funkcije (eigenfunction).

	Cesto se u izrazima moze naci mnozenje matrice susjedstva ili Laplasove matrice sa nekim vektorom. Posmatrajmo cemu su jednaki ti proizvodi.
	\[
	A_G \vec{x} = 
	\begin{bmatrix}
		\sum_{\{v_1,v_j\} \in E} \vec{x}_1 \\
		\sum_{\{v_2,v_j\} \in E} \vec{x}_2 \\
		\vdots \\
		\sum_{\{v_n,v_j\} \in E} \vec{x}_n \\
	\end{bmatrix}_{n \times 1} =
	\begin{bmatrix}
		\sum_{\{v_1,v_j\} \in E} f(v_1) \\
		\sum_{\{v_2,v_j\} \in E} f(v_2) \\
		\vdots \\
		\sum_{\{v_n,v_j\} \in E} f(v_n) \\
	\end{bmatrix}_{n \times 1} =
	\]
	Sada je jasno da je rezultat vektor dimenzije $n \times 1$ kod kojeg koordinata sa indeksom i predstavlja sumu vrijednosti koje vektor $\vec{x}$ dodijeli svim
	susjedima cvora sa indeksom i.
	
	\[
		\begin{split}
		L_G \vec{x} & =
		\begin{bmatrix}
				deg(1) * \vec{x_1} - \sum_{\{v_1,v_j\} \in E} \vec{x}_j \\
			deg(2) * \vec{x_2} - \sum_{\{v_2,v_j\} \in E} \vec{x}_j \\
			\vdots \\
			deg(n) * \vec{x_n} - \sum_{\{v_n,v_j\} \in E} \vec{x}_j \\
		\end{bmatrix} =
		\begin{bmatrix}
			\sum_{\{v_1,v_j\} \in E} \vec{x}_1\vec{x}_j \\
			\sum_{\{v_2,v_j\} \in E} \vec{x}_2\vec{x}_j \\
			\vdots \\
			\sum_{\{v_2,v_j\} \in E} \vec{x}_n\vec{x}_j \\
		\end{bmatrix} \\
		& = 
		\begin{bmatrix}
			\sum_{ j \neq 1}  \vec{x}_1\vec{x}_j \\
			\sum_{ j \neq 2}  \vec{x}_2\vec{x}_j \\
			\vdots \\
			\sum_{ j \neq n}  \vec{x}_n\vec{x}_j \\
		\end{bmatrix} =
		\begin{bmatrix}
			\sum_{ j \neq 1}  f(v_1) - f(v_j) \\
			\sum_{ j \neq 2}  f(v_2) - f(v_j) \\
			\vdots \\
			\sum_{ j \neq n}  f(v_n) - f(v_j) \\
		\end{bmatrix} =
		\end{split}
	\]

	Rezultat mnozenja Laplsove matrice sa proizvoljnim vektorom dimenazije $n \times 1$ je ponovo vekor dimenzije $n \times 1$
	kod kojeg je vrijednost koordinate sa indeksom i u stvari suma razlika vrijednosti dodijeljene svoru sa indeksom i i njegovom susjedu.

	I domenu grafova, kao i u lineranoj alebri matrice nisu nista drugo do reprezentacije operatora. S toga, na osnovu ove prirode matrica i 
	na osnovu dobijenih izraza mozemo zakljuciti sljedece:
	Matrica susjedstva odgovara opertoru koji racuna sumu vrijednosti dodijeljenih svim susjedima cvorva grafa, tj okolini svakog cvora.
	Laplasova matrica odgovara operatoru koji koji racuna razliku izmedju vrijednosti dodijeljene svakom cvoru i vrijrdnosti dodiljenih
	cvorovima u njegov okolini. Ovaj operator pokazuje koliko vrijednosti koja funkcijxa $\vec{x}$ dodijeli nekom cvoru odstupa od vrijednsoti
	dodijeljenih u njegovoj okolini.

	Izraz $\sum_{ j \neq i}  f(v_i) - f(v_j)$ predstavlja ujedno i analogiju Laplasove matrice i istoimenog operatora. Naime, Laplasov operator
	$\nabla_2$ funkcije f u tacki p, pokazuje kolika je devijacija vrijednosti funkcije u tacki u odnosu na prosjecnu vrijednost u njenoj okolini.
	 Upravo to radi i Laplasova matrica sa vektora $\vec{x}$, tj. pokazuje koliko vrijdnost vektora $\vec{x}$ u cvoru v odstupa od vrijednosti
	vektora u njegovim susjednim cvorovima, ponasajuci se kao "diskretna" verzija Laplasovog operator.

	\section{Primjena spektralne teorije grafova}

	\begin{example}
		Posmatrajmo sljedeci graf G
		\begin{figure}[h]
			\centering
			\includegraphics[width=1\textwidth]{Figure_1.png}
		\end{figure}
		\[
		A_G =
		\begin{bmatrix}
			0  &  1  &  1  &  1  &  0  & 0  &  0  &  0  &  0  &  0\\
			1  &  0  &  0  &  1  &  0  & 0  &  0  &  0  &  0  &  0\\
			1  &  0  &  0  &  0  &  0  & 0  &  0  &  0  &  0  &  0\\
			1  &  1  &  0  &  0  &  0  & 0  &  0  &  0  &  0  &  0\\
			0  &  0  &  0  &  0  &  0  & 1  &  1  &  0  &  0  &  0\\
			0  &  0  &  0  &  0  &  1  & 0  &  1  &  0  &  0  &  0\\
			0  &  0  &  0  &  0  &  1  & 1  &  0  &  0  &  0  &  0\\
			0  &  0  &  0  &  0  &  0  & 0  &  0  &  0  &  1  &  0\\
			0  &  0  &  0  &  0  &  0  & 0  &  0  &  1  &  0  &  0\\
			0  &  0  &  0  &  0  &  0  & 0  &  0  &  0  &  0  &  0\\
		\end{bmatrix}
		\]

		Jasno je da se graf sastoji od 3 povezane komponente. Formirajmo njegovu Laplasovu matricu i oderedimo njen spektar.

		\[
		L_G =
		\begin{bmatrix}
		3   &  -1  &  -1  &  -1  &  0  &  0   &  0   &  0   &  0   &  0\\
		-1  &  2   &  0   &  -1  &  0  &  0   &  0   &  0   &  0   &  0\\
		-1  &  0   &  1   &  0   &  0  &  0   &  0   &  0   &  0   &  0\\
		-1  &  -1  &  0   &  2   &  0  &  0   &  0   &  0   &  0   &  0\\
		0   &  0   &  0   &  0   &  2  &  -1  &  -1  &  0   &  0   &  0\\
		0   &  0   &  0   &  0   &  -1 &  2   &  -1  &  0   &  0   &  0\\
		0   &  0   &  0   &  0   &  -1 &  -1  &  2   &  0   &  0   &  0\\
		0   &  0   &  0   &  0   &  0  &  0   &  0   &  1   &  -1  &  0\\
		0   &  0   &  0   &  0   &  0  &  0   &  0   &  -1  &  1   &  0\\
		0   &  0   &  0   &  0   &  0  &  0   &  0   &  0   &  0   &  0\\
		\end{bmatrix}
		\]
		Numerickim metodama i upotrbom racunara dolazimo do spektra ove matrice
		\[
		\begin{split}
			\alpha_1=0.0 \text{     } & \vec{e_1}^T = \begin{bmatrix} -0.5  &   -0.5  &  -0.5  &  -0.5  &  0  &  0  &  0  &  0  &  0  & 0      \end{bmatrix} \\ 
			\alpha_2=0.0 \text{     } & \vec{e_2}^T = \begin{bmatrix} 0  &   0  &  0  &  0  &  -0.58  &  -0.58  &  -0.58  &  0  &  0  & 0      \end{bmatrix} \\ 
			\alpha_3=0.0 \text{     } & \vec{e_3}^T = \begin{bmatrix} 0  &  0  &  0  &  0  &  0  &  0  &  0  &  0.71  &  0.71  &  0            \end{bmatrix} \\
			\alpha_4=0.0 \text{     } & \vec{e_4}^T = \begin{bmatrix} 0  &  0  &  0  &  0  &  0  &  0  &  0  &  0  &  0  &  1                  \end{bmatrix} \\
			\alpha_5=1.0 \text{     } & \vec{e_5}^T = \begin{bmatrix} 0  &  0.41  &  -0.82  &  0.41  &  0  &  0  &  0  &  0  &  0  &  0        \end{bmatrix} \\
			\alpha_6=2.0 \text{     } & \vec{e_6}^T = \begin{bmatrix} 0  &  0  &  0  &  0  &  0  &  0  &  0  &  0.71  &  -0.71  &  0           \end{bmatrix} \\
			\alpha_7=3.0 \text{     } & \vec{e_7}^T = \begin{bmatrix} 0  &  0.71  &  0  &  -0.71  &  0  &  0  &  0  &  0  &  0  &  0           \end{bmatrix} \\
			\alpha_8=3.0 \text{     } & \vec{e_8}^T = \begin{bmatrix} 0  &  0  &  0  &  0  &  0.82  &  -0.41  &  -0.41  &  0  &  0  &  0       \end{bmatrix} \\
			\alpha_9=3.0 \text{     } & \vec{e_9}^T = \begin{bmatrix} 0  &  0  &  0  &  0  &  0  &  0  &  0.29  &  -0.81  &  0.51  &  0        \end{bmatrix} \\
			\alpha_1=4.0 \text{    } & \vec{e_1}^T = \begin{bmatrix} 0.87  &  -0.29  &  -0.29  &  -0.29  &  0  &  0  &  0  &  0  &  0  &  0  \end{bmatrix}
		\end{split}
		\]
		Ukoliko sada u vidu grafa predstavimo sopstvene vektore koji odgovaraju sopstvenim vrijednosti 0, dobijamo
		\pagebreak
		\begin{figure}[h]
			\centering
			\includegraphics[width=0.8\textwidth]{plot.png}
		\end{figure}
		
		Za ocekivati je bilo da sopstvena vrijednost ima visestrukorst 4 jer se graf sastoji od 4 komponente povezanosti, ali zanimljivo je
		da svaki sopstveni vektor koji odgovara sopstvenoj vrijednosti nula istice sve cvorove koji pripadaju istoj komponenti povezanosti dodjeljujuci
		im istu vrijednost, a svim ostalim cvorovima vrijednost nula. Najmanja nenula sopstvena vrijednost je mala, sto ukazuje da graf nije cvrsto povezan, odnosno da je
        jako blizu da ima 5 povezanih komponenti. To se jasno moze postici uklanjanjem samo jedne grane.
	\end{example}

	
	\begin{example}
		Posmatrajmo sada drugi graf.
		\begin{figure}[h]
			\centering
			\includegraphics[width=0.8\textwidth]{Figure_2.png}
		\end{figure}
		Odogvarajuca Laplasova matrica je data sa
		\[ 
		L_G =
		\begin{bmatrix}
		3 & -1 & 0 & 0 & -1 & -1 & 0 & 0 \\
		-1 & 2 & -1 & 0 & 0 & 0 & 0 & 0 \\
		0 & -1 & 2 & -1 & 0 & 0 & 0 & 0 \\
		0 & 0 & -1 & 2 & -1 & 0 & 0 & 0 \\
		-1 & 0 & 0 & -1 & 2 & 0 & 0 & 0 \\
		-1 & 0 & 0 & 0 & 0 & 3 & -1 & -1 \\
		0 & 0 & 0 & 0 & 0 & -1 & 2 & -1 \\
		0 & 0 & 0 & 0 & 0 & -1 & -1 & 2
		\end{bmatrix}
		\]
        \[
            \begin{split}
                \alpha_1=0.0 \text{      } & \vec{e_1}^T = \begin{bmatrix} 0.35  &   0.35  &  0.35  &  0.35  &  0.35  &  0.35  &  0.35  &  0.35      \end{bmatrix} \\ 
                \alpha_2=0.31 \text{     } & \vec{e_2}^T = \begin{bmatrix} 0.06  &   0.26  &  0.37  &  0.37  &  0.26  &  -0.34  &  -0.49  &  -0.49  \end{bmatrix} \\ 
                \alpha_3=1.38 \text{     } & \vec{e_3}^T = \begin{bmatrix} 0  &  0.6  &  0.37  &  -0.37  &  -0.6  &  0  &  0  &  0                  \end{bmatrix} \\
                \alpha_4=1.67 \text{     } & \vec{e_4}^T = \begin{bmatrix} -0.56  &  -0,31  &  0,46  &  0,46  &  -0,31  &  -0,13  &  0,19  &  0,19   \end{bmatrix} \\
                \alpha_5=3 \text{        } & \vec{e_5}^T = \begin{bmatrix} 0  &  0  &  0  &  0  &  0  &  0  & -0,71  &  0,71                         \end{bmatrix} \\
                \alpha_6=3.33 \text{     } & \vec{e_6}^T = \begin{bmatrix} 0,34  &  -0,37  &  0,16  &  0,16  &  -0,37  &  0,64  &  -0,27  &  -0.27   \end{bmatrix} \\
                \alpha_7=3.62 \text{     } & \vec{e_7}^T = \begin{bmatrix} 0  &  0.37  &  -0.6  &  0.6  &  -0.37  &  0  &  0  &  0            \end{bmatrix} \\
                \alpha_8=4.69 \text{     } & \vec{e_8}^T = \begin{bmatrix} -0,67  &  0,28  &  -0,07  &  -0,07  &  0.28  &  0.58  &  0.16  &  -0,16        \end{bmatrix}
            \end{split}
            \]
            Kako je dati graf povezan bilo je i za ocekivati da sopstveni vektori koji odgovara sopstvenoj vrijednosti nula svakom cvoru dodjeljuje istu vrijednost.
            Zanimljiv je sopstveni vektor koji odgovara najmanjoj pozitivnoj sopstvenog vrijednosti. Ukoliko pogledamo vrijednosti koje on dodjeljuje cvorovima grafa, uvidjamo da su pozitivne 
            vrijednosti dodijeljene prvih pet cvorova. Ovo nam ukazuje da se prilikom kreiranja klastera, ovi cvorovi trebaju biti u istom klasteru.
	\end{example}


    \subsection
    Lema. Za graf G stepena d i njegovu najvecu sopstvenu vrijdnost Laplasove matrice vazi $\alpha_n \geq d$.

    Dokaz.

    Kako bismo dokazali navedenu lemu upotrijebicemo Courrant-Fischer teoremu

    \begin{theorem}[Courrant-Fischer]
        Neka je data simetricna kvadratna matrica $M_{n \times n}$. Njene sopstvene vrijednosti $\alpha_1 \leq \alpha_2 \leq \dots \alpha_{n-1} \geq \alpha_n$.
        mozemo izracunati putem sljedeceg izraza
        \[
            \alpha_i = \min_{\vec{x} \neq \vec{0} \land \vec{x} \perp \vec{x}_i \forall i \in \{1,2, \dots, i-1\}} \frac{\vec{x}^T M \vec{x}}{\vec{x}^T\vec{x}} 
        \]
        gdje je $x_i$ sopstveni vektor koji odogvara $\alpha_1$. 
    \end{theorem}

    Courrant-Fischer teorema na govori da se sopstvena vrijednost $\alpha_i$ moze dobiti kao minimum Rayleough qoutiten po svim vektorima 
    ortogonalnim sa sopstvenim vektorima koji odogvaraju manjim sopstvenim vrijednostima. 

    Koristeci navedenu teoremu i steceno znanje o kvadratnoj formi Laplasove matrice dobijamo
    \[
        \alpha_n = \max_{\vec{x} \neq \vec{0}} \frac{\vec{x}^T M \vec{x}}{\vec{x}^T\vec{x}} = \max_{\vec{x} \neq \vec{0}} \frac{ \sum_{\{u,v\} \in E} (\vec{x}_u - \vec{x}_v)^2 }{ \sum_{i \in \{1,2 \dots n\}} \vec{x}_i^2}
    \]
        
    Ukoliko sad uzmemo da je $\vec{x}$ jedan od vektora iz standardne baze, tj da vazi za neko $j \in \{1,2, \dots n\} \vec{x}_j = 1$, a za $i \neq j$ vazi $\vec{x}_i = 0$ dobijamo
    \[
        \alpha_n > \frac{ d (\vec{x}_j - 0)^2 }{1} = d \times 1  
    \]
    \[
        \alpha_n > d  
    \]

    Razmislimo sada kako bismo mogli utvrditi koji dio grafa je najmanje povezan sa ostatkom grafa. Idealno bi bilo kada bismo mogli definisati mjeru povezanosti podskupa cvorova S grafa G sa ostatkom cvorova.
    Jasno je da cemo posmatrati grane koje povezuju neki od cvorova podskupa sa cvorom van tog skupa. Na ovaj nacin dolazimo do izraza za provodnost grafa
    \[ \partial(S) = \{ \{v_1,v_2\}: v_1 \in S \land v_2 \in (V \setminus S)\} \]
    Provodnost skupa podskupa cvorova S grafa G predstavlja skup grana koje povezuju cvorove iz skupa S sa ostalim cvorovima.


    Kada bismo kao mjeru povezanosti podskupa svorova S uzeli kardinalost provodnosti, to zasigurno ne bi predstavljalo dobru mjeru, iz razloga sto kardinalost
    posmatrnog skupa utice i na kardinalnost provodnosti, tj. za ocekivati da veci skup cvorova S ima i vecu provodnosti. S toga, kao jedna od ideja namece se da izvrsimo normalizajiju sa brojem 
    cvorova u posmatranom skupu. Na taj nacin dolazimo da definicije izometrijskog odnosa skupa S.
    \[
        h(S) = \frac{|\partial(S)|}{|S|}
    \]
    Jasno je da manja vrijednost izometrijskog odnosa $h(S)$ ukazuje na vecu izolovanosti skupa cvorova S.
    Pored izometrijskog odnosa podskupa cvorova definise se i izomterijski odnosr grafa kao
    \[
         h(G) = \min_{|S| <= |V| / 2} h(S)
    \]
    Izometrijski odnos grafa G predstavlja izometrijski odnos njenogovj najizolovanijeg podgrafa.
    
    Bilo jos bolje kada bismo mogli utvrditi donju granicu za izometrijski odnos. Upravo to nam omogucava sljedeca teorema.

    \begin{theorem}[Lower bound of Cheegar]
        Neka je dat povezan graf $G=(V,E)$ i neka je $S \subset V$. Tada vazi
        \[ h(S) \geq \alpha_2 (1 - \frac{|S|}{|V|})\]
        gdje je $\alpha_2$ sopstvena vrijednost Laplasove matrice grafa.

        Jasno je da je za povezan graf ocito koji podskup cvorova je najizolovaniji.
    \end{theorem}

    \begin{custom_proof}
        Prema Courant-Fischer teoremi imamo
        \[
        \alpha_2 = \min_{\vec{x} \perp \vec{1}} \frac{\vec{x}^T L_G \vec{x}}{\vec{x}^T \vec{x}} 
        = \min_{\vec{x} \perp \vec{1}} \frac{ \sum_{\{u,v\} \in E} (\vec{x}_u - \vec{x}_v)^2}{\vec{x}^T \vec{x}}
        \]

        Posmatrajmo sada vektor cije su koordinate date Sa
        \[
            \vec{v} = 
            \begin{cases}
                1 - \frac{|S|}{|V|} & i \in S\\ 
                - \frac{|S|}{|V|} & i \in V \setminus S
            \end{cases}
        \]

        Motivacija za posmatranje ovog vektora je vise, prije svega ovaj vektor je ortogonalna jedinicnom vektoru.

        \[
        \begin{split}
            \vec{v}^T\vec{1} & = \sum_{i \in \{1,2, \dots n\}} \vec{v}_i * 1 \\
            & = \sum_{i \in \{1,2, \dots n\}} \vec{v}_i \\
            & = |S| * (  1 - \frac{|S|}{|V|} ) + ( -\frac{|S|}{|V|}) * |V \setminus S| \\
            & = |S| - |S| * \frac{|S|}{|V|} - \frac{|S|}{|V|} * |V \setminus S| \\
            & = |S| - \frac{|S|}{|V|} * (|V \setminus S| + |S|) \\
            & = |S| - \frac{|S|}{|V|} * |V| \\
            & = |S| - |S| = 0 
        \end{split}
        \]

        Uocima da je $\vec{v}_u - \vec{v}_v = 0$ ukoliko oba cvora pripadaju ili ne pripadaju skupu S, a da je $\vec{v}_u - \vec{v}_v = 1$ ukoliko tacno jedan od cvorova u i v pripada skupu S.
         
        \[
        \begin{split}
            \alpha_2  & \leq \frac{ \sum_{\{u,v\} \in E} (\vec{v}_u - \vec{v}_v)^2}{\vec{v}^T \vec{v}} \\
            \alpha_2  & \leq \frac{ |\partial(S)|}{\sum \vec{v}_i^2} 
        \end{split}
        \]
        
        Odredimo sada modul vektora v
        \[
            \begin{split}
                \vec{v}^T\vec{v} & = \sum_{i \in \{1,2, \dots n\}} v_i \\ 
                \vec{v}^T\vec{v} & = |S| * (1 - \frac{|S|}{|V|})^2 + |V \setminus S| (-\frac{|S|}{|V|})^2 \\ 
                \vec{v}^T\vec{v} & = |S| - 2 * |S| * \frac{|S|}{|V|} + |S| * (\frac{|S|}{|V|})^2 + |V \setminus S| (\frac{|S|}{|V|})^2 \\ 
                \vec{v}^T\vec{v} & = |S| - 2 * |S| * \frac{|S|}{|V|} + (|S| + |V \setminus S|) * (\frac{|S|}{|V|})^2 \\ 
                \vec{v}^T\vec{v} & = |S| - 2 * |S| * \frac{|S|}{|V|} + |V| * (\frac{|S|}{|V|})^2 \\ 
                \vec{v}^T\vec{v} & = |S| - 2 * \frac{|S|^2}{|V|} + (\frac{|S|^2}{|V|}) \\ 
                \vec{v}^T\vec{v} & = |S| - \frac{|S|^2}{|V|} \\ 
                \vec{v}^T\vec{v} & = |S| * ( 1 - \frac{|S|}{|V|}) \\ 
            \end{split}
            \]
            
            Nakon uvrstavanja dobijenog izraza, dobijamo
            \[
            \begin{split}
                \alpha_2  & \leq \frac{ \sum_{\{u,v\} \in E} (\vec{v}_u - \vec{v}_v)^2}{\vec{v}^T \vec{v}} \\
                \alpha_2  & \leq \frac{ |\partial(S)|}{|S| * ( 1 - \frac{|S|}{|V|})} \\ 
                \alpha_2 * |S| * ( 1 - \frac{|S|}{|V|}) & \leq |\partial(S)| \\
                \alpha_2 * ( 1 - \frac{|S|}{|V|}) & \leq \frac{|\partial(S)|}{|S|} \\ 
                \frac{|\partial(S)|}{|S|} & \geq  \alpha_2 * ( 1 - \frac{|S|}{|V|}) \\ 
                h(S) & \geq  \alpha_2 * ( 1 - \frac{|S|}{|V|}) \\ 
                h(G) & = \min_{|S| \leq |V| / 2}  \alpha_2 * ( 1 - \frac{|S|}{|V|}) \\ 
                h(G) & = \alpha_2  - \max_{|S| \leq |V| / 2}  \alpha_2 * \frac{|S|}{|V|} \\ 
                h(G) & = \alpha_2  - \alpha_2 * \frac{|V|/2}{|V|} \\ 
                h(G) & = \alpha_2  * (1 - 1/2) \\ 
                h(G) & = \alpha_2 / 2  
            \end{split}
            \]

            Dobijeni izraz je jos jedna potvrda cinjenice da najmanja pozitivna sopstvena vrijednost laplasove matrice pokazuje koliko dobro je graf povezan.
            Ukoliko je ova sopstvena vrijednmost veca, veci je i izomterijski odnos grafa, tj. njenogoj najizoolovanij dio je bolje povezan sa ostatkom grafa.

    \end{custom_proof}

    \subsection
    Primjena spektralne teoreme grafova

    Spektralna teorema grafova je jako zastupljena u raynim naucnim olastima, narocito u racunarskim naukama.
    Laplasova matrica i njen spektar se koristi za spektralno praticionisanja grafova prilikom segmentacije slike rasporednvanja tranzistora na VLSI cipovima. Prilikom klasifikacije dokumanta na sonocu semnatickih pveyanistu izmedju rijeci.
    Prilikon analize podataka i spektralnog uparivanja.
    
    Sopstveni vektori Laplasvoe matrice se jos i nazivaju i eigenfunctions.

    Fiedler value nad vector

    Graph cut problem se svodi na partcionisanje grafa tako da teyine grana unutar particije budu sto vece a da grane koje povezuju 2 particije budu sto manje

    Spectaln graph embedding on a line

    \begin{example}
        Regulatorna agencija na raspolaganju ima k radio frekvencija. Na konkurs za dodjelu frekvencije prijavilo se n radio stanica, pri cemu je $n \gg k$.
        Prostorno udaljenim radio stanicima ciji se reoni emitovanja ne poklapaju agencija moze dodijeliti istu frekvenciju ali to nije slucaj sa svim radio stanicama.
        Kako ce agencija odrediti kojoj radio stanici ce didijeliti koju radio frekvenciju?

        Kako bismo opravdali pojavljivanje ovoga zadatku u radu vezanom za grafove, naravno da cemo i ovaj put svesti problem na domen grafova.
        Radio stanice cemo predstavljati sa cvorovima a relaciju preklapanja reona emitovanja izmedju dvije stanice cemo modelovati sa granom izmedju cvorova.

        Vec na prvi pogled zadatak se svodi na bojenje ovako formiranog grafa sa k boja, odnosno dodjeljivanje boje (neke vrijednosti) svakom cvoru grafa tako da
        susjednim cvorovima nisu dodijeljene iste boje. Bojenje grafa nije nista drugo do odredjivanje funkcije koja slika skup cvorova u skup dostupnih boja $ c : V \to C$.
        Problem bojenja grafa sa k boja, nije uvijek rjesiv i zavisi od strukture grafa. Ukoliko je pak moguce obojiti dati graf sa k boja za graf G se
        kaze da je "k obojiv". Na osnovu definicije bojenja grafa lako se moze zakljuciti da graf nije "k obojiv" ukoliko bar jedan od cvorova ima vise od $k-1$ susjeda, tj.
        ukoliko vazi $\exists v \in V deg(v) \geq k-1$. Najmanje k za koje je G "k obojiv" naziva se hromatski broj grafa G u oznaci $\chi(G)$.

        Vec na pocetku, nismo sigurni ni da li je moguce rijesiti postavljeni zadatak, ali smanjimo kriterijum i zadovoljimo se time da nakon podjele frekvencija imamo sto manje preklapanja.
        
        Kao bismo rijesili postavljeni zadatak potrebno je odrediti vektor $\vec{c}$ takava da $\vec{c}_i$ ukazuje na to koju boju je potrebno dodijeliti radio stanici i.

        Prisjetimo se cemu je jednaka kvadratna forma Laplasove matrice formiranog grafa.
        \[
            \vec{x}^T L_G \vec{x} = \sum_{\{i,j\} \in E} (\vec{x}_i - \vec{x}_j)^2
        \]
        
        Ukoliko je pak $\vec{x}$ sopstveni vektor Laplasove matrice imamo da je 
        \[
            \vec{x}^T L_G \vec{x} = \sum_{\{i,j\} \in E} (\vec{x}_i - \vec{x}_j)^2 = \alpha
        \]
        
        Ukoliko odaberemo vektore koji odgovaraju najvecim sopstvenim vrijednostima Laplasove matrice, jasno je da u tom slucaju suma iz prethodnog izraza
        dostize svoj maksim, a upravo ona predstavlja sumu razlika izmedju vrijednosti dodijeljenih susjednim cvorovima. S toga, cvorovi kojima ovi sopstveni 
        vekori dodjeljuju bliske vrijednosti teze da budu sto "udaljeniji" u grafu.

        S toga, kao jedna od ideja namece spektralno particionaisanje ali upotrebom sopstvenih vektora koji odgovara 2-3 najvece sopstvene vrijednosti.
        Nakon izracunavanja sopstvenih vektora, njihove elemente je potrebno podijeliti u k grupa na osnovu njihove vrijednosti (upotrebom k-susjeda ili nekog drugog algoritma) i svakoj grupu dodijeliti boju.
        Element sopstvenog vektora $\vec{e}_i$ koji pripada j-toj grupi, kojoj je pridruzena boja $c_j$ ukazuje na to da bismo cvor $v_i$ trebali obojit sa $c_j$ bojom.

    \end{example}


	\subsection
	Spectal graph drawing	

	Pogledajmo kako se spektralna teorema moze iskoristiti za crtanje grafova. Jasno da je kod prikazivanja grafa sa mnostvom cvorova i grana problem predstavlja presijecanja grana, takodje jasno je i da je to uvecini slucajeva nemoguce izbjeci,
	iz razloga sto vecina grafova nije planarna. Medjutim, ukoliko bismo mogli da nadjemo nacin da predstavimo graf u ravan na nacin da se grane sto manje presjecaju. Kako bismo to postigli potrebno je da strateski pozicioniramo susjedne cvorove. 
	Pozicioniranjem susjednih cvorova tako da su sto blize jedan drugom, smanjujemo vjerovatnocu da ce se grane presjecati. Odnosno, ukoliko posmatramo priakaz grafa u ravni koordinatno sistema, susjedni cvorovoi treba da imaju bliske koodinate.
	Ovo na mvec zvuci poznato, jer se radi o minimizacije vrijednosti dodiljenih svakom cvori grafa (koordinata). Laplsova matroca i njen spektar se namece kao logicno rjesenja postavljenog problema.

	\begin{example}
		Pokusajmo u ravni prikazati ciklus sa 20 cvorova. Jasno da je idealan prikaz ovog grafa pravilan dvadesetougao.

		Laplasova matrica ciklusa sa 20 cvorova je data sa
		\[
			\begin{pmatrix}
				2 & -1 & 0 & 0 & \cdots & 0 & -1 \\
				-1 & 2 & -1 & 0 & \cdots & 0 & 0 \\
				0 & -1 & 2 & -1 & \cdots & 0 & 0 \\
				0 & 0 & -1 & 2 & \cdots & 0 & 0 \\
				\vdots & \vdots & \vdots & \vdots & \ddots & \vdots & \vdots \\
				0 & 0 & 0 & 0 & \cdots & 2 & -1 \\
				-1 & 0 & 0 & 0 & \cdots & -1 & 2
			  \end{pmatrix}	
		\]

		Spektar date laplasove matrice je dat sa
		\[
		\]

		Iscrtajmo sada graf u ravni tako da je x koordinata cvora data sa odgovarajucom vrijednosti sopstvenog vektora koji odgovara najmanjoj nenula sopstvenoj vrijednosti, a y koordinata sa odgovarajucom vrijednosti sopstvenog vektora koji odgovara drugoj najmanjoj nenula sopstvenoj vrijednosti.
		

		Zanimvljivo, na osnovu spektra Laplsove matrice uspjeli smo jako dobro prikazati dati graf u ravni. Da li je to slucajnost.

		Ne nije slucajnost, zato sto priroda sopstvenoh vektora Laplsove matrice jeste da teze da susjednim cvorovima, dodijele bliske vrijednosti, u ovom slucaju koordinate. Jasno je i da prilikom odredjivanja y koordinate punovo smo tezili da susjednim cvorovima dodjelimo bliske vrijednosti, ali da raspodjela y koordinata ne bude linearno zavisna raspodjeli x koordinata.
		Upravo to nam omogucava sopstveni vektor koji odgovara drugoj najmanjoj nenula sopstvenoj vrijednosti, jer je i on tezi istom cilju a pri tome je linearno nezavisan sa sopstvenim vektorom koji odgovara najmanjoj nenula sopstvenoj vrijednosti.

	\end{example}

	\end{document}
	